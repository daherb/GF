\documentclass[fontsize=12pt,abstract=on,titlepage,bibliography=totoc,ngerman,listof=totoc]{scrreprt}

%\usepackage[utf8]{inputenc}
\usepackage[onehalfspacing]{setspace}
\usepackage[ngerman]{babel}
\usepackage{fontspec}
\usepackage[a4paper, left=3cm, right=3cm, top=3cm]{geometry}
\usepackage{amsmath}
%\usepackage{qtree}
\usepackage{listings}
\usepackage{floatrow}
%\usepackage{capt-of}
\usepackage{fancyvrb}
\usepackage{graphicx}
\usepackage[section]{placeins}
\usepackage{url}
\usepackage{multirow}
\usepackage{arydshln}
\usepackage[style=alphabetic-verb,backend=biber]{biblatex}
\usepackage{rotating}
\setlength{\parindent}{10pt} 
\floatstyle{plain}

\newfloat{program}{thp}{lop}
\floatname{program}{Beispiel}

\renewcommand{\abstractname}{Zusammenfassung}

\urldef\persalatin\url{http://www.perseus.tufts.edu/hopper/text?doc=Perseus%3atext%3a1999.04.0059}
\urldef\perselemlat\url{http://www.perseus.tufts.edu/hopper/text?doc=Perseus%3atext%3a1999.04.0060}
\urldef\vatlatinitas\url{http://www.vatican.va/roman_curia/institutions_connected/latinitas/documents/rc_latinitas_20040601_lexicon_it.html}
\lstdefinelanguage{gf}
{
  morekeywords={abstract, flags, cat, fun, incomplete, concrete, of, open, in, lincat, lin, resource, param, oper, variants, table, interface, instance, def, data, lindef, printname,},
  sensitive=false,
  morecomment=[l]{--},
  morestring=[b]",
  stringstyle={\textit}
}
\lstset{language=gf,captionpos=b,numbers=left, numberstyle=\tiny, numbersep=5pt}

\bibliography{MA.ebib}
\begin{document}
\setcounter{tocdepth}{2}
\date{30.9.2013}
\makeatletter
\begin{titlepage}
{\sffamily
  \begin{center}
    \vspace{4cm}
    \begin{huge}
      Hausarbeit \\
      zur Erlangung des Magistergrades \\
      an der Ludwig-Maximilians-Universität München \\[3cm]
    \end{huge}
    {\Huge Erstellen einer Lateingrammatik im Grammatical Framework \\[4cm] }
    {\LARGE vorgelegt von Herbert Lange \\[4cm] }
  \end{center}
  \parindent0mm
  \begin{huge} 
    Fach: Computerlinguistik  \\[0.3cm]
    Referent: Prof. Dr. Klaus U. Schulz \\[0.3cm]
    München, den \@date 
  \end{huge}
}
\end{titlepage}
\begin{abstract}
In dieser Arbeit sollen an einem konkreten Beispiel die nötigen Schritte gezeigt werden, um eine computergestützte Grammatik für eine natürliche Sprache zu entwerfen. Am Beispiel der lateinischen Sprache wird gezeigt, wie ein Lexikon, ein Morphologiesystem und eine Syntax implementiert werden kann, die sich in ein größeres, multilinguales Grammatiksystem einfügen lässt. Dadurch können zum einen in der implementierten Sprache Sätze einfach geparsed werden, aber auch in jede andere im System vorhandene Sprache übersetzt werden können. Gezeigt wird ein rein regelbasierter Ansatz der sich von den statistischen Methoden durch seine Striktheit und Beschränktheit, aber auch durch seine Zuverlässigkeit abhebt.
\end{abstract}
\makeatother
\tableofcontents
%\pagebreak
\clearpage
\setcounter{page}{1}
\chapter{Einleitung}
\label{chap:einleitung}
\section{Motivation}
\label{sec:motivation}
Im Bereich der Computerlinguistik haben sich im Laufe der Zeit zwei Lager gebildet, die jeweils ihren Ansatz zur Sprachverarbeitung vertreten. Der heute häufiger anzutreffende Ansatz ist der statistische Ansatz, denn nach aktuellem Stand kann man durch statistische Methoden in der Sprachverarbeitung mit relativ geringem Aufwand brauchbare bis gute Ergebnisse erzielen. Allerdings bedarf der statistische Ansatz möglichst guter Trainingsdaten, die nicht immer leicht zu beschaffen und zu bewerten sind. \par
Der zweite, ältere Ansatz, ist die regelbasierte Sprachverarbeitung. Er prägte die Anfänge der Computerlinguistik stark, wurde jedoch im Laufe der Zeit vom statistischen Ansatz verdrängt. Dies ist unter anderem auf die steigende Leistung heutiger Rechner zur Verarbeitung großer Datenmengen und vor allem auf die Fülle an Daten, die über das Internet verfügbar sind, zurückführen. Die Grundlagen regelbasierter Grammatiken sind in etwa so alt wie die Wissenschaft der Linguistik selbst und sollte auch weiterhin zum Wissen eines jeden Computerlinguisten gehören. \par
In dieser Arbeit sollen an einem konkreten Beispiel die nötigen Schritte gezeigt werden, um eine computergestützte Grammatik für eine natürliche Sprache zu entwerfen. Am Beispiel der lateinischen Sprache wird gezeigt, wie ein Lexikon, ein Morphologiesystem und eine Syntax implementiert werden können, die sich in ein größeres, multilinguales Grammatiksystem einfügen lassen. Eine Sprache wie Latein, die zum einen für ihre Regelmäßigkeit aber auch für ihre linguistischen Besonderheiten bekannt ist, kann zu interessanten Erkenntnissen im Bereich der Grammatikentwicklung führen. \par
Ein weiterer Aspekt dieser Arbeit ist es, einen kleinen Beitrag zu einem größeren Projekt zu leisten. Denn die lateinische Grammatik, die im Rahmen dieser Arbeit entwickelt wurde, fließt direkt in das Grammatical Framework\footnote{\url{http://www.grammaticalframework.org/}}, das für diese Arbeit gewählten multilingualem Grammatik-, Parsing- und Übersetzungssystem ein.
\pagebreak
\section{Ziel der Arbeit}
\label{sec:ziel}
Ziel dieser Arbeit ist es, zum einen eine soweit funktionstüchtiges Grammatiksystem zu entwickeln, dass es in der Lage ist, grundlegende Sätze zu verarbeiten und durch die Integration in ein multilinguales Grammatiksystem in andere, moderne, Sprachen zu übersetzen. Zum anderen sollen aber auch die allgemeinen Schritte einer Grammatikentwicklung exemplarisch an der lateinischen Sprache dargelegt werden. \par
Der Schwerpunkt soll dabei zunächst auf der Nähe zu einer gewöhnlichen, im bayerischen Schulunterricht verwendeten, lateinischen Schulgrammatik liegen. Dies spiegelt sich in der Abfolge der Schritte und in einigen Entscheidungen beim Entwurf der Grammatik wieder. So ist eine lateinische Grammatik grob in folgende Abschnitte unterteilt: Phonologie, Wortarten und Wortbildung, Morphologie und Syntax. Zwar werden die ersten drei Teile in dieser Arbeit größtenteils vernachlässigt, die logische Folge der zwei verbleibenden Teile wird aber auch hier beibehalten. Allerdings wird ihnen der üblicherweise in Grammatiken nicht in dieser Form auffindbare Teil über die Lexikonentwicklung vorangestellt. \par
Die Grammatik soll zusätlich im bestehenden System des Grammatical Framework verwendet werden können. Deshalb müssen möglichst große Teile der für eine Sprache vom Grammmatical Framework geforderten Schnittstellen zur Verfügung gestellt werden. Diese Schnittstellen werden im Grammatical Framework in der Form einer sogenannten abstrakten Syntax definiert, die im folgenden noch genauer beschrieben wird. So wird eine Menge von Modulen, Funktionen und Regeln definiert. Das geplante Ziel ist es zu ermöglichen, dass alle von der Grammatik beschriebene Sätze in andere Sprachen, die untertsützt werden, übersetzt werden können.
\pagebreak
\section{Aufbau der Arbeit}
\label{sec:aufbau}
Zu Beginn der Arbeit werden in Kapitel \ref{chap:grundlagen} die nötigen Grundlagen für die weiteren Teile der Arbeit erörtert. Zunächst einmal werden die Grundlagen des Grammatical Framework in Abschnitt \ref{sec:gf} erklärt. Am Anfang steht eine grundlegende Beschreibung des Umfangs und der Funktionen dieses Programmpakets. Anschließend folgt eine Einführung in den Formalismus, der bei der Entwicklung der Grammatiken für das Grammatical Framework verwendet wird. Dieser Abschnitt wird mit einigen Informationen zur Ressource Grammar Library, der multilingualen Grammatikbibliothek des Grammatical Frameworks, abgeschlossen. Nach den technischen Grundlagen folgen einige Informationen zur lateinischen Sprache in Abschnitt \ref{sec:latein}. Zunächst erfolgt eine sprachwissenschaftliche Einordnung dieser Sprache unter besonderer Hervorhebung einiger interessanter Merkmale. Darauf folgt ein kurzer Abschnitt, der die Relevanz dieser als tot geltenden Sprache in der heutigen Zeit beleuchtet. \par
Nach dieser allgemeinen Einführung erfolgt im nächsten großen Teil, dem Kapitel \ref{chap:grammatik}, die Beschreibung der nötigen Schritte, die umgesetzt wurden, um eine Lateingrammatik im Grammatical Framework zu entwickeln. Es ist in die drei Abschnitte Lexikon, Morphologie und Syntax unterteilt, da diese drei getrennte Module in der Ressource Grammar Library bilden. Bei der Entwicklung der Grammatik zeigten sich allerdings oft Abhängigkeiten zwischen den drei Bestandteilen, so dass im Laufe der Zeit auch Änderungen in anderen Komponenten nötig waren. Im Abschnitt \ref{sec:lexikon} wird dargestellt, wie das Lexikon, das für eine Grammatik im Grammatical Framework nötig ist, erstellt wird. Darauf folgt in Abschnitt \ref{sec:morpho} die Beschreibung der lateinischen Wortflexion und wie sie im Grammatical Framework umgesetzt werden kann. Als letzter Teil dieses Kapitels wird in Abschnitt \ref{sec:syntax} erläutert, welche syntaktischen Regeln in der Grammatik nötig sind, um eine grundlegend funktionierende Grammatik zu erhalten, was ja zu den Hauptzielen dieser Arbeit gehört. \par
Abgerundet wird die Arbeit in Kapitel \ref{chap:fazit}, in dem ein Fazit der Arbeit gezogen und ein Ausblick auf Erweiterung und Verwendung gegeben wird. So wird gezeigt, welchen Sprachumfang die Grammatik bisher umfasst, welche Erweiterungen möglichst gewinnbringend sein können und auch in welchen Bereichen das Ergebnis dieser Arbeit Anwendung finden kann.
\section{Konventionen}
Zum Schluss dieser Einleitung sollen noch kurz die in dieser Arbeit verwendeten typografischen Konventionen erläutert werden. Grammatik-Quelltexte des Grammatical Frameworks, so wie Teile daraus, Befehle auf der Betriebssystem-Ein\-ga\-be\-auf\-for\-de\-rung und Befehle in einer interaktiven Sitzung des Grammatical Framework-Systems werden in einer \texttt{schreibmaschinenähnlichen Schrift} gesetzt. Befehle die in der Betriebssystem-Eingabeaufforderung einzugeben sind an einer Ein\-ga\-be\-auf\-for\-de\-rung in Form von \texttt{\$} vor dem Befehl und Befehle im Grammatical Framework-System an einem \texttt{>} zu erkennen. So ist \texttt{cat S;} ein Ausschnitt aus einem Quelltext, \texttt{\$ gf} ein Befehl in der Betriebssystem-Ein\-ga\-be\-auf\-for\-de\-rung und \texttt{> generate\_random} ein Befehl in einer interaktiven Grammatical Framework-Sitzung. \par
Alle Dateinamen und Pfade werden \textbf{fett} dargestellt. Ist bei einer Datei kein Pfad angegeben, so befindet sie sich im Verzeichnis, in dem die Lateingrammatik enthalten ist. Dieses Verzeichnis befindet sich entweder auf der beiliegenden CD oder im offiziellen Quelltext des Grammatical Frameworks im Unterverzeichnis \textbf{lib/src/latin}. Ist dagegen ein Ordnerpfad angegeben, so ist dieser Relativ zu einem Verzeichnis, in dem sich der Quelltext des Grammatical Framework befindet. In Quelltext-Listings werden zur einfacheren Lesbarkeit auch Schlüsselwörter zusätzlich \texttt{\textbf{fett}} gedruckt.\par
Lateinische Wörter werden zur leichteren Erkennbarkeit \textit{kursiv} gesetzt. Zusätzlich sind in Quelltext-Listings die Inhalte von Zeichenketten ebenfalls \texttt{\textit{kursiv}}. \par
Endet eine Zeile in einem Listing mit einem umgekehrten Schrägstrich ("`\textbackslash"') so bedeutet dies, dass lediglich aus Platzgründen ein Zeilenumbruch eingefügt werden musste und die nächste Zeile eine Fortführung der Zeile an dieser Stelle darstellt.
\pagebreak
\chapter{Grundlagen}
\label{chap:grundlagen}
\section{Das Grammatical Framework}
\label{sec:gf}
\input{tex/GF.tex}
\pagebreak
\FloatBarrier
\section{Die Lateinische Sprache}
\label{sec:latein}
\input{tex/Latein.tex}
\pagebreak
\chapter{Grammatikerstellung}
\label{chap:grammatik}
Nach der Einführung in die nötigen Grundlagen, um die folgenden Schritte zu verstehen, die nötig sind um im Grammatical Framework eine Grammatik zu entwickeln, folgt nun eine Schilderung der konkreten Schritte die nötig waren um eine Lateingrammatik im Grammatical Framework zu entwickeln. \par
Es sei noch anzumerken, dass es bereits früher Bestrebungen von Aarne Ranta gab, eine Lateingrammatik für die Ressource Grammar Library des Grammatical Frameworks zu entwickeln. Diese Arbeit baut auf der Arbeit Rantas auf, kann aber insofern als selbständige und vollwertige Arbeit angesehen werden, da die bisherige Implementierung sehr rudimentär und noch nicht funktionstüchtig war und seit ca. 2005 nicht mehr weiterentwickelt wurde. Im Anhang ist der Quelltext meiner Arbeit im Verhältnis zum Zustand vor Beginn der Arbeit zu finden. \par
Die Gliederung folgt dem gewählten Vorgehen bei der Implementierung. Die Begründung für die Reihenfolge der einzelnen Schritte wird jeweils zu Beginn der einzelnen Kapitel kurz dargelegt.
\pagebreak
\section{Lexikon}
\label{sec:lexikon}
Den Beginn dieser Grammatikimplementierung bildete die Erstellung des minimal nötigen Lexikons. Durch die abstrakte Syntax der der Ressource Grammar Library für Lexika\footnote{vgl. \textbf{Lexicon.gf} und \textbf{Structural.gf} im Ordner \textbf{lib/src/abstract}} ist eine Liste von etwas über 450 englischen Bezeichnern für Worte vorgegeben, die in jeder Sprache umgesetzt werden sollten. \par
Für die Erstellung eines Lexikon, wie es in einer Grammatik verwendet werden kann, sind zwei Schritte nötig. Einerseits müssen für jeden vorgegebenen Bezeichner, in diesem Falle alle lexikalischen Funktionen aus dem abstrakten Lexikon die zugeordneten Zeichenketten zugeordnet werden. Und zum anderen muss das Lexikon auch all jene Informationen enthalten, die später zum bilden der Vollformen und zur Konstruktion grammatischer Einheiten nötig sind. So z.B. bei Nomen das Geschlecht, wenn es nicht abgeleitet werden kann. \par
Der erste Schritt ist also, einfach das Lexikon einer anderen Sprache, in diesem Falle Englisch, zu kopieren. Normalerweise ist es vernünftiger, mit einer Sprache zu beginnen, die der zu implementierenden Sprache möglichst nahe steht.\footnote{vgl. \cite{RANTA2011} S. 224f.}. Allerdings wurde dieser Schritt bereits von Aarne Ranta dadurch begonnen die englischen lexikalischen Ressourcen zu kopieren und anzupassen. Das ist auch insofern verständlich, da für die verwendeten Bezeichner in der Grammar Library bereits die englischen Begriffe zusammen mit einem Marker für die Wortart gewählt wurden. Anschließend werden zunächst alle Zeichenketten durch mögliche lateinische Entsprechungen ersetzt. Um eine mögliche Übersetzung für die verschiedenen Lexikoneinträge zu finden, müssten teilweise verschiedene Vorgehensweisen bemüht werden. Hauptsächlich wurden hier, soweit möglich gedruckte Wörterbücher für die Übersetzung verwendet, gelegentlich waren aber auch Onlineresourcen unumgänglich. Des weiteren wird der Übersetzungsschritt von den englischen Bezeichnern zu den deutschen Entsprechungen, die für das weitere Vorgehen verwendet wurden, nicht genauer erläutert. Es sei nur so viel gesagt, dass die Bedeutung der meisten Bezeichner ohne weitere Hilfsmittel ersichtlich ist. Im Falle dessen, dass einmal Unklarheiten herrschten, wurde ein bekanntes Onlinewörterbuch\footnote{\url{http://dict.leo.org/}} verwendet.\par
\subsection{Wörterbücher}
\label{subsec:woerterbuch}
Um eine passende lateinische Übersetzung für die Lexikoneinträge zu finden, wurde primär der deutsch-lateinische Teil eines handelsüblichen Schulwörterbuchs (\cite{LANGENSCHEIDT1981}), soweit ein entsprechender Eintrag im diesem Wörterbuch zu finden war. Allerdings gibt es bereits an diesem Punkt diverse Herausforderungen. Denn eine Art von Wörtern, die allgemein zu Problemen bei der Übersetzung, und somit auch bei der Erstellung dieses Lexikons, führten, sind Wörter mit ambiger Bedeutung, oder auch homonyme Begriffe, wie das häufig als Beispiel angeführte Wort \textit{Bank} bzw. \textit{bank}, das in vielen Sprachen mehrere verschiedene Bedeutungen haben kann, z.B. im Deutschen als Sitzgelegenheit und als Geldinstitut oder im Englischen als Geldinstitut oder als Ufer eines Flusses.\footnote{vgl. \cite{METZLER2004} Homonymie: S. 3927} Für diesen und ähnliche Begriffe wurde willkürlich eine der plausiblen Bedeutungen gewählt, da keine Hinweise zur gewünschten Bedeutung in der Grammar Library gefunden werden konnte. Die Entscheidung eine einzige Bedeutung zu wählen, und nicht verschiedene Bedeutungen als Varianten des Wortes zu implementieren, muss getroffen werden um die Anzahl der möglichen Übersetzungen eines Ausdrucks möglichst gering zu halten. Für den Umgang mit ambigen Wörtern in einem Lexikon für das Grammatical Framework gibt es keine klaren Regeln, die angebrachteste Methode scheint aber zu sein, für jede Bedeutung einen eigenen Bezeichner zu wählen. So wäre möglicherweise in einem Lexikon \textit{bank1\_N} die Sitzgelegenheit und würde im englischen mit \textit{bench} übersetzt und \textit{bank2\_N} das Geldinstitut, das mit \textit{bank} übersetzt würde.\par
Ein weiteres Problem bei einer so alte Sprache wie Latein ist, dass bei vielen, meist moderneren Begriffen, nicht immer entsprechende Wörterbucheinträge gefunden werden können. Zwar gibt es auch andere Wörterbücher, wie das Schulwörterbuch von PONS (\cite{PONS2012}), das einen umfangreicheren lateinisch-deutschen Teil enthält, und somit mehr moderne Begriffe abdeckt, allerdings gibt es auch dort Begriffe, für die auch hier kein Eintrag zu finden ist. Für diesen Fall müssen neben den bewährten gedruckten Wörterbüchern auch andere Quellen, vor allem Onlinequellen zu Rate gezogen werden. Einige davon werden im Folgenden kurz gezeigt.\par
\subsection{Onlinequellen}
\label{subsec:online}
Als mögliche Lösung bei der Suche nach Übersetzungen, die im Wörterbuch nicht zu finden sind, bietet sich die Nutzung von, meist  kollaborativen, Internetquellen an. Eine der interessantes Quelle für moderne Begriffe aus dem Bereich der Substantive ist wohl die lateinische Wikipedia\footnote{\url{http://la.wikipedia.org/wiki/Pagina\_prima}}. Obwohl Latein als tote Sprache gilt, existieren dort über 90000 lateinische Artikel\footnote{\url{http://la.wikipedia.org/wiki/Specialis:Census}; Stand: 30.7.2013}, die von einer recht Lebendigen Gemeinschaft gepflegt werden. Natürlich muss man immer bedenken, dass es keine Garantie für die Qualität von kollaborativen Onlinequellen gibt. Allerdings hat sich das Prinzip der Wikipedia ja auch in anderen Sprachen bewährt, wenn auch die Qualitätssicherung durch manuelle Korrekturen, und damit auch die Qualität der einzelnen Artikel, direkt von der Größe der an dem Projekt arbeitenden Community zusammenhängt. Neben der Wikipedia, die vom Konzept her eigentlich eine allgemeine Enzyklopädie ist, und nur im Nebeneffekt linguistische Ressourcen zur Verfügung stellt, gibt es noch weitere Internetquellen, die bei der Erstellung eines Lexikons helfen können. So gibt es das deutsche Lateinportal Auxilium-online.net\footnote{\url{http://www.auxilium-online.net/}}, das englischsprachige Wiktionary\footnote{\url{http://en.wiktionary.org/}} und die Lateinressourcen bei der Perseus Digital Libary\footnote{\url{http://www.perseus.tufts.edu/hopper/}}. \par
Das erstere bezeichnet sich selbst als das größte deutschsprachige Lateinportal im Internet und bietet ein kostenloses Onlinewörterbuch, sowohl in der Richtung Lateinisch-Deutsch als auch in umgekehrter Richtung, das von registrierten Benutzern erweitert und korrigiert werden kann. Allerdings liegt bei diesem Wörterbuch der Schwerpunkt auch eher auf dem klassischen Vokabular. \par
Das englischsprachige Wiktionary hilft zwar nicht direkt bei der Suche nach einer direkten Übersetzung aus einer anderen Sprache, es bietet aber für ein umfangreiches Vokabular sowohl eine morphologische Analyse für viele Wortformen als auch detaillierte Informationen über Verwendung und Formenbildung für lateinische Vokabeln. \par
Die Perseus Digital Library, und vor allem die darin enthaltenen Wörterbücher, fallen eher in die Kategorie klassischer, gedruckter Wörterbücher, was primär daher rührt, dass diese Wörterbücher Digitalisate seit Jahrzehnten bewährter Wörterbücher sind.\footnote{\textit{A Latin Dictionary. Founded on Andrews' edition of Freund's Latin dictionary. revised, enlarged, and in great part rewritten by. Charlton T. Lewis, Ph.D. and. Charles Short, LL.D. Oxford. Clarendon Press. 1879.} (\persalatin) und \textit{Lewis, Charlton, T. An Elementary Latin Dictionary. New York, Cincinnati, and Chicago. American Book Company. 1890.} (\perselemlat)} Jedoch bietet Perseus die Möglichkeit einer erweiterten Suchfunktion so wie einer Angabe zur Wortfrequenz im verfügbaren Korpus. \par
Eine der Onlinequellen für moderne lateinische Begriffe wurde nicht verwendet, da sie nur zwischen Latein und Italienisch übersetzt. Dies würde aus verschiedenen Gründen zu Problemen führen. Diese Quelle soll allerdings trotzdem kurz erwähnt werden, denn sie ist die offizielle Liste des Vatikans zur Übersetzung moderner Alltagsbegriffe\footnote{\vatlatinitas}.
\subsection{Geschlossene Kategorien}
\label{subsec:geschlossene}
Das Lexikon einer Ressource Grammar ist unterteilt in zwei Dateien. Die erste Datei, \textbf{StructuralLat.gf}, enthält die Einträge für die so genannten geschlossenen Kategorien, so wie einige weiter Einträge die eher eine strukturelle als eine lexikalische Bedeutung haben. Die meisten Wortarten in diesem Teil des Lexikons gehören zu den so genannten Partikeln, die nicht flektiert werden. Dazu gehören vor allem Adverbien, Präpositionen und Konjunktionen.\footnote{vgl. \cite{BAYER-LINDAUER1994} S.12} \par
Adverbien gehören eigentlich nicht wirklich zu den geschlossenen Kategorien, jedoch gibt es eine gewisse Anzahl von Adverbien und adverbial benutzten Wörtern, die den meisten Sprachen gemein sind, weswegen sie als strukturelle Bestandteile aufgefasst werden können. Meist werden Adverbien aus Adjektiven gebildet, weswegen man sie zu den offenen Kategorien rechnen sollte. Jedoch ist dies nur eine von verschiedenen Möglichkeiten zur Verwendung von Adverbien. Vor allem im  Bereich der lokalen Adverbien (auf die Fragen wo?, wohin?, woher?) sowie vergleichende Adverbien gibt es nur ein eingeschränktes Vokabular, das zu Recht zu den geschlossenen Kategorien gerechnet werden kann.\footnote{vgl. \cite{BAYER-LINDAUER1994} S.44} 
Konkret als \texttt{Adv}\footnote{verb-phrase-modifying adverb vgl. \cite{RANTA2011} S. 298} gekennzeichnet, sind im Falle der Ressource Grammar Library die Bezeichner \texttt{everywhere\_Adv},\texttt{here\_Adv},\texttt{here7to\_Adv}, \texttt{here7from\_Adv}, \texttt{somewhere\_Adv}, \texttt{there\_Adv}, \texttt{there7to\_Adv} und \texttt{there7from\_Adv}. Betrachtet man die Übersetzung dieser Bezeichner, so stellt sich heraus, dass die lateinischen Wörter \textit{ubique}, \textit{hic}, \textit{huc}, \textit{hinc}, \textit{usquam}, \textit{ibi}, \textit{eo} und \textit{inde} in der Lateingrammatik nicht als Adverbien, sondern als Pronominaladverbien, aufgeführt werden, also eher zur geschlossenen Kategorie der Pronomen, allerdings mit adverbialer Verwendung gehören. \par
Zur selben grammatischen Kategorie gehören die meisten der im Grammatical Framework als \texttt{IAdv}\footnote{interrogative adverb ebd.} bezeichneten Vokabeln \texttt{how\_IAdv} (lat. \textit{qui}), \texttt{when\_IAdv} (lat. \textit{quando}) und \texttt{where\_IAdv} (lat. \textit{ubi}). Das Wort \texttt{how8much\_IAdv} (lat. \textit{quantum}) wird als korrellatives Pronomen bezeichnet, lediglich das Fragewort \texttt{why\_IAdv} (lat. \textit{cur}) ist in der gegebenen Grammatik nicht explizit eingeordnet, hat aber offensichtlich eine verwandtschaftliche Beziehung zu (Interrogativ-)Pronomen\footnote {vgl. wer? -  lat. \textit{quis}, Dat. \textit{cur}}. \par
Die Einträge für die eben genannten Kategorien sind allerdings recht einfach, da sie meist nur die Zeichenkette mit einer einzigen Form enthalten. Anders verhält es sich bei den Interogativpronomen (\texttt{IP}) und Interrogativquantifikatoren. Denn diese bilden im Fall der Interrogativpronomen kasusabhängige Formen, im Falle der Quantifikatoren sind die Formen zusätzlich von Genus und Numerus abhängig. Da es jedoch nur wenige Einträge dieser Kategorien gibt, ist es nicht rentabel für sie eine zentrale, morphologische Funktion zu definieren. Deshalb müssen alle Formen direkt im Lexikon gelistet werden. \par
Nichts direkt mit Pronomen zu tun haben die folgenden vergleichenden Adverbien. Um genau zu sein sind es Ausdrücke von zwei Adverbien, die zusammen ein Verhältnis zwischen zwei Objekten ausdrücken, zwischen welche sie gesetzt werden. So drückt \texttt{less\_CAdv} (lat. \textit{minus ... quam}) aus, dass etwas kleiner oder geringer ist als etwas anderes, und \texttt{more\_CAdv} (lat. \textit{magis quam}) dagegen drückt aus, dass etwas größer ist. Dagegen drückt \texttt{as\_CAdv} (lat. \textit{ita ... ut}) die Gleichheit aus. Objekte dieser Kategorie werden mit der Funktion \texttt{mkCAdv} aus den beiden Zeichenketten erstellt. \par
Eine weitere recht interessante Kategorie ist die Kategorie des Determinans. Diese werden meist auf Basis von Adjektiven gebildet. Dadurch ist es möglich mit einer einzigen Wortform auszukommen, um alle nötigen Formen zu bilden. \par
Die letzte hier zu erwähnende, einfache Kategorie von Wörtern sind Präpositionen. Präpositionen werden gemeinhin verwendet um die Funktion verschiedener Kasus genauer zu spezifizieren. Deshalb gibt es zu jeder Präposition zwingend auch eine Angabe, mit welchem Kasus sie verwendet werden kann.\footnote{vgl. \cite{BAYER-LINDAUER1994} S. 160f.} Allerdings haben die Kasus im Lateinischen bereits eine relativ feste Funktion, die in anderen Sprachen durch Präpositionen zusammen mit einem entsprechenden Kasus ausgedrückt werden. Deshalb gibt es in Latein auch einige ``leere'' Präpositionen, die also keine Zeichenkette produzieren, aber die Verwendung eines bestimmten Falles erzwingen. Zu diesen Präpositionen gehören unter anderem \texttt{part\_Prep} und \texttt{posses\_Prep}, deren Bedeutung schon allein durch einen Genitiv ausgedrückt wird. Andere, recht häufige, Präpositionen wie \textit{in} können dagegen auch mit mehreren Fällen benutzt werden. Dies wird in GF durch so genannte freie Variationen ermöglicht (Listing \ref{GF-Structural-in}). So kann \textit{in} sowohl zusammen mit Akkusativ oder Ablativ gebraucht werden.
\begin{lstlisting}[float=ht,label={GF-Structural-in},caption={Beispiel für freie Variation}]
in_Prep = mkPrep "in" ( variants { Abl ; Acc } ) ; -- (Langenscheidts)
\end{lstlisting}
Kurz zu erwähnen sind auch Konjunktionen. Denn sie bestehen nicht nur aus einer einzelnen Zeichenkette, sondern aus einem Verbund aus zwei Zeichenketten, denn es gibt Konjunktionen wie z.B im Deutschen ``sowohl ... als auch ...'', die aus zwei teilen Bestehen, von denen der erste vor die erste zu verbindene Phrase gesetzt wird und der zweite zwischen die Teile. Dieser Verbundtyp wird durch eine Hilfsfunktion namesn \texttt{sd2} erzeugt, die allgemein in der Datei \textbf{Prelude.gf}\footnote{im Ordner \textbf{lib/src/prelude}} definiert ist. \par
Es gibt in diesem Teil des Lexikons noch einige weitere einfache Kategorien wie Satzbeginnende Konjunktionen, deren Einträge allerdings so selbsterklärend sind, dass sie hier nicht gesondert aufgeführt werden müssen.
Allerdings sind hier auch einige komplette Nominalphrasen enthalten. Viele davon werden wieder durch Pronomen ausgedrückt, wie z.B. \texttt{everybody\_NP}, \texttt{somebody\_NP}, \texttt{something\_NP}, \texttt{nobody\_NP} und \texttt{nothing\_NP}. Diese werden allgemein durch Indefinitpronomina ausgedrückt. Allerdings müssen sie unter Berücksichtigung aller Kasus-Formen, dem Genus und dem Numerus mit Hilfe der Funktion \texttt{regNP} in die Form einer Nominalphrase gebracht werden. Lediglich \texttt{everything\_NP} wird passender durch das Nomen \textit{omnis} im Plural ausgedrückt. Auch hier kommt die Funktion \texttt{regNP} unter Angabe aller Kasus-Formen zum Einsatz.
\FloatBarrier
\subsection{Offene Kategorien}
\label{subsec:offene}
Das Lexikon der offenen Kategorien, \textbf{LexiconLat.gf}, enthält eine kleine Anzahl aus Wörtern aus den so genannten offenen Kategorien, vornehmlich Nomen, Verben und Adjektive. Die Einträge haben unterschiedliche Umfang. Dies ist zum einen abhängig von der Menge an Informationen, die nötig ist um das gesamte Paradigma des generieren. Wovon dies abhängig ist wird im Kapitel über die Morphologie genau beschrieben. Im allgemeinen reicht, bei regelmäßiger Deklination\footnote{Nomenflexion} und Konjugation\footnote{Verbflexion}, eine einzelne Wortform aus um daraus das gesamte Paradigma, also die Menge aller Wortformen abhängig von den variablen Merkmalen, zu erzeugen. \par
Deshalb ist es bei den Nomen der ersten, zweiten, vierten und fünften meist nicht nötig weitere Informationen anzugeben als die Nominativ-Singular-Form. Allerdings gibt es Ausnahmen, z.B. wenn bei einem Wort das Genus vom üblichen Geschlecht abweicht, das normalerweise mit der entsprechenden Endung kodiert wird. Also ist sowohl für Nomen der dritten Deklination, so wie für Nomen der anderen Deklinationsklassen nötig, statt einer Wortform zwei Wortformen, den Nominativ und Genitiv Singular, und das Geschlecht anzugeben. Bei einigen Nomen, wie z.B. Bezeichnungen für Tiere, kann das entsprechende Nomen bei gleicher Wortform beide Genera annehmen. Deshalb wird in diesem Falle die Funktion zum Erzeugen des Paradigmas mit freier Variation über die möglichen Geschlechter, meist Femininum und Maskulinum, versehen (vgl. Listing \ref{GF-Structural-in}). Andere Nomen haben dagegen sowohl eine männliche als auch eine weibliche Form, wobei diese meist sehr klar den üblicherweise entsprechenden Deklinationsklassen entsprechen. So gibt es im englischen nur ein geschlechtsunspezifisches Nomen für Cousin und Cousine. Des halb heißt das entsprechende Symbol \texttt{cousin\_N} ausgedrückt. Die lateinische Übersetzung ist aber \textit{consobrinus} für den männlichen Cousin und \textit{consobrina} für das weibliche Pendant. Auch in diesem Falle kann die freie Variation, wenn auch auf einem höheren Level eingesetzt werden. Es wird nicht nur der Wert eines Parameters variiert, sondern über zwei verschiedene, komplette Nomen-Objekte. \par 
Ein besonderer Nomeneintrag ist allerdings der Eintrag für \texttt{camera\_N}, denn die Übersetzung dieses Begriffs im Lateinischen kann nicht durch einen einzelnen Begriff ausgedrückt werden. Stattdessen wird es mit \textit{camera photographica} paraphrasiert. Deshalb muss für diesen Ausdruck zunächst einmal die Phrase aus ihren Bestandteilen konstruiert werden bevor sie in die Form eines einfachen Nomens gebracht wird. Dazu werden sowohl Syntaxfunktionen verwendet, die im entsprechenden Abschnitt beschrieben werden, um die Phrase zu erzeugen, als auch eine Hilfsfunktion \texttt{useCNasN}, die die Phrase in die Form eines einfachen Nomens bringt. \par
Eine weitere Form speziellerer Nomen sind Nomen, die nur im Plural vorkommen können. Dies wird im Lexikoneintrag dadurch kodiert, dass das Nomen durch eine weitere Funktion namens \texttt{pluralN} gefiltert wird. Die genauere Bedeutung dieser Funktion wird im Laufe der Morphologie genauer geschildert. Ein solches Nomen ist \texttt{science\_N} das mit \textit{literae}, der Pluralform von \textit{litera} (Buchstabe), übersetzt wird. \par
Zusätzlich zu den einfachen Nomen gibt es Relationalnomen, die eine Beziehung zwischen Objekten ausdrücken. Ein Beispiel hierfür ist das Wort Vater, das neben seiner einfachen Verwendung auch die Verwendung im Sinne ``Vater von ...'' haben kann. Deshalb benötigen diese Nomen neben ihrer einfachen Wortform auch die Information, wie diese Beziehung zu anderen Objekten ausgedrückt werden kann. Dies wird allgemein durch Pronomen kodiert, das heißt diese Nomen haben in ihrem Lexikoneintrag zusätzlich zu den Informationen, die nötig sind das Paradigma zu generieren, auch die Informationen zur Verwendung in Form der Angabe eines oder mehrerer Pronomens. \par
Nun bleiben noch einige der moderneren Begriffe aus dem Bereich der Nomen zu nennen, nämlich \texttt{airplane\_N}, \texttt{bank\_N}, \texttt{bike\_N}, \texttt{car\_N}, \texttt{carpet\_N}, \texttt{computer\_N}, \texttt{fridge\_N}, \texttt{paper\_N}, \texttt{planet\_N}, \texttt{plastic\_N}, \texttt{radio\_N}, \texttt{train\_N} und noch einige mehr. Die Übersetzungen dieser Begriffe wurden nach geschilderter Methode erfolgreich gefunden. Abgesehen vom Auffinden einer möglichen Übersetzung ist die Bildung allerdings relativ unproblematisch. \par
Die Einträge für Adjektive in diesem Lexikon sind alles in allem unproblematisch. Adjektive der ersten und zweiten Deklination können wieder aus einer einzigen Wortform, der maskulinen Nominativ-Singular-Form, erstellt werden. Lediglich bei Adjektiven der dritten Deklination wird zusätzlich die Genitiv-Singular-Form benötigt. Es gibt auch im Bereich des Lexikons kein Adjektiv dessen Übersetzung in irgendeiner Form problematisch gewesen wäre. \par
Bei den regelmäßigen Verben der ersten, zweiten und vierten Konjugation ist die einzige benötigte Information im Lexikon die Infinitiv-Präsens-Form. Dafür werden bei unregelmäßigen Verben und Verben der dritten Konjugation, wenn vorhanden, vier Verbformen benötigt. Dazu gehören neben dem Infinitiv die 1. Person Präsens Indikativ Aktiv, die 1. Person Perfekt Indikativ Aktiv und das Partizip Perfekt Passiv. \par
Die zwei einzigen Verben, deren Übersetzung problematisch war, sind \texttt{switch8off\_V2} und \texttt{switch8on\_V2}. Da hier auch nicht die Wikipedia von Hilfe sein konnte, wurden zwei nahe liegende lateinische Begriffe gewählt, deren Bedeutung näherungsweise passend erschienen, nämlich \textit{exstinguere} (löschen) und \textit{accendere} (entzünden). Obwohl es keine direkten Übersetzungen sind, ist die Verwendung insofern gerechtfertigt, dass das entsprechende italienische Wort für ``einschalten'' auch \textit{accendere} sein kann und \textit{exstinguere} das Gegenteil davon ist. \par
Hiermit sollten alle problematischen oder interessanten Aspekte des Lexikons erwähnt worden sein. Wie die Paradigmen aus diesen Lexikoneinträgen erzeugt werden können, wird im kommenden Abschnitt über die lateinische Morphologie ausführlich erläutert.
\FloatBarrier
\pagebreak
\section{Morphologie}
\label{sec:morpho}
Eine der großen Herausforderung bei der Implementierung einer Lateingrammatik, ist die Morphologie. Dies bedingt daraus, dass Latein eine flektierende Sprache ist, und deshalb viele grammatische Merkmale in der Wortform kodiert. Dadurch bedingt ist eine große Menge an Wortformen für jeden Lexikoneintrag. Um so wichtiger ist es, möglichst viele dieser Formen mit möglichst wenig Informationen zu generieren. Deshalb ist es ratsam, das Konzept der so genannten Smart Paradigms zu implementieren. Dabei wird versucht Mit Hilfe von Stringanalysen möglichst viele Informationen zur Wortbildung zu extrahieren. Im Falle der lateinischen Sprache werden dabei Wortsuffixe zu Rate gezogen. Die Implementierung der Morphologie ist hauptsächlich in der Quelltextdatei \textbf{MorphoLat.gf} zu finden, wobei die Konstruktion der konkreten Datenstrukturen, und damit auch ein Teil der Morphologie, in der Datei \textbf{ResLat.gf} zu finden ist.
\begin{lstlisting}[float=ht,caption={Beispiel für ein Smart Paradigm mit Hilfe von Pattern matching},label={GF-Morpho-Noun}]
noun : Str -> Noun = \verbum -> 
case verbum of {
  _ + "a"  => noun1 verbum ;
  _ + "us" => noun2us verbum ;
  _ + "um" => noun2um verbum ;
  _ + ( "er" | "ir" ) => noun2er verbum ( (Predef.tk 2 verbum) + "ri" ) ;
  _ + "u"  => noun4u verbum ;
  _ + "es" => noun5 verbum ;
  _  
    => Predef.error ("3rd declinsion cannot be applied " ++ 
                     "to just one noun form " ++ verbum)
  } ;
\end{lstlisting}
\subsection{Nomenflexion}
\label{subsec:nomen}
\subsubsection{Reguläre Nomenformen}
In der lateinischen Sprache gibt es fünf Deklinationsklassen für Nomen. Sie werden entweder durchnummeriert oder aber durch ihren Kennlaut bestimmt. Demnach unterscheidet man die erste bis fünfte Deklination bzw. die ā-, ǒ-, ǐ-, ǔ- und ē-Deklination. Den zur Identifikation kann man den Kennlaut am leichtesten nach Abtrennung der Endung \textit{-um} im Genitiv Plural erkennen. \footnote{vgl. \cite{BAYER-LINDAUER1994} S. 21}\par
Allerdings kann man meist die Deklinationsklasse auch an der Endung der Nominativ Singular Form erkennen. So haben z.B alle Nomen der ā-Deklination den Ausgang \textit{-ǎ} und Genus Femininum. Es gibt keine wirklich relevanten Ausnahmen, so können lediglich Flußnamen und männliche Personennamen männliches Geschlecht haben. Deshalb ist es bei fast allen Nomen dieser Deklinationsklasse nicht nötig mehr als die Nominativ Singular Form anzugeben. \par
Bei der zweiten Deklinationsklasse gibt es eine größere Anzahl möglicher Wortausgänge, nämlich \textit{-us}, \textit{-um} und \textit{-er} bzw. \textit{-ir}. Grundsätzlich sind Nomen mit dem Ausgang \textit{-um} Neutra, Nomen mit den Endungen \textit{-us} und \textit{-r} Maskulina. \par
Die dritte oder auch ǐ-Deklination wird auch als Mischdeklination bezeichnet, da sie in zwei Unterklassen unterteilt werden kann, in Nomen mit konsonantischem oder vokalischem, also auf \textit{-ǐ} auslautendem Stamm. Auf Grund dessen gehört die dritte Deklination zu den schwerer zu handhabenden Flexionsklassen. In Folge dessen reicht auch nicht eine einzige Wortform für die Generierung des Paradigmas aus. \par
Die vierte Deklinationsklasse hingegen ist wieder unkomplizierter. Sie hat im Nominativ Singular die Endungen \textit{-ū} oder \textit{-us} und die Nomen sind, wenn sie auf \textit{-us} enden, maskulin und, wenn sie auf \textit{-ū} enden, Neutra. Da die Nominativ Singular Form bei den Nomen auf \textit{-us} nicht von Nomen der zweiten Deklination mit der gleichen Endung zu unterscheiden sind, kann das Smart Paradigm für nur eine Wortform nur bei den Nomen auf \textit{-ū} angewandt werden, da die Endung \textit{-us} schon die zweite Deklination identifiziert. In diesem Falle wird ebenfalls das Paradigma mit Hilfe den drei Parameter Nominativ Singular, Genitiv Singular und Geschlecht bestimmt. \par
Bei der fünften Deklination ist die Nominativ Singular-Endung an sich wieder eindeutig, sie enden alle auf \textit{-es}. Jedoch können, wie oben bereits beschrieben, auch Nomen der dritten Deklination im Nominativ Singular auf \textit{-es} enden. Man kann die unterschiedlichen Deklinationen aber klar an der Genitiv Singular-Form unterscheiden. Deshalb ist die sicherste Möglichkeit Fehler zu vermeiden, auch diese Genitivform im Lexikon anzugeben. Dies ist jedoch nicht nötig, da wenn nur die Nominativform angegeben ist und diese auf \textit{-es} endet, das Smart Paradigm so definiert ist, dass ein Paradigma der fünften Deklination generiert wird. \par
Die Bildung der Wortformen für Nomen ist fast schon trivial. Der Wortstamm wird meist dadurch gefunden, dass man, wenn nötig, die Endung abtrennt. Anschließend werden alle zwölf Wortformen, für die zwei Numeri und die sechs Kasus, durch anfügen der passenden Endung gebildet. \par
Bei der ersten Deklination ist der Wortstamm angenehmerweise gleich der Nominativ Singular-Form. Ebenfalls identisch zum Wortstamm sind die Ablativ und Vokativ Singular-Formen. Die Endungen für die restlichen Kasus sind \textit{-m} für den Akkusativ Singular, \textit{-e} für den Genitiv und Dativ Singular so wie Nominativ und Vokativ Plural, \textit{-rum} für den Genitiv Plural, und \textit{-s} für Akkusativ Plural. Etwas anders verhält es sich bei Dativ und Ablativ Plural. Bei diesen zwei Fällen wird die Endung \textit{-is} nicht an den Wortstamm, sondern an den Wortstock, also den Wortstamm, ohne den Kennvokal, angefügt.\footnote{vgl. \cite{BAYER-LINDAUER1994} S.21f.} \par
\begin{table}[h]
\begin{tabular}{|r:c:l|}
\hline
\multicolumn{2}{|l:}{Wortstamm} & Endung \\
\hline
terr & a & e \\
\hline
Wortstock & \multicolumn{2}{:l|}{Wortausgang} \\
\hline
\end{tabular}
\caption{Bestandteile eines lateinischen Nomens im Genitiv Singular (Vgl. \cite{BAYER-LINDAUER1994} S. 21)}
\end{table}
Bei der zweiten Nomendeklination ist das Vorgehen ganz ähnlich zur ersten Deklination, zumindest bei den Nomen auf \textit{-us} und \textit{-um}. Diesmal muss von der Nominativ Singular-Form die Endung abgespalten werden um, in diesem Fall den Wortstock, zu erhalten. An diesen werden nun die kasusabhängigen Ausgänge angehängt. Diese sind in den meisten Fällen für alle Nomen dieser Deklinationsklasse gleich, in manchen Kasus unterscheiden sie sich aber je nach Nominativ Singular-Endung. Somit ist logischerweise der Nominativ Singular nicht einheitlich sondern hat die Endungen \textit{-us}, \textit{-um} oder \textit{-r}. Bei den Nomen auf \textit{-r} wird meist im Nominativ und Vokativ ein \textit{-e-} eingefügt um die Aussprache zu erleichtern. Allerdings gibt es einige Nomen, die auf \textit{-r} enden, bei denen ein \textit{-e-} zum Wortstamm gehört, weswegen es in keinem Fall entfallen kann.\footnote{vgl. \cite{BAYER-LINDAUER1994} S. 24}. Die selben Endungen haben all diese Nomen im Genitiv, Dativ, Akkusativ und Ablativ Singular (\textit{-i}, \textit{-o}, \textit{-um}, \textit{-o}) sowie im Genitiv, Dativ und Ablativ Plural (\textit{-orum}, \textit{-is} und \textit{is}). Unterschiede gibt es in den verbleibenden Fällen Vokativ Singular so wie Nominativ, Akkusativ und Vokativ Plural. Der Vokativ Singular stimmt bei Nomen auf \textit{-um} und \textit{-r} mit der Nominativ Singular-Form überein, Nomen auf \textit{-us} bilden dagegen die eigenständige Form auf \textit{-e}. Im Plural bilden die Nomen auf \textit{-us} und \textit{-r} die selben Formen, im Nominativ und Vokativ mit den Endung \textit{-i} und im Akkusativ mit \textit{-os}. Die Neutra auf \textit{-um} bilden in allen drei Fällen Formen mit der Endung \textit{-a}.\footnote{vgl. \cite{BAYER-LINDAUER1994} S. 23f.} Zwar bietet sich hier möglicherweise eine Wiederverwendung gleicher Programmteile zur Implementierung an, jedoch wurde der Übersicht halber für jede Unterklasse der zweiten Deklination eine eigene Funktion implementiert, da diese Funktionen noch von so einer niedrigen Komplexität sind, dass die Wiederverwendung von Codeteilen keinen deutlichen Vorteil bringt. \par
Die dritte Deklination ist wohl die komplexeste Deklinationsklasse. Sie wird anhand der Wortstämme in zwei Klassen unterteilt, die Nomen mit einem Wortstamm, der auf einen Konsonanten endet, und die Nomen, deren Wortstamm auf ein kurzes \textit{-ǐ} endet. \par
Die Unterscheidung ist alles andere als unproblematisch, wenn man, wie bei dieser Arbeit, darauf verzichten möchte Vokalqualitäten zu unterscheiden. Denn die Regel sind sowohl von Silbenzahlen als auch von Lautgesetzen abhängig. Und die Bestimmung von Silbengrenzen so wie die Anwendung von Lautgesetzen verlangt nach der Markierung von Langvokalen. Dies würde jedoch die Anwendung der Grammatik behindern. Die Markierung im Lexikon währe nur mit einem etwas größeren Arbeitsaufwand verbunden aber noch relativ leicht machbar, da es ein einmaliger Mehraufwand wäre. Allerdings müssten auch bei jeder Eingabe für das Parsen und Übersetzen die Vokallängen unterschieden werden, was kaum einem Benutzer zugemutet werden sollte. \par
Deshalb wurde versucht, diese Problematik zu umgehen, was zu einigen Regeln führte, die in zumindest im beschränkten Rahmen dieser Grammatik funktionieren. Sie wurden allerdings ad hoc entworfen um die eigentlichen Regeln anzunähern und verlangen nicht nach Allgemeingültigkeit. So ist zunächst das Wort \textit{bos} so unregelmäßig, dass es in diesem Falle einfacher ist, alle Formen einfach aufzulisten, an statt Regeln für die Generierung zu entwerfen. Für einige andere Nomen wird direkt festgelegt nach welchem Deklinationsschema die Formen gebildet werden sollen. So wird \textit{nix} wie ein Nomen des \textit{ǐ}-Stammes und \textit{sedes}, \textit{canis}, \textit{iuvenis}, \textit{mensis} und \textit{sal} wie Nomen der Konsonantenstämmen dekliniert, obwohl dies nach den nachfolgenden Regeln nicht so wäre. Diese Wörter werden aber auch in Grammatikbüchern als aufnahmen gelistet.\footnote{vgl. \cite{BAYER-LINDAUER1994} S. 28} \par
Die nachfolgenden Regeln versuchen die nicht umsetzbaren Entscheidungsregeln, die in der Literatur zu finden sind, mit den gegebenen Informationen zu imitieren. So gehören Nomen der dritten Deklination, die im Nominativ Singular auf \textit{-e}, \textit{-al} und \textit{-ar} enden, zu den \textit{ǐ}-Stämmen. Dagegen ist die nächste Regel Über die Zugehörigkeit zu den Konsonantenstämmen komplizierter. Die ursprüngliche Regel besagt, dass Nomen zu den Konsonantenstämmen gehören, wenn die Nominativ Singular-Form gegenüber dem Wortstamm verändert ist. Es folgen eine Liste von Lautgesetzen, die hier Anwendung finden. Diese werden mit einer Folge von Pattern versucht anzunähern. So kann sich bei einer Ablautung des letzten Vokals bei einer Nominativ Singular-Form auf \textit{-ter} so verändern, dass der Wortstamm nur noch auf \textit{-tr} endet. Es kann auch zu einer Klangfarbenänderung des letzten Vokals kommen, so dass aus der Nominativ Singular-Form auf \textit{-en} der Wortstamm \textit{-in} wird. Des weiteren gibt es noch die Veränderung von \textit{-s} zu \textit{-r} zwischen Nominativ Singular und Wortstamm. Lediglich die Regel, dass sich auch nur die Vokallänge ändern kann, konnte nicht verwirklicht werden. Relativ problemlos dagegen ist wieder die Regel, dass Nomen, deren Wortstock auf zwei oder mehr Konsonanten enden, zu den \textit{ǐ}-Stämmen gehören. Und die letzte Regel ist wieder abhängig von der Silbenzahl und kann deshalb nur angenähert werden. Statt der Silbenzahl wird bei Nomen auf \textit{-es} und \textit{-is} anhand der Buchstabenanzahl entschieden zu welcher der beiden Kategorien ein Wort gehört. Haben Nominativ Singular und Genitiv Singular die selbe Länge, so gehört das Wort zu den \textit{ǐ}-Stämmen, und sonst gehört es zu den Konsonantenstämmen.\footnote{vgl. \cite{BAYER-LINDAUER1994} S. 26ff.} \par
Alle Wörter, auf die keine der bisherigen Regeln zutrifft, werden einfach als den Konsonantenstämmen zugehörig angesehen. \par
Hat man die Nomen der dritten Deklination in \textit{ǐ}- und Konsonantenstämme unterteilt, so kann man relativ problemlos die komplette Paradigmen erzeugen. Bei der dritten Deklination hat man zum Erstellen des Paradigmas die Nominativ und Genitiv Singular-Form, so wie das Geschlecht, gegeben. Zunächst trennt man bei der Genitiv Singular-Form die Endung \textit{-is} ab um den Wortstamm zu erhalten. Da Nomen der dritten Deklination allen drei Geschlechtern angehören können, und die Akkusativ Singular-, Nominativ Plural- und Akkusativ Plural-Form geschlechtsabhängig sind, müssen diese Endungen abhängig vom Geschlecht des Wortes bestimmt werden. Dabei sind bei weiblichen und männlichen Nomen die Akkusativ Singular-Form mit der Endung \textit{-em} und die beiden Pluralfälle bilden Formen mit der selben Endung \textit{-es}. Neutra bilden in allen drei Fällen die Endung \textit{-a}. Dies gilt bei den \textit{ǐ}-Stämmen jedoch nur für Nomen, deren Stamm auf zwei Konsonanten endet. Ist dies nicht der Fall, so sind die Endungen jeweils \textit{-ia}. Bei den Konsonantenstämmen wird also ein Paradigma gebildet, mit den folgenden Singularformen: der gegebenen Nominativ-Form, der gegebenen Genitiv-Form, im Dativ dem Wortstamm mit der Endung \textit{-i}, der vorher aus dem Geschlecht bestimmten Akkusativ-Form, der Ablativform mit der Endung \textit{-e}, und im Vokativ erneut die Nominativform. Im Plural sind es die ebenfalls bereits bestimmten Nominativ Plural-Form, im Genitiv die Endung \textit{-um}, im Dativ und Ablativ die selbe Endung \textit{-ibus} und im Akkusativ wieder die selbe Form wie im Nominativ. Bei den \textit{ǐ}-Stämmen werden im Singular die Formen nach dem selben Schema gebildet, abgesehen von der Ablativform. Denn bei Nomen der \textit{ǐ}-Stämme, die im Nominativ Singular auf \textit{-e}, \textit{-al} und \textit{-ar} enden, bilden den Ablativ mit der Endung \textit{-i}, alle anderen, wie die Konsonantenstämme, mit \textit{-e}. Im Plural weicht die Genitivform ab, denn die Endung lautet hier \textit{-ium} statt \textit{-um}.\footnote{vgl. \cite{BAYER-LINDAUER1994} S. 28} \par
Nomen der vierten Deklination können in der Nominativ Singular-Form auf \textit{-u} oder \textit{-us} enden. Die Nomen auf \textit{-us} im Nominativ Singular haben diese Endung auch im Genitiv und Vokativ Singular so wie im Nominativ, Akkusativ und Vokativ Plural. Im Dativ Singular enden die Formen auf \textit{-ui} und im Akkusativ Singular auf \textit{-um}. Die verbleibenden Endungen im Plural sind \textit{-uum} im Genitiv und \textit{-ibus} sowohl im Dativ als auch im Ablativ. Da die meisten Endungen mit einem \textit{-u-} beginnen, wird dieser Buchstabe in der Implementierung direkt bei der Abspaltung der Nominativ Singular-Endung am Wortstamm belassen und nur in für Dativ und Ablativ Plural explizit entfernt. Es werden also für die Erzeugung des Paradigmas zwei temporäre Zeichenketten verwendet, zum einen den Wortstamm und zum anderen den Wortstamm mit dem \textit{-u}. Bei den Nomen auf \textit{-u}, die größtenteils Neutra sind, enden fast alle Formen des Singular auf \textit{-u}. Lediglich im Genitiv enden die Formen auf \textit{-us}. Um Plural hat man die für Neutra relativ üblichen Endungen auf \textit{-a} bzw. hier auf \textit{-ua} im Nominativ, Akkusativ und Vokativ. Die verbleibenden Pluralendungen sind identisch zu denen der Nomen auf \textit{-us}. Zur Generierung des Paradigmas werden bei den Nomen auf \textit{-u} drei Zeichenketten verwendet. Zum einen die Nominativ Singular-Form, die fünf mal im Paradigma ohne Endung und zwei mal mit verschiedenen Endungen vorkommt. Dann den Wortstamm, der zwei mal mit Endungen vorkommt. Und schließlich den Wortstamm mit der Endung \textit{-ua}, der immerhin drei mal im Paradigma vertreten ist.\footnote{vgl. \cite{BAYER-LINDAUER1994} S. 33} \par
Und die letzte Deklinationsklasse, die fünfte oder \textit{e}-Deklination, besteht aus den Nomen, die im Nominativ Singular auf \textit{-es} enden. Die Vokativ-Form im Singular und Plural ist auch hier, wie fast immer, gleich der Nominativ-Form. Zusätzlich hat auch der Akkusativ Plural die gleiche Form. Genitiv und Dativ Singular enden beide auf \textit{-ei}, und Akkusativ Singular auf \textit{-em}. Der Ablativ Singular hat nur den Ausgang \textit{-e}, also keine Endung am Wortstamm. Im Plural endet die Genitiv-Form auf \textit{-erum} und Dativ so wie Ablativ auf \textit{-ebus}. Um das Paradigma zu erzeugen wird neben der Nominativ Singular-Form, der davon abgeleitete Wortstamm so wie die Form, die aus dem Wortstamm mit dem Ausgang \textit{-i} besteht, verwendet.\footnote{vgl. \cite{BAYER-LINDAUER1994} S. 34} \par
Betrachtet man die Nomenendungen im Gesamtüberblick so kann man auch oberhalb der Deklinationsklassen Muster erkenne, die man versuchen könnte zu formalisieren. Allerdings wäre der Nutzen davon wohl eher gering und würde zu größeren Problemen in den Details führen. Außerdem war ein Aspekt bei dieser Arbeit die Nähe zu einer gegebenen gedruckten Schulgrammatik. Deshalb wurde auch bei der Implementierung die etablierte Einteilung in die Deklinationsklassen gewahrt.
\subsubsection{Sonderformen}
Die häufigste Sonderform von Nomen dürften die bereits im Lexikon-Kapitel erwähnten Nomen sein, die in der gewünschten Bedeutung nur Pluralformen bilden. Um das Paradigma für diese Nomen zu bilden wird lediglich das vollständige Nomenparadigma gebildet und anschließend durch die Hilfsfunktion \texttt{pluralN} alle Singularformen durch die Fehlerzeichenkette \textit{\#\#\#\#\#\#}, die signalisiert, dass diese Formen nicht existieren.
\subsection{Adjektivflexion}
\label{subsec:adjektiv}
Im Lateinischen müssen Adjektive mit dem Nomen in Genus, Numerus und Kasus übereinstimmen. Zusätzlich gibt es drei Steigerungsstufen, Positiv, Komparativ und Superlativ. Deshalb werden sie in diesen Merkmalen flektiert. Auf diese Art und Weise enthält das Paradigma ziemlich viele Wortformen, diese zu generieren ist aber relativ einfach, nachdem man bereits eine funktionierende Nomenflexion hat. Denn die Adjektive bilden, durch ihre Kongruenz mit Nomen, meist die gleichen Formen wie die durch sie attribuierten Nomen. Viele Adjektive gehören zur ersten und zweiten Deklination. Adjektive dieser Klasse bilden für Feminina die selben Endungen wie Nomen der ersten Deklination und verhalten sich auch für Maskulina und Neutra jeweils analog zu Nomen der zweiten Deklination auf \textit{-us} (Maskulina) und \textit{-um} (Neutra). Alle weiteren Adjektive werden zur dritten Adjektivdeklinationsklasse gezählt. Diese haben ebenfalls einige Gemeinsamkeiten. So haben all diese Adjektive die Endung \textit{-e} bzw. \textit{-i} im Ablativ Singular, \textit{-um} bzw. \textit{-ium} im Genitiv Plural und \textit{-a} bzw. \textit{-ia} im Nominativ, Vokativ und Akkusativ Plural des Neutrums, je nach Zugehörigkeit zu den Konsonanten- oder \textit{ǐ}-Stämme. Denn diese werden auch hier wieder unterschieden.\footnote{vgl. \cite{BAYER-LINDAUER1994} S. 38} \par
Bei Adjektiven der ersten und zweiten Deklination kann wieder das ganze Paradigma aus einer einzigen Zeichenkette erzeugt werden. Ebenfalls ist die bei Adjektiven auf \textit{-is} und \textit{-x} möglich, die zur dritten Deklination gehören. Sollte eine Zeichenkette nicht genügen, wie bei den meisten Adjektiven aus der dritten Deklinationsklasse, so gibt es, wie bereits von den Nomen bekannt, die Möglichkeit, das Paradigma aus zwei gegebenen Formen, Nominativ Singular und Genitiv Singular der maskulinen Form, zu generieren. Für sehr seltene Adjektive, für die auch diese Möglichkeit nicht ausreichend ist, kann aus drei Wortformen, nämlich den drei Nominativ Singular-Formen, das Paradigma generiert werden. Diese Option wird jedoch im bisherigen Lexikon nicht benötigt.\footnote{\textbf{ParadigmsLat.gf} und \textbf{MorphoLat.gf}} \par
Die Deklinationsklasse der Adjektive wird, wenn nur eine Form, die Nominativ Singular-Form bei Maskulinum, gegeben ist, wieder anhand der Endung bestimmt. Ist diese \textit{-us}, so ist das Adjektiv drei-endig und gehört zur ersten und zweiten Deklination. Hat es eine andere Endung, so gehört es zur dritten Deklination. Sind zwei Wortformen vorhanden, so wird zusätzlich die gegebene Genitiv Singular-Form bei Maskulina betrachtet. Ist die gegebene Nominativ-Endung \textit{-us} und die Genitiv-Endung \textit{-i}, so ist wie bereits gesagt, das Adjektiv drei-endig. Ist dagegen die Genitiv-Endung \textit{-is}, so ist das Adjektiv Teil der dritten Deklination. Zur dritten Deklination gehören auch alle anderen Adjektive, die im Genitiv bei Maskulina auf \textit{-is} enden so wie alle Adjektive die im gegebenen Nominativ auf \textit{-is} und im entsprechenden Genitiv auf \textit{-e} enden. Alle anderen Adjektive mit zwei Wortformen führen zu einem Fehler. Sollte dieser Fehler für ein Adjektiv im Lexikon auftreten, so muss man zur Erzeugung des Paradigmas die Funktion verwenden, die die drei Nominativ Singular-Formen verwendet.\footnote{vgl. \cite{BAYER-LINDAUER1994} S. 36 u. S. 38} \par
Das Paradigma für Adjektive der ersten und zweiten Deklination wird folgendermaßen gebildet. Zunächst wird der Wortstamm bestimmt. Normalerweise entspricht der Wortstamm der Nominativ Singular-Form ohne die geschlechtsspezifische Endung. Also in diesem Fall, bei Maskulina, \textit{-us}. Endet das Adjektiv allerdings auf \textit{-er} statt auf \textit{-us}, so ist der Wortstamm diese Nominativ-Form ohne das \textit{-e-}. In einigen wenigen Ausnahmefällen entspricht er allerdings der Nominativ Singular Maskulin-Form. Zu diesen Ausnahmen gehören unter anderem \textit{asper}, \textit{liber}, \textit{miser}, etc. Bei diesen Adjektiven auf \textit{-er} bleibt also das \textit{-e-} auch in allen anderen Formen erhalten. Als nächstes wird aus der maskulinen Nominativ Singular-Form ein Nomenparadigma generiert, als ob es sich bei der Adjektivform um die Grundform eines Nomens handelt, und für die spätere Verwendung zwischengespeichert. Dazu wird genau die oben genannte Methode verwendet, um ein Nomenparadigma der zweiten Deklination auf \textit{-us} zu bilden. Damit ist theoretisch schon ein Drittel des Adjektivparadigmas im Positiv, also der ungesteigerten Form, erzeugt. Bevor diese Steigerungsstufe vervollständigt wird, werden aber zuerst die Komparativ- und Superlativformen, also die Zwei Steigerungsstufen, erzeugt. Normalerweise wird die Steigerung von Adjektiven durch Flexion ausgedrückt. Es müssen also eigene Wortformen für jede Steigerungsstufe generiert werden. Bei manchen Adjektiven wird dies jedoch statt dessen mit der Positivform und entsprechenden Adverbien umschrieben. Adjektive, die so eine Umschreibung benötigen, enden allgemein auf \textit{-us}, wobei aber der Wortstamm selbst wieder entweder auf einen Vokal oder \textit{-r-} endet. Nach dieser Regel wird z.B. bei \textit{arduus} und \textit{mirus} nicht wie später beschrieben die Steigerung morphologisch kodiert, sondern mit den Adverbien \textit{magis} (Komparativ) und \textit{maxime} (Superlativ) umschrieben. Deshalb muss für diese Wörter nur die Positivform generiert werden. Bei allen anderen Adjektiven der ersten und zweiten Deklination werden für die Steigerung neue Wortstämme gebildet, an die wiederum die für die erste und zweite Deklination übliche Endungen angehängt werden.\footnote{vgl. \cite{BAYER-LINDAUER1994} S. 36f} \par
Die Bildung des neuen Wortstammes ist nicht ganz trivial. Zunächst einmal gibt es in der lateinischen Sprache Adjektive, deren Komparativ- und Superlativstamm kaum Gemeinsamkeiten mit der Grundform haben. Dazu zählen z.B \textit{bonus} (komp. \textit{melior}, sup. \textit{optimus}), \textit{malus} (komp. \textit{peior}, sup. \textit{pessimus}), \textit{magnus} (komp. \textit{maior}, sup. \textit{maximus}), \textit{parvus} (komp. \textit{minor}, sup. \textit{minimus}), etc. Für jedes dieser Wörter muss eine eigene Regel existieren, wie der Wortstamm im Komparativ und Superlativ aussieht. Teilweise sind es wirklich nur Abbildungen auf eine neue Genitiv- und eine neue Nominativ-Form, die wie bei den regelmäßigen Adjektiven für die Bildung der Steigerungsformen verwendet werden können. Teilweise wird aber auch sogleich das entsprechende Nomenparadigma für den Superlativ gebildet. Dies hängt davon ab, ob die Superlativform, eine der üblichen Superlativendungen hat, oder nicht. Zur Zuordnung, so wie zur Generierung der Komparativformen, wird als kennzeichnende Form die Genitiv Singular-Form des bereits erstellten Nomenparadigmas verwendet. Dieser hat den Vorteil, dass er bei allen Adjektiven den wirklichen Wortstamm enthält, was im Nominativ wie schon ausgeführt, nicht der Fall ist. Der Superlativ dagegen wird aus der Nominativ-Form gebildet.\footnote{vgl. \cite{BAYER-LINDAUER1994} S. 40ff.} \par
Der Komparativ wird üblicherweise durch das Nominativ-Suffix \textit{-ior} für Feminina und Maskulina und \textit{-ium} für Neutra ausgedrückt. Adjektive sind also im Komparativ zwei- statt drei-endig. Die Endungen im Singular sind \textit{-ior}, \textit{-ioris}, \textit{-iori}, \textit{-iorem}, \textit{-iore} im Nominativ/Vokativ, Genitiv, Dativ, Akkusativ und Ablativ. Im Plural sind es entsprechend \textit{-iores}, \textit{-iorum}, \textit{-ioribus}, \textit{-iores} und \textit{-ioribus}. Die Neutrumformen unterscheiden sich nur im Nominativ, Vokativ und Akkusativ von den Femininum-/Maskulinumformen, nämlich \textit{-ius} im Singular und \textit{-ia} im Plural. Diese Endungen werden an den Wortstamm, also die Genitiv-Form ohne die Genitivendung \textit{-i} bzw. \textit{-is}, angehängt.\footnote{vgl. \cite{BAYER-LINDAUER1994} S. 40f.} \par
Der Superlativ ist hingegen wieder drei-endig und bildet die Formen nach der ersten und zweiten Deklination. Dafür ist es für den Superlativ nicht so leicht das passende Suffix zu bilden, das abhängig von der Nominativ Singular Maskulin-Form des Adjektivs ist. Endet diese auf \textit{-er} so werden die Suffixe \textit{-rimus,-a,-um} verwendet, dagegen bei Wörtern auf \textit{-lis} verwendet man die Suffixe \textit{-limus,-a,-um}, in allen anderen Fällen die Suffixe \textit{-issimus,-a,-um}. Hinzu kommt allerdings noch eine mögliche Lautveränderung am Ende des Wortstocks. So wird beim Anhängen von \textit{-issimus,-a,-um} aus einem \textit{-x} ein \textit{-c-} und endet die Grundform des Wortes auf \textit{-ns}, so wird es im Wortinneren zu einem \textit{-nt-}. Das komplette Superlativ-Paradigma wird wieder nach dem Schema für Nomen der ersten und zweiten Deklination erzeugt.\footnote{vgl. \cite{BAYER-LINDAUER1994} S. 42} \par
Für die oben genannten Adjektive, die ihre Komparation durch Adverbien und nicht durch Morphologie kodieren, wird zusätzlich zur entsprechenden Steigerungsstufe das passende Adverb gespeichert, oder wenn keines benötigt wird, der leere Zeichenketten. Die Gesamtstruktur der Adjektive wird also gebildet aus dem Positiv, der wiederum aus den Nomenparadigmen der ersten und zweiten Deklination für die drei Genera besteht, und den beschriebenen Komparativ- und Superlativformen, zusammen mit den möglicherweise benötigten Adverbien.\par
Für die Adjektive der dritten Deklination stimmt das Vorgehen größtenteils mit dem gerade beschriebenen Vorgehen für die erste und zweite Deklination überein. Allerdings sind zwei Wortformen für die Generierung des Paradigmas nötig, die Nominativ- und die Genitiv-Form. Zunächst wird die Endung von der gegebenen Nominativ-Form abgetrennt und die Nominativformen für alle drei Genera gebildet. Dabei können Adjektive der dritten Deklination entweder drei-endig (m. \textit{acer}, f. \textit{acris}, n. \textit{acre}), wenn sie als Nominativ Maskulin-Form auf \textit{-er} enden, zwei-endig (m./f. \textit{fortis}, n. \textit{forte}), wenn die Nominativform auf \textit{-is} endet, oder sonst auch ein-endig (m./f./n. \textit{felix}) sein. Anschließend wird das Nomenparadigma für ein Maskulinum gebildet.\footnote{vgl. \cite{BAYER-LINDAUER1994} S 38f.} \par
In diesem Falle kann nicht so klar auf die Nomenflexion zurückgegriffen werden. Statt dessen wird das Paradigma direkt aus zwei gegebenen Formen und dem gewünschten Geschlecht hergeleitet. Dazu wird zunächst der Wortstamm durch das Abtrennen der Genitivendung von der entsprechenden Form gebildet. Anschließend wird von Geschlecht abhängig die Akkusativ Singular- (Endung: m./f. \textit{-em}, n. keine Endung) und Nominativ/Vokativ/Akkusativ Plural-Form (Endung: m./f. \textit{-es}, n. \textit{-ia}) gebildet. Ebenso wie die geschlechtsunanhängige Dativ/Ablativ Singular-Form mit der Endung \textit{-i}. Zusammen mit der feststehenden Genitiv Singular- (\textit{-is}), Genitiv Plural- (\textit{-ium}) und Dativ/Ablativ Plural-Endung (\textit{-ibus}) können alle Formen des Paradigmas gebildet werden.\footnote{\texttt{noun3adj} in \textbf{ResLat.gf}} \par
Darauf folgt die Bildung der Komparativ- und Superlativformen genauso wie bei den Adjektiven der vorherigen Klasse. Abschließend werden die Nomenparadigmen für die zwei fehlenden Geschlechter, genauso wie bei der maskulinen Form, gebildet und schlussendlich zu einer Adjektivstruktur zusammengesetzt. Bei Adjektiven dieser Deklination werden die Steigerungsformen immer durch Flexion kodiert. Deshalb bleiben die Felder für die Steigerungsadverbien immer leer.
\subsection{Verbflexion}
\label{subsec:verb}
Verben bilden im Lateinischen von allen Wortarten die meisten Formen. Denn bei finiten Verben werden folgende Merkmale unterschieden: Person, Numerus, Tempus, Modus, und Generum, jeweils mit unterschiedlichen Wertebereichen. So gibt es drei Personen, zwei Numeri (Singular und Plural), sechs Zeitformen (Präsens, Imperfekt, Perfekt, Plusquamperfekt, Futur I und Futur II), drei Modi (Indikativ, Konjunktiv, Imperativ I und Imperativ II) und zwei Genera\footnote{vielleicht passender Diathesen} (Aktiv und Passiv). Hinzukommen infinitivische Verbformen wie der Infinitiv im Präsens, Perfekt und Futur, das Gerundium, das Supin, Partizipien im Präsens, Perfekt und Futur so wie das Gerundiv. Dabei zählen die Infinitive, das Gerundium und das Supin nach ihrer Verwendung und Formenbildung zu den substantivischen Formen und die Partizipien und das Gerundiv zu den adjektivischen Verbformen.\footnote{vgl. \cite{BAYER-LINDAUER1994} S. 66} Für jede dieser Formen ist im Datentyp (vgl. Listing \ref{GF-Res-Verb} für lateinische Verben im Grammatical Framework ein Feld vorhanden. In den Typen der Felder sind auch jeweils kodiert, in welchen Merkmalen diese Form flektiert wird. So werden Aktiv-Formen eines Verbes nach Zeitstufe und Tempus, die zusammen die üblichen Tempora bilden, Numerus und Person flektiert.  \par
Die wenigsten der lateinischen Verben bilden jedoch tatsächlich alle diese Verbformen. So kann ein komplettes Passiv nur von transitiven Verben gebildet werden\footnote{vgl. \cite{BAYER-LINDAUER1994} S. 67}. Dagegen gibt es im Lateinischen so genannte Deponentia, die zwar aktivisch verwendet werden, allerdings nur passive Formen bilden\footnote{vgl. \cite{BAYER-LINDAUER1994} S. 83}. Ganz zu schweigen von all den Besonderheiten der unregelmäßigen Verben\footnote{vgl. \cite{BAYER-LINDAUER1994} S. 105ff.}. Um fehlende Verbformen zu markieren wurde eine Zeichenkette gewählt, die in Eingaben mit an Sicherheit grenzender Wahrscheinlichkeit nicht vorkommt. Die gewählte Zeichenkette ist \textit{\#\#\#\#\#\#}. Sie sollte entsprechend an den nötigen Stellen behandelt werden. Allerdings ist die Fehlerbehandlung im Grammatical Framework momentan noch eher rudimentär. Die Behandlungen von fehlenden Werten soll aber in Zukunft durch die Einführung eines aus Haskell und anderen funktionalen Programmiersprachen bekannten\footnote{???} Maybe- oder auch Option-Datentyps ermöglicht werden. Dieser polymorphe Datentyp hat für einen konkreten Datentyp zwei mögliche Werte, entweder \textit{Just a} mit \texttt{a} einem konkreten Wert eines anderen Datentyps, wenn solch ein wert vorhanden ist, oder \texttt{Nothing}, wenn kein Wert vorhanden ist. Diese Möglichkeit mit dem Maybe-Datentyp zu arbeiten war allerdings zu Beginn der Arbeit noch nicht gegeben.\par
\subsubsection{Konjugationsklassen}
In der lateinischen Sprache gibt es für die Konjugation\footnote{Verbflexion} der Verben, ähnlich wie die Deklinationsklassen der Nomen, vier Klassen. Diese Konjugationsklassen verhalten sich teilweise auch ganz analog zu den Deklinationsklassen der Nomen und Adjektiven. Die Konjugationsklassen werden wieder anhand von Kennlauten unterschieden. Die erste Konjugation hat, wie die erste Deklination, als Kennlaut den Vokal \textit{-ā-}, die zweite den Kennlaut \textit{-ē-} und die vierte den Kennlaut \textit{-ī-}. Die dritte Konjugation ist wieder unterteilt, zum einen in die konsonantische Konjugation und zum anderen in die Kurzvokalische Konjugation. Wie der Name schon sagt, ist der Kennlaut der konsonantischen Konjugation ein Konsonant, und bei der kurzvokalischen Konjugation entweder der Kurzvokal \textit{-ǔ-} oder \textit{-ǐ-}.\footnote{vgl. \cite{BAYER-LINDAUER1994} S. 68f.} \par
Diese Klassen gelten sowohl für die Konjugation der regulären Verben als auch für die Deponentia. Für die Bildung des Verbparadigmas gibt es wieder zwei Arten von Verben. Das Paradigma für die meisten Verben der ersten, zweiten und vierten Konjugation, sowohl bei den normalen Verben als auch bei den Deponentia, kann von einer einzigen Verbform, der Infinitiv-Präsens-Aktiv-Form, gebildet werden. Für Verben der dritten Konjugation so wie unregelmäßigere Verben der anderen Konjugationsklassen werden vier Verbformen verwendet. Neben der Infinitiv-Präsens-Aktiv-Form wird die 1.-Person-Singular-Präsens-Indikativ-Aktiv-Form, die 1.-Person-Singular-Perfekt-Indikativ-Aktiv-Form, so wie die Partizip-Perfekt-Passiv-Form.  \par
Denn für die Verben der ersten, zweiten und vierten Konjugation kann die Zugehörigkeit zur entsprechenden Konjugationsklasse am Wortausgang\footnote{Kennlaut zusammen mit der Endung} der Infinitiv-Präsens-Aktiv-Form abgelesen werden. Endet die Verbform auf \textit{-a-}, \textit{-e-} oder \textit{-i-} mit der Endung \textit{-ri}, so handelt es sich um ein Deponens der ersten, zweiten oder vierten Konjugation. Endet die Verbform dagegen auf \textit{-a-}, \textit{-e-} oder \textit{-i-} mit der Endung \textit{-re}, so handelt es sich entsprechend um ein reguläres Verb der ersten, zweiten oder vierten Konjugation. Zur Einordnung der Verben der dritten Konjugation sind mindestens zwei Wortformen, neben der Infinitiv-Präsens-Aktiv- auch noch die 1.-Person-Singular-Präsens-Indikativ-Aktiv-Form, nötig. Endet die Verbform nämlich nur auf \textit{-i}, so gehört sie zu den Deponentia der dritten Konjugation. Der Kennlaut zur Unterscheidung in Konsonantenstämme oder Kurzvokalstämme erscheint erst bei den finiten Präsensformen, weshalb eine von diesen, nämlich die erwähnte 1.-Person-Singular-Präsens-Indikativ-Aktiv-Form, zu Rate gezogen wird. Ist in diesem Falle der Wortausgang \textit{-ior}, so gehört das Wort zu den Kurzvokalstämmen der dritten Konjugation, sonst gehört es zu den Konsonantenstämmen. Endet die Infinitiv-Form dagegen auf \textit{-ere} muss folgendermaßen unterschieden werden: Endet die 1.-Person-Singular-Präsens-Indikativ-Aktiv-Form auf einen Konsonanten gefolgt von \textit{-o}, so gehört das Verb zu den Konsonantenstämmen der dritten Konjugation, endet diese Form auf \textit{-eo}, so gehört sie statt zur dritten zur zweiten Konjugation, was jedoch nicht an der Infinitivform zu unterscheiden ist. Endet die Wortform auf einen der beiden Kennvokale der kurzvokalischen dritten Deklination gefolgt von einem \textit{-o}, so ist das Wort offensichtlich Teil der Kurzvokalstämme, in allen anderen Fällen ist es teil der Konsonantenstämme der dritten Konjugation.\footnote{vgl. \cite{BAYER-LINDAUER1994} S. 68f.} \par
\begin{lstlisting}[float=ht,label={GF-Res-Verb},caption={Datentyp eines Verbs im Grammatical Framework},basicstyle=\footnotesize]
param
  Agr = Ag Gender Number Case ; -- Agreement for NP et al.
  VActForm  = VAct VAnter VTense Number Person ;
  -- No anteriority in passive because perfect forms are built using participle
  VPassForm = VPass VTense Number Person ;  
  VInfForm  = VInfActPres | VInfActPerf Gender | VInfActFut Gender 
            | VInfPassPres | VInfPassPerf Gender | VinfPassFut ;
  VImpForm  = VImp1 Number | VImp2 Number Person ;
  VGerund   = VGenAcc | VGenGen |VGenDat | VGenAbl ;
  VSupine   = VSupAcc | VSupAbl ;
  VPartForm = VActPres | VActFut | VPassPerf ;

oper
  Verb : Type = {
    act   : VActForm => Str ;
    pass  : VPassForm => Str ;
    inf   : VInfForm => Str ;
    imp   : VImpForm => Str ;
    ger   : VGerund => Str ;
    geriv : Agr => Str ; 
    sup   : VSupine => Str ;
    part  : VPartForm =>Agr => Str ;
    } ;
\end{lstlisting}
\FloatBarrier
\subsubsection{Verbstämme}
Für jede Konjugationsklasse werden nun einige Wortformen und -stämme gebildet, aus denen das gesamte Paradigma erstellt werden kann. Einige der dafür erstellten Formen mögen in einer der Konjugationsklassen redundant sein, das heißt für mehrere Merkmale wird die selbe Zeichenkette gebildet, dies liegt jedoch in der einheitlichen Form der Behandlung aller Verben. In Lateingrammatiken findet man meist drei Arten von Wortstämmen eines Verbes. Zu diesen gehört zunächst der Präsensstamm, von dem allgemein die Präsens- Imperfekt- und Futur-I-Formen im Aktiv und Passiv, das Gerundium, das Gerundiv und das Partizip so wie der Infinitiv Präsens gebildet werden. Der Perfekt dagegen wird verwendet um die Perfekt-, Plusquamperfekt- und Futur II-Formen im Aktiv so wie den Infinitiv Perfekt im Aktiv zu bilden. Und zuletzt den Partizipialstamm, von dem die Perfekt-, Plusquamperfekt- und Futur II-Formen im Passiv, zusammen mit dem Hilfsverb esse, so wie das Partizip und der Infinitiv Futur im Aktiv gebildet werden.\footnote{vgl. \cite{BAYER-LINDAUER1994} S. 66} \par
Zur zusätzlichen Erleichterung bei der Formenbildung werden zusätzlich zu den drei Verbstämmen für jede Zeitform und jeden Modus der entsprechende Wortstock und der Infinitiv-Präsens-Aktiv verwendet. Alles in allem wird das gesamte Verbparadigma also aus 15, bei Deponentia lediglich 9 wegen des unvollständigen Paradigmas, verschiedenen Zeichenketten durch Anhängen der Endungen gebildet, die die Merkmale wie Person, Numerus, Diathese, etc. kodieren. In der lateinischen Sprache werden grammatische Merkmale bei Verben an verschiedenen Stellen kodiert. Einerseits gibt es Infixe, die Tempus, im Bereich von Präsens, Imperfekt und Futur, und Modus, im Bereich von Indikativ und Konjunktiv, kodieren. Zum anderen werden in der Endung die Diathese, also Aktiv oder Passiv, das Zeitverhältnis\footnote{???}, der Modus des Imperativs, vor allem jedoch Numerus und Person, kodiert.\footnote{vgl. \cite{BAYER-LINDAUER1994} S. 71f.} \par
Diese 15 bzw. 9 Formen werden für jede Konjugationsklasse unabhängig gebildet. Für reguläre Verben und Deponentia der ersten, zweiten und dritten Konjugation werden alle Zeichenkette alleinig aus der Infinitiv-Präsens-Aktiv-Form gebildet. Der erste Schritt auf dem Weg zum vollen Paradigma ist es, die Infinitiv Präsens-Endung, \textit{-re} bei regulären Verben und \textit{-ri} bei Deponentia abzutrennen um den Präsensstamm zu erhalten. Der Worstock im Präsens Indikativ ist zu diesem identisch, womit schon die ersten zwei der benötigten Formen vorhanden sind. Die dritte, der Stamm bei Präsens Konjunktiv, wird in der ersten Konjugation durch Ersetzen des Kennlauts gebildet. Der Kennlaut \textit{-ā-} wird durch ein \textit{-e-} ersetzt. Bei der zweiten und vierten Konjugation wird statt dessen an den Präsensstamm ein \textit{-a-} angehängt. Für das Imperfekt wird an den Präsensstamm ein Suffix angefügt, das sowohl diese Zeitstufe als auch den Modus ausdrückt. Dieses Suffix ist für den Indikativ \textit{-ba-} und für den Konjunktiv \textit{-re-}. Lediglich bei der vierten Konjugation wird zwischen Stamm und Suffix noch ein \textit{-e-} eingeschoben. Die sechste Form, der Worstock des Futur Indikativ, wird analog zum Imperfekt, durch ein Suffix ausgedrückt, in diesem Falle \textit{-bi-} bei Verben der ersten und zweiten und \textit{-e-} bei Verben der vierten Konjugation. Damit sind alle Wortstöcke, die auf dem Präsensstamm basieren, gebildet. Als nächstes folgt der Perfektstamm. Dieser, so wie alle auf ihm basierenden Wortstöcke, existieren nur bei den regulären Verben und nicht bei den Deponentia, denn diese bilden alle vorzeitigen Formen\footnote{Perfekt, Plusquamperfekt und Futur II im Gegensatz zu Präsens, Imperfekt und Futur} mit Hilfe des Partizips zusammen mit einer Form des Hilfsverbs \textit{esse}. Bei allen anderen Verben wird die Vorzeitigkeit durch das Suffix \textit{-v-} bzw. \textit{-u-} ausgedrückt. Deshalb wird bei diesen Verben der Perfektstamm im Grunde aus dem Präsensstamm durch Anhängen des passenden Suffixes gebildet. Bei Verben der ersten und vierten Konjugation geschieht dies wirklich nur durch Anhängen des Suffixes \textit{-v-}. Bei der zweiten Konjugation jedoch entfällt der Kennvokal \textit{-ē-}und statt des Suffixes \textit{-v-} wird das Suffix \textit{-u-} angehängt. Der Perfekt-Indikativ-Stamm ist wieder identisch zum Perfektstamm. Im Konjunktiv wird dagegen ein weiteres Suffix \textit{-eri-} benötigt. Das selbe betrifft die Plusquamperfekt-Stämme mit den Suffixen \textit{-era-} im Indikativ und \textit{-isse-} so wie den Futur II-Stamm mit dem Suffix \textit{-eri-}. Die letzte fehlende Zeichenkette um das Paradigma generieren zu können ist der Partizipialstamm, der auch wieder für die Deponentia gebildet wird. Dazu wird an den Präsensstamm einfach das Suffix \textit{-t-} angehängt. Lediglich bei Veben und Deponentia der zweiten Konjugation wird zusätzlich der Kennvokal \textit{-ē-} zu einem \textit{-i-}. Zusammen mit der als Ausgangsbasis gewählten Infinitiv sind damit alle 15 bzw. 9 Formen bzw. Formenbestandteile gebildet, die verwendet werden um das Paradigma zu generieren.\footnote{vgl. \cite{BAYER-LINDAUER1994} S. 68ff.} \par
Die Bildung der soeben genannten Formen ist für die Verben und Deponentia der dritten Konjugation etwas anders, vor allem weil die Formen nicht nur von einer einzelnen Wortform, wie bei den anderen Konjugationsklassen, sondern von drei Wortformen gebildet wird. Die zusätzlichen Wortformen sind 1.-Person-Singular-Perfekt-Indikativ-Aktiv und das Partizip-Perfekt-Passiv. Der Präsensstamm wird bei Konsonantenstämmen und kurzvokalischen Stämmen unterschiedlich gebildet. Zunächst wird bei beiden die Infinitiv-Endung, diesmal \textit{-ere}, abgetrennt. Und bei den kurzvokalischen Stämmen wird stattdessen ein Suffix \textit{-i-} angefügt. Bei den Deponentia wird bei den Konsonantenstämmen die Endung \textit{-or} von der 1.-Person-Singular Präsens-Indikativ-Aktiv abgetrennt, bei den Kurzvokalstämmen fällt der Präsensstamm mit der 1.-Person-Singular-Präsens-Indikativ-Aktiv zusammen. Die restlichen Präsensformen werden genauso gebildet, wie bei der vierten Konjugation, ebenso der Stamm bei Imperfekt-Indikativ. Beim Imperfekt-Konjunktiv-Stamm wird lediglich zusätzlich vor dem Suffix ein \textit{-e-} eingefügt. Der Futur I-Stamm fällt bei den Konsonantenstämmen mit dem Präsensstamm zusammen, bei den kurzvokalischen Stämmen wird wie bei der vierten Konjugation das Suffix \textit{-e-} angehängt. Der Perfektstamm, wo vorhanden, wird von der gegebenen Perfektform gebildet, indem die Endung \textit{-i} abgetrennt wird. Die restlichen Perfekt-, Plusquamperfekt- und Futur II-Formen werden analog zur vierten Konjugation aus dem Perfektstamm gebildet. Der Partizipstamm, als letzte nötige Zeichenkette, wird aus der gegebenen Partizip-Form durch Abtrennen der Endung \textit{-us} gebildet.\footnote{vgl. \cite{BAYER-LINDAUER1994} S. 68ff.} \par
Hat man all diese Wortstämme und -stöcke, so kann man einen großen Teil des Paradigmas allein durch das Anhängen der entsprechenden Endung bilden. Die Endungen sind zunächst einmal anhand des Modus in zwei Gruppen unterteilt, die Endungen für Indikativ und Konjunktiv so wie die Imperativ-Endungen. Erstere sind wieder unterteilt in Aktivendungen (\textit{-m} - 1. Person Singular, \textit{-s} - 2. Person Singular, \textit{-t} - 3. Person Singular, \textit{-mus} - 1. Person Plural, \textit{-tis} - 2. Person Plural, \textit{-nt} - 3. Person Plural), Aktivendungen bei Vorzeitigkeit\footnote{auch Perfekt-Aktiv-Endungen} (\textit{-i}, \textit{-is-ti}, \textit{-it}, \textit{-imus}, \textit{-is-tis}, \textit{-er-unt-}) und Passivendungen (\textit{-r}, \textit{-ris}, \textit{-tur},\textit{-mur}, \textit{-mini}, \textit{-ntur}). Zweitere sind unterteilt in Aktivendungen (\textit{-e} oder keine Endung - 2. Person Singular Imperativ I, \textit{-to} - 2./3. Person Imperativ Singular II, \textit{-te} - 2. Person Plural Imperativ I, \textit{-to-te} - 2. Person Plural Imperativ, \textit{-nto} - 3. Person Plural Imperativ II) und Imperativendungen bei Deponentia (\textit{-re} - 2. Person Singular Imperativ I, \textit{-tor} - 2./3. Person Singular Imperativ II, \textit{-mini} - 2. Person Plural Imperativ I, \textit{-ntor} - 3. Person Plural Imperativ II).\footnote{vgl. \cite{BAYER-LINDAUER1994} S. 72} Allerdings kommen dabei immer wieder Lautgesetze zu tragen, weswegen es immer wieder Besonderheiten bei der Formenbildung gibt. So werden, wenn der entsprechende Wortstamm auf einen gewissen Laut endet, so wird bei einigen Formen zwischen Stamm und Endung noch ein zusätzlicher Vokal eingefügt. \par
\subsubsection{Reguläre Formenbildung}
Zunächst einmal soll die Paradigmenbildung der regulären Verben geschildert werden. Beginnt man bei den Präsens-Indikativ-Aktiv-Formen, so ist man schon bei der 1.-Person-Singular mit einer recht unregelmäßigen Form konfrontiert. Denn es ist neben der 1.-Person-Singular-Futur-I-Indikativ-Aktiv-Form die einzige die nicht die übliche 1.-Person-Singular-Endung \textit{-m} sondern die Endung \textit{-o} hat. Des weiteren wird, wenn der Stamm auf ein \textit{-a} endet, dieses entfernt, bevor die Endung angefügt wird. Bei den restlichen Formen wird lediglich unter Umständen ein zusätzlicher Vokal, abhängig vom Ende des Präsens-Stamm, eingefügt. Endet der Stamm auf \textit{-a} oder \textit{-e}, so wird kein zusätzlicher Vokal eingefügt, endet der Stamm auf einen Konsonanten, so wird bei der 3. Person Plural der Vokal \textit{-u-} und bei jeder anderen Form der Vokal \textit{-i} eingefügt, und bei jedem anderen Vokal wird nur bei der 3. Person Plural der Vokal \textit{-u} eingefügt, bei anderen Formen allerdings nichts. Im Konjunktiv dagegen wird einfach die zu Person und Numerus passende Endung einfach an den Präsens-Konjunktiv-Aktiv-Stamm angehängt um alle Formen zu bilden. Analoges erfolgt bei den beiden Modi des Imperfekt. Lediglich bei den Futur-I-Aktiv-Formen muss der 1. Person Singular und der 3. Person Plural besondere Beachtung geschenkt werden. Endet der Futur-I-Wortstamm auf \textit{-bi} so wird bei der 1. Person Singular das \textit{-i} durch ein \textit{-o} ersetzt und keine weitere Endung angefügt. Andernfalls wird der letzte Buchstabe des Stammes durch ein \textit{-a} ersetzt und anschließend die 1.-Person-Singular-Präsens-Indikativ-Aktiv-Endung angefügt. Bei der 3. Person Plural wird, falls der Stamm auf \textit{-bi} endet, das \textit{-i} durch ein \textit{-u} ersetzt, bevor die Endung angehängt wird. Endet der Stamm nicht auf \textit{-bi} so wird nur die entsprechende Endung angehängt. Bei den Perfekt-Indikativ-Formen muss lediglich die passende Perfekt-Aktiv-Endung an den der Zeitstufe entsprechenden Wortstamm angehängt werden. Bei den Plusquamperfekt-Formen und Futur-II-Formen wird dagegen die Präsens-Aktiv-Endung an den der Zeitform entsprechenden Wortstamm angehängt. Wobei bei der 1. Person Singular wieder ein Sonderfall eintritt. Statt eine Endung anzuhängen wird der letzte Buchstabe des Wortstamms entfernt und durch ein \textit{-o} ersetzt. Damit sind schon einmal alle Aktiv-Formen des Paradigmas gebildet.\footnote{vgl. \cite{BAYER-LINDAUER1994} S. 74f., S. 78f. u. S. 84f.} \par
Als nächstes soll die Bildung der Passivformen beschrieben werden. Im Passiv müssen nur die Präsens-, Imperfekt- und Futur-I-Formen gebildet werden, denn alle vorzeitigen Verbformen werden wieder durch das Partizip-Perfekt-Passiv zusammen mit einer passenden Form des Hilfsverbs \textit{esse} gebildet. Wie bei den Aktiv-Formen folgt die Mehrzahl der Formen, einige allerdings weichen vom grundlegenden Schema ab. Dies beginnt bereits bei den Präsens-Indikativ-Formen. Schon bei der Form für die 1. Person Singular muss, wenn der Präsens-Indikativ-Stamm auf ein \textit{-a} endet, dieses abgetrennt werden, bevor die passende Endung angefügt werden kann. Bei der 2. Person Singular muss, wenn wenn der Imperativ-Stamm auf einen Konsonanten und der Präsens-Indikativ-Stamm auf ein \textit{-i} endet, dieses entfernt und durch ein \textit{-e} ersetzt werden. Endet der Präsens-Indikativ-Stamm jedoch nicht auf ein \textit{-i} wird nur ein \textit{-e} angehängt bevor die passende Endung angefügt wird. Trifft die Bedingung bei dem Imperativ-Stamm nicht zu, wird dir Endung einfach an den Präsens-Stamm angehängt. Die restlichen Formen des Präsens-Indikativ verhalten sich wie bei den Aktiv-Formen. Es werden lediglich die Passiv-Endungen statt der Aktiv-Endungen angehängt. Ebenso bei Präsens-Konjunktiv-, Imperfekt- und den meisten der Futur-I-Formen. Lediglich die 2. Person Singular verhält sich anders als im Aktiv. Denn in diesem Falle wird zwischen Stamm und Endung noch ein \textit{-e-} eingefügt. Da, wie bereits gesagt, im Passiv alle Perfekt-, Plusquamperfekt- und Futur-II-Formen durch Partizipien umschrieben werden, sind damit auch alle Passiv-Formen beschrieben.\footnote{vgl. \cite{BAYER-LINDAUER1994} S. 76f u. S. 80f.} \par
Damit sind die häufigsten Formen des Verbparadigmas bereits bekannt. Als nächstes folgen die Infinitiv-Formen. Infinitive gibt es in Latein für Präsens, Perfekt und Futur jeweils als Aktiv- und Passiv-Form. Die Infinitiv-Präsens-Aktiv-Form ist schon von Grund auf bekannt, da es die Verbform ist, die in jedem Verbeintrag im Lexikon vorkommen muss. Die Infinitiv-Perfekt-Aktiv-Form dagegen muss aus dem Perfekt-Stamm mit der Endung \textit{-isse} gebildet werden. Im Infinitiv-Futur-Aktiv dagegen basiert die Form auf dem Partizip-Futur-Aktiv und ist deshalb geschlechtsabhängig. An den Partizip-Stamm wird für Maskulina und Neutra \textit{-urum} und für Feminina \textit{-uram} angehängt. Des weiteren basiert dieser Infinitiv auf der Infinitiv-Präsens-Aktiv-Form des Hilfsverbs \textit{esse}, welche aber las Zeichenkette im Paradigma, hier und auch zukünftig, nicht explizit enthalten sein wird. Im Passiv ist der Infinitiv-Präsens wie der Infinitiv-Aktiv-Präsens, jedoch mit der Endung \textit{-ri} statt der Endung \textit{-re}. Der Infinitiv-Perfekt-Passiv bildet wie der Infinitiv-Futur-Aktiv genusabhängige Formen zusammen mit dem Hilfsverb \textit{esse}. Die Formen sind der Partizipialstamm mit den Endungen \textit{-um}, \textit{-am} und \textit{-um}. Und der Infinitiv-Futur-Passiv wird aus dem Partizipialstamm mit der Endung \textit{-um} und dem Verbform \textit{iri}\footnote{Infinitiv-Präsens-Passiv des Verbs \textit{ire}} gebildet.\par
Es folgen die Imperativ-Formen, von denen es üblicherweise sechs Stück gibt. Zum einen den Imperativ I, der direkte Aufforderungen ausdrückt, in einer Singular- und einer Plural-Form. Und den Imperativ II, der eher in die fernere Zukunft gerichtet ist oder allgemeiner verwendet wird. Von ihm gibt es zwei Singular- und zwei Plural-Formen, jeweils für die 2. und 3. Person. Dabei fallen allerdings die beiden Singular-Formen zusammen, bilden also die gleiche Zeichenkette. Der Imperativ-I-Singular besteht aus dem Imperativ-Stamm an den keine Endung angehängt wird, außer er endet auf einen Konsonanten. Dann wird der Vokal \textit{-e} angehängt. Im Plural wird an den Stamm die Endung \textit{-to} angehängt und, wenn der Stamm auf einen Konsonanten endet ein Vokal \textit{-i-} eingeschoben. Die beiden Imperativ-II-Singular-Formen werden genau so wie der Imperativ-I-Plural gebildet, jedoch mit der Endung \textit{-to} statt der Endung \textit{-te}. In Plural dagegen wird bei der 2. Person die Endung \textit{-tote} angehängt. Davor wird allerdings, wenn der Stamm auf einen Konsonant endet, ein \textit{-i-} eingefügt. Und bei der 3. Person wird, wenn der Stamm nicht auf ein \textit{-a} oder \textit{-u} endet, ein \textit{-u-} eingefügt, bevor die Endung \textit{-nto} angefügt wird.\footnote{vgl. \cite{BAYER-LINDAUER1994} S. 82} \par
Die Gerundivformen zählen zwar zu den substantivischen Nominalformen des Verbs, bilden allerdings nur vier Kasusformen, Genitiv, Dativ, Akkusativ und Ablativ. Diese Formen entsprechen im Grunde den Formen eines Nomens der zweiten Deklination. Gebildet werden sie, indem an den Präsens-Stamm die Endungen \textit{-ndi} (Genitiv), \textit{-ndo} (Dativ und Ablativ) und \textit{-ndum} (Akkusativ) angehängt werden. Und wieder wird, wenn der Stamm nicht auf ein \textit{-a} oder \textit{-e} endet, ein \textit{-e-} eingeschoben.\par
Das Gerundiv hingegen wird adjektivisch verwendet und verhält sich so wie ein drei-endiges Adjektiv der ersten und zweiten Deklination. Die Nominativformen sind analog zum Gerund und bestehen aus dem Präsens-Stamm, unter Umständen gefolgt von einem \textit{-e-} und den Endungen \textit{-ndus}, \textit{-nda} und \textit{-ndum}. Die restlichen Formen werden, wie von den Adjektiven gewohnt, gebildet.\par
Ebenfalls wie Adjektive gebildet und verwendet werden die Partizipien. Es gibt drei Partizipien im Lateinischen. Das Partizip-Präsens-Aktiv, das Partizip-Futur-Aktiv und das Partizip-Perfekt-Passiv. Das Partizip-Präsens-Aktiv wird wie ein ein-endiges Adjektiv gebildet. Die Nominativ-Form besteht, wie bei Gerund und Gerundiv, aus dem Präsens-Stamm, gefolgt von einem \textit{-e-} wenn der Stamm nicht auf ein \textit{-e} oder \textit{-a} endet, und der Endung \textit{-ns}. Der Genitiv wird analog mit der Endung \textit{-ntis} gebildet. Das Partizip-Futur-Aktiv so wie das Partizip-Perfekt-Passiv werden wieder wie drei-endige Adjektive gebildet. Die Nominativ-Formen werden mit Hilfe des Partizip-Stammes gebildet. Beim Partizip-Futur-Aktiv lauten die drei Nominativ-Endungen \textit{-urus}, \textit{-urum} und \textit{-urum}, beim Partizip-Perfekt-Passiv \textit{-us}, \textit{-a} und \textit{-um}.
Die beiden letzten Verbformen sind die Supin-Formen. Diese Formen werden gebildet wie die Maskulin-Akkusativ- und Maskulin-Ablativ-Singular-Form des Partizip-Perfekt-Passivs.\footnote{vgl. \cite{BAYER-LINDAUER1994} S. 82} \par
Auf die Verwendung einiger dieser Formen wird später im Syntaxteil noch genauer eingegangen. Die meisten der soeben beschriebenen Formen ist die Verwendung außerhalb des Bereichs, der in dieser Arbeit abgedeckt ist. An dieser Stelle kann man auch einen kurzen Gedanken daran verschwenden, ob es sinnvoll ist all diese Verbformen an einer einzelnen Stelle zu bilden. Denn lediglich die Aktiv- und Passiv-Formen werden in der Syntax verwendet, wie man es für Verben gewohnt ist. Deshalb kann man möglicherweise nur die Bildung dieser Formen als Flexionsmorphologie\footnote{FlexMorph} ansehen, die an dieser Stelle dargestellt werden sollte. Alle anderen Formen, die hier gebildet werden, kann man dagegen als Derivationsmorphologie\footnote{DerivMorph} auffassen, die als eigenständige Funktion andernorts umgesetzt werden sollte um bei Bedarf aus gegebenen Verben andere Wortarten formen zu können. Jedoch findet man die Verbflexion in der hier besprochenen Form in gängigen Schulgrammatiken vor. Deshalb wurde diese Form auch für diese Arbeit gewählt. %\par
% Beschreibung Partizipien, Supin, Etc.
\subsubsection{Deponentia}
Die zweitgrößte Gruppe der lateinischen Verben nach den regulären Verben sind die so genannten Deponentia. Deren Name kommt vom Verb \textit{deponere} (ablegen), da man sagen kann, dass diese Verben ihre aktiven Formen bzw. ihre passive Bedeutung abgelegt haben.\footnote{vgl. \cite{BAYER-LINDAUER1994} S. 83} \par
Aus diesem Grunde ist das Paradigma, im Vergleich zu den regulären Verben, nicht vollständig. Es müssen also auch weniger Formen gebildet werden. Deshalb werden für die Bildung des ganzen Paradigmas auch weniger Wortstämme benötigt. Wie diese gebildet werden, ist bereits in einem vorangegangenen Kapitel, beschrieben worden. \par
Den Anfang bilden wieder die Aktiv-Formen. Die Präsens-Formen sind wieder geprägt von leichten Unterschieden zum Grundschema. So hängt die 1. Person Singular wieder von der Endung des Präsens-Stammes ab. Endet dieser auf \textit{-a} so wird dieses entfernt bevor die Endung \textit{-o} angefügt wird. Die Bildung der 2. Person ist dagegen von der Infinitiv-Form abhängig. Endet diese auf \textit{-ri} bleibt der Präsens-Stamm unverändert. Andernfalls wird, wenn der Präsens-Stamm auf eine \textit{-i} endet, dieses durch ein \textit{-e} ersetzt, oder falls nicht, das \textit{-e} einfach an den Präsens-Stamm angehängt. Schlussendlich folgt die 2.-Person-Singular-Passiv-Endung. Bei der 3. Person Singular wird die passende Endung entweder direkt an den Präsens-Stamm gehängt, außer dieser endet nicht auf ein \textit{-a} oder \textit{-e}, dann wird zwischen Stamm und Endung ein \textit{-u-} eingefügt. Bei allen weiteren Präsens-Formen wird die Endung entweder direkt an den Stamm gefügt, oder es wird, wenn der Stamm auf einen Konsonanten endet, der Vokal \textit{-i-} zwischen Stamm und Endung eingeschoben. Der Präsens-Konjunktiv und Imperfekt-Indikativ so wie Imperfekt-Konjunktiv haben wieder keine vom Grundschema, Stamm plus Endung, abweichenden Formen. Erst bei den Futur-I-Formen sind wieder Abweichungen zu finden. Bei der 1. Person Singular wird zunächst vom Futur-I-Stamm der letzte Buchstabe entfernt. War er teil des Suffixes \textit{-bi}, so wird er durch ein \textit{-o-} ersetzt, sonst durch ein \textit{-a-}. Zum Schluss wird die 1.-Person-Singular-Passiv-Endung angefügt. Bei der 2. Person Singular wird, wenn der Stamm auf das Suffix \textit{-bi} endet, wieder das \textit{-i} durch ein \textit{-e-} ersetzt, bevor die entsprechende Endung eingesetzt wird. Und schließlich wird bei der 3. Person Plural der letzte Buchstabe des Stammes entweder durch ein \textit{-u-} ersetzt, wenn der Stamm auf das Suffix \textit{-bi} endet, bevor die Endung angefügt wird. Andernfalls wird er durch ein \textit{-e-} ersetzt. \par
Da alle weiteren Aktiv-Formen mit Hilfe des Partizips umschrieben werden, sind alle möglichen Aktiv-Formen beschrieben. Passiv-Formen kommen bei Deponentia naturgemäß nicht vor. Deshalb können diese Formen durch die Fehlerzeichenkette \textit{\#\#\#\#\#\#} ersetzt werden, um zu markieren, dass sie im Paradigma nicht vorhanden sind. \par
Die nächsten im Paradigma teilweise vorhandenen Formen sind die Infinitiv-Formen. Der Infinitiv-Präsens-Aktiv ist identisch zum gegebenen Infinitiv-Stamm. Die Infinitive für Perfekt-Aktiv und Futur-Aktiv müssen, wie auch bei den regulären Verben, wieder im Geschlecht mit dem Bezugswort übereinstimmen und bilden deshalb drei Formen. Diese basieren auf dem Partizip-Stamm an den jeweils die drei geschlechtsspezifischen Endungen angefügt werden. Bei dem Infinitiv-Perfekt-Aktiv sind das \textit{-um}, \textit{-am} und \textit{-um}, bei dem Infinitiv-Futur-Aktiv sind es \textit{-urum}, \textit{-uram} und \textit{-urum}. Die passiven Infinitiv-Formen entfallen wieder und werden durch die Fehlerzeichenkette ersetzt. \par
Als nächstes folgen die Imperativ-Formen, von denen es den Imperativ-I im Singular und Plural sowie den Imperativ II in der 2. und 3. Person Singular und der 3. Person Plural gibt. Die erste Form, der Imperativ-I-Singular entspricht, wenn der Infinitiv-Stamm auf \textit{-ri} endet, einfach aus dem Imperativ-Stamm, im anderen Falle aber aus dem Imperativ-Stamm bei dem der letzte Buchstabe durch ein \textit{-e} ersetzt ist. Der Imperativ-I-Plural besteht einfach aus dem Imperativ-Stamm mit der Endung \textit{-mini} und die 2. und 3. Person des Imperativ II im Singular aus dem Stamm mit der Endung \textit{-tor}. Die 3. Person des Imperativ II im Plural dagegen besteht aus dem Präsens-Indikativ-Stamm, dem Vokal \textit{-u-}, wenn eben dieser Stamm nicht auf \textit{-a} oder \textit{-e} endet, und der Endung \textit{-ntor}. \par
Die Bildung des Gerunds ist relativ analog zur Bildung des Gerundivs bei regulären Verben. Die vier Formen des Gerund werden aus dem Präsensstamm, dem Vokal \textit{-e-}, wenn der Präsens-Indikativ-Stamm nicht auf ein \textit{-a} oder \textit{-e} endet, und den vier Endungen \textit{-ndum}, \textit{-ndi}, \textit{-ndo} und \textit{-ndo}. \par
Das Gerundiv bildet wieder alle Formen eines drei-endigen Adjektivs der ersten und zweiten Deklination. Die drei Nominativ-Grundformen sind dabei, wie eben schon beim Gerund der Präsens-Stamm, unter den bereits genannten Bedingungen der Vokal \textit{-e-} und den Endungen \textit{-ndus}, \textit{-nda} und \textit{-ndum}. \par
Als nächstes folgen wieder die Partizipien. Obwohl bisher alle passiven Formen aus dem Paradigma gefallen sind, sind nun allerdings alle drei Partizipien vorhanden. Deshalb möchte ich an dieser Stelle nicht von aktiven und passiven Partizipien, sondern nur von Partizip Präsens, Partizip Perfekt und Partizip Futur sprechen. Diese werden allerdings genau so gebildet wie das Partizip-Präsens-Aktiv, das Partizip-Perfekt-Passiv und das Partizip-Futur-Aktiv bei den regulären Verben. \par
Nun zur letzten Form des Deponens-Paradigmas, dem Supin. Dieses wird zum Abschluss auch wieder genau so gebildet, wie bei den regulären Verben. Damit ist auch die zweite größere Klasse lateinischer Verben komplett behandelt.\footnote{vgl. \cite{BAYER-LINDAUER1994} S. 86} \par
\subsubsection{Unregelmäßige Verben}
Zusätzlich zur relativ großen Gruppe der Deponentia, gibt es in der lateinischen Sprache noch einige andere unregelmäßige Verben. Diese bilden entweder stark vom simplen ``Stamm plus Endung''-Schema abweichende Formen oder nur kleine Teile des Paradigmas, oder auch beides. Deshalb müssen sie gesondert von den andren Verben behandelt werden. Dies geschieht in einem eigenen Teil der Grammatik zur Behandlung unregelmäßiger Wortbildung, den Dateien \textbf{IrregLatAbs.gf} und \textbf{IrregLat.gf}. Bisher werden die Formen einiger sehr wichtiger Verben auf diese Weise gebildet, so z.B. das Kopula \textit{esse}. Das vorgehen ist meist recht ähnlich. Zuerst werden die 9 oder 15 Wortstämme, je nachdem ob das Verb eher wie ein Deponens oder ein reguläres Verb gebildet wird, per Hand aufgelistet. Als nächstes wird die Funktion verwendet um aus den gegebenen Stämmen ein komplettes Paradigma zu erzeugen. Und als letzter Schritt werden in diesem kompletten Paradigma noch einmal Wortformen korrigiert, die falsch gebildet wurden und Wortformen entfernt, die im Zielparadigma nicht vorhanden sind. Auf diese Weise werden die Wörter \textit{esse} (\texttt{be\_V}), \textit{posse} (\texttt{can\_VV}), \textit{ferre} (\texttt{bring\_V}), \textit{velle} (\texttt{want\_V}), \textit{ire} (\texttt{go\_V}) und \textit{fieri} (\texttt{become\_V}) gebildet. \par
Anders verhalten sich dagegen die so genannten \textit{Verba impersonalia}. Sie kommen gewöhnlich nur in der 3. Person Singular oder im Infinitiv auf.\footnote{vgl. \cite{BAYER-LINDAUER1994} S. 111} Deshalb kann man annehmen, dass das Verbparadigma für diese Verbart nur diese Formen enthält. Aus diesem Grunde ist es erheblich effizienter eben nur diese Formen aufzuzählen statt ein komplettes Verbparadigma zu generieren und anschließend die nicht vorkommenden Formen zu eliminieren. Dieses Vorgehen wurde bei \texttt{rain\_V0} (lat. \textit{pluit}) angewandt.
\subsection{Pronomenflexion}
\label{subsec:pronomen}
Ein Kapitel über Pronomenflexion im Sinne der vorhergehenden Kapitel über Nomen, Adjektive und Verben ist aus mehreren Gründen nicht unproblematisch. Zunächst einmal gehören Pronomen zu den geschlossenen Kategorien, also zu den Wortarten, zu denen nur wenig Wörter gehören. Schon deswegen stellt sich die Frage, ob der Aufwand gerechtfertigt ist, eine allgemeine Flexion für eine Klasse von Wörtern zu entwerfen, wenn nur drei oder vier Wörter zu dieser Klasse gehören. Oder ob es einfacher wäre im Lexikon einfach alle Wortformen im entsprechenden Eintrag aufzulisten. Verschlimmert wird diese Problematik dadurch, dass die Wortart der Pronomen keine homogene Wortart ist, sondern aus verschiedensten Unterklassen besteht, die teilweise nach unterschiedlichen Merkmalen flektiert werden. So kann man die Pronomen untergliedern in Personalpronomen, Possesivpronomen, Demonstrativpronomen, Relativpronomen, Interrogativpronomen, Indefinitpronomen und ein paar mehr. Manche, wie die Personalpronomen, werden wie Nomen dekliniert, andere, wie Possesivpronomen, werden wie eher wie Adjektive dekliniert.\footnote{vgl. \cite{BAYER-LINDAUER1994} S. 48ff.} \par
Es wurde deshalb ein Mittelweg gewählt. Denn für die Personal- und Possesivpronomen wurde eine möglichst allgemeine Flexionsfunktion implementiert. Für alle weiteren Klassen von Pronomen wurde dagegen die Lösung gewählt, alle Wortformen im Lexikon direkt aufzulisten. Dies wurde ja bereits im Lexikonkapitel kurz erwähnt. Deshalb soll hier die Flexion von Personal- und Possesivpronomen erläutert werden. Dabei wird ein Typ namens Objekt definiert, der als feste Felder Genus, Numerus und Person hat. Zusätzlich hat er zwei Tabellenfelder, eines für das diesen fixen Merkmalen entsprechende Personalpronomen und eines für das entsprechende Possesivpronomen. Die Form der Personalpronomen ist primär vom Kasus abhängig und Genus sowie Numerus sind fix. Possesivpronomen werden dagegen wie Adjektiv dekliniert, die Form ist also sowohl von Kasus, als auch von Genus und Numerus, abhängig. Deshalb sind für die Possesivpronomenform die festen festen Felder nicht von Bedeutung. Dafür hängen diese Pronomenarten von ein bis zwei weiteren Merkmalen ab. Zum einen, ob der Pro-Drop-Parameter gesetzt ist, also ob das Personalpronomen in der Subjektposition entfallen kann bzw. hier etwas allgemeiner, welche alternative Form das Personalpronomen in der Subjektposition annehmen kann.\footnote{vgl. \cite{METZLER2004} S. 7585} Und zum anderen bilden die Pronomen der 3. Person unterschiedliche Formen, abhängig davon, ob sie reflexiv verwendet werden oder nicht. Deshalb gibt es das Merkmal der Reflexivität. \par
Zunächst einmal sind die Formen der Pronomen von Numerus und Person abhängig. Die Pro-Drop-Form der Personalpronomen ist immer der leere String, denn sie können alle in der Subjektposition entfallen, da die nötige Information bereits im Verb kodiert ist. Die reguläre Personalformen des Pronomens in der ersten Person Singular sind die der Nomenflexion entsprechend die vom Kasus abhängigen Formen \textit{ego} (Nominativ), \textit{mei} (Genitiv), \textit{mihi} (Dativ), \textit{me} (Akkusativ) und \textit{me} (Ablativ). Der Vokativ existiert bei Personalpronomen aus offensichtlichen Gründen nicht. Die possesiven Formen dagegen entsprechen den Formen eines Adjektivs der ersten und zweiten Deklination mit den Grundformen \textit{meus, -a, -um}. Allerdings lauten die Vokativformen \textit{mi, mea, meum}. Bei Pronomen der 1. und 2. Person spielt die Reflexivität noch keine Rolle. In der 2. Person Singular lauten die regulären Personalformen entsprechend \textit{tu}, \textit{tui}, \textit{tibi}, \textit{te} und \textit{te} und die possesiven Formen folgen wieder der ersten und zweiten Deklination bei den Grundformen \textit{tuus, -a, -um}. Bei der 1. Person Plural bildet das Personalpronomen die Formen \textit{nos}, \textit{nostri}, \textit{nobis}, \textit{nos} und \textit{nobis} und die Formen des entsprechenden Possesivpronomens werden aus aus den Grundformen \textit{noster, nostra, nostrum} gebildet. Die Formen der 2. Person Plural sind analog dazu \textit{vos}, \textit{vostri}, \textit{vobis}, \textit{vos} und \textit{vobis} und beim Possesivpronomen werden sie aus den Grundformen \textit{vester, vestra, vestrum} gebildet. In der 3. Person Singular und Plural gibt es nun mehr Formen. Zum einen werden unterschiedliche Formen bei reflexivem und irreflexivem Gebrauch verwendet. Zum anderen unterscheiden sich bei der 3. Person auch die irreflexiven Personalpronomen-Formen nach Geschlecht. Dafür entfallen eben diese irreflexiven Formen bei den Possesivpronomen ganz. Also ergibt sich für die 3. Person Singular der Personalpronomen folgendes Formenschema: Bei den irreflexiven, maskulinen Formen \textit{is}, \textit{eius}, \textit{ei}, \textit{eum} und \textit{eo}, bei femininen \textit{ea}, \textit{eius}, \textit{ei}, \textit{eam} und \textit{ea} und bei Neutra \textit{id}, \textit{eius}, \textit{ei}, \textit{id} und \textit{eo}. Bei den reflexiven Formen fallen wieder alle drei Geschlechter zusammen. Zusätzlich fehlt eine Nominativ-Form. Die verbleibenden Formen sind \textit{sui} im Genitiv, \textit{sibi} im Dativ und \textit{se} sowohl im Akkusativ als auch im Ablativ. Die irreflexiven Possesiv-Formen der 3. Person existieren eigentlich nicht, werden aberdurch die Genitiv-Singular- bzw. Plural-Form von \textit{is, ea, id} nämlich \textit{eius} im Singular und \textit{eorum, -a, -um} im Plural ersetzt. Dafür sind die reflexiven Formen dem Schema entsprechend, sowohl im Singular als auch im Plural, wieder analog zu Adjektivformen mit der Grundform \textit{suus, -a, -um}.\footnote{vgl. \cite{BAYER-LINDAUER1994} S. 48ff.} \par
Damit sind alle möglichen Formen der Personal- und Possesivpronomen behandelt. Um im Lexikon gezielt auf ein einzelnes Pronomen, gekennzeichnet durch Genus, Numerus und Person, zugreifen zu können, existiert die Funktion \textit{mkPronoun}, die anhand dieser Merkmale die passenden Pronomenformen aus der Menge aller möglichen Formen auswählt. Anschließend werden in je einem Feld diese Merkmale fest gespeichert und in dem Feld \texttt{pers} das gewählte Personalpronomen so wie im Feld \texttt{poss} das entsprechende Possesivpronomen abgelegt. Die gesamte Pronomenbehandlung findet in der Datei \textbf{ResLat.gf} statt. \par
Nachdem hier die Formenbildung aller wichtigen Wortarten für das Lateinische beschrieben ist, sind alle Bestandteile vorhanden um aus den Wörtern und Wortformen der Lexikoneinträge größere Einheiten bilden zu können. Dies geschieht im allgemeinen in Form von Syntaxregeln, deren Form und Aufbau der letzte Teil der Grammatik gewidmet ist.
\FloatBarrier
\pagebreak
\section{Syntax}
\label{sec:syntax}
\input{tex/Syntax.tex}
\FloatBarrier
\pagebreak
\chapter{Ausblick}
\label{chap:ausblick}
\begin{sidewaysfigure}[h]
\includegraphics[scale=0.25]{graphics/LatinDependencyGraph.eps}
\caption{Die Abhängigkeiten zwischen den Modulen der Lateingrammatik}\label{GF-DepGraph}
\end{sidewaysfigure}
\FloatBarrier
\pagebreak
\clearpage
\pagenumbering{roman}
\setcounter{page}{1}
%\chapter*{Bibliographie}
%\bibliographystyle{natdin}
\printbibliography
\pagebreak

\part{Anhang}
\label{part:anhang}
\section{Quelltext}
\label{sec:quelltext}
%\input{tex/Source.tex}
\end{document}
