%\documentclass[handout]{beamer}
\documentclass{beamer}
%\usetheme{Singapore} % anderes Layout
%\usetheme{Antibes} % anderes Layout
%\usecolortheme{lily}
\usepackage{color}
\usepackage{german}
\usepackage{latexsym,amssymb}
\usepackage{amsmath} % für begin{cases} ... \end{cases}
%\usepackage[utf8]{inputenc}
%\usepackage[T1]{fontenc}
%\usepackage{algorithm}
%\usepackage{algorithmicx}
%\usepackage{algpseudocode}
\usepackage{multicol}
\usepackage{graphicx}
\usepackage{fontspec}

\setbeamertemplate{footline}[frame number]
\parindent0pt
\parskip1.2ex

\def\nat{{\mathbb N}}
\def\bool{{\mathbb B}}
\def\real{{\mathbb R}}

\title{Implementierung einer Lateingrammatik im Grammatical Framework \\ \quad \\ \large Kolloquium Computerlinguistisches Arbeiten SS 2013 }
\author{Herbert Lange}

\date{\today} 

\begin{document}
\frame{\titlepage}

\section*{Überblick}
\subsection*{Inhalt}
\begin{frame}{Inhalt}
%\begin{multicols}{2}
\tableofcontents
%\end{multicols}
\end{frame}
\section{Einführung}
\subsection{Das Grammatical Framework}
\begin{frame}[fragile]{Das Grammatical Framework}
\begin{itemize}
\item Mächtigkeit äquivalent zu PMCFG (Parallel Multiple Context-Free Grammars)
$\rightarrow$ zwischen mild und voll kontext-sensitiv
\item Trennung von abstrakter und konkreter Syntax
\item Verschiedene API-Ebenen und Einbindungsmöglichkeiten (u.a. Java, JavaScript)
\end{itemize}
\end{frame}
\subsection{Die Ressource Grammar Library}
\begin{frame}{Die Ressource Grammar Library}
\begin{itemize}
  \item Minimaler Satz gemeinsamer Bestandteile verschiedener Sprachen (Beispielvokabular, Wort-/Satzarten, Syntaxregeln)
  \item ca. 38 Sprachen voll oder teilweise umgesetzt
  \item ca. 43 geschlossene Kategorien (Determiner, ...) und Phrasentypen
  \item ca. 22 offene Kategorien (Nomen, Verben, Adjektive, ...)
\end{itemize}
\end{frame}
\begin{frame}

\begin{block}{Beispiel}
{\scriptsize\ttfamily
> i alltenses/LangGer.gfo \\
linking ... OK \\
Languages: LangGer \\
47884 msec \\
Lang> p ''der Mann sieht die Frau'' \\
PhrUtt NoPConj (UttS (UseCl (TTAnt TPres ASimul) PPos (PredVP (DetCN (DetQuant DefArt NumSg) (UseN man\_N)) (ComplSlash (SlashV2a see\_V2) (DetCN (DetQuant DefArt NumSg) (UseN woman\_N)))))) NoVoc
}
\end{block}
{\tiny
\begin{tabular}{|l|l|l|l|}
\hline
Funktion & Wert & Argumente & Übersetzung\\
\hline
PhrUtt & Phr & (PConj) Utt (Voc) & der Mann sieht die Frau \\
NoPConj & PConj &  & (Keine Konjunktion) \\
UttS & Utt & S & der Mann sieht die Frau \\
UseCl & S & (Temp) (Pol) Cl & der Mann sieht die Frau \\
TTAnt & Temp & Tense Ant & (Tempus und Aspekt) \\
TPres & Tense & & (Präsens) \\
ASimul & Ant & & (Gleichzeitigkeit) \\
PPos & Pol & & (Positive Aussage) \\
PredVP & Cl & NP VP & der Mann sehen die Frau \\
DetCN & NP & Det CN & der Mann \\
DetQuant & Det & Quant Num & der \\
DefArt & Quant & & der \\
NumSg & Num & & (Singular) \\
UseN & CN & N & Mann \\
man\_N & N & & Mann \\
ComplSlash & VP & VPSlash NP & sehen die Frau \\
%SlashV2a & VPSlash & V2 & sehen \\
%see\_V2 & V2 & & sehen \\
\multicolumn{4}{|c|}{...} \\
NoVoc & Voc & & (Keine Anrede) \\
\hline
\end{tabular}
}
\end{frame}
\subsection{Die Lateinische Sprache}
\begin{frame}[fragile]{Die Lateinische Sprache}
Teil der indogermanischen Sprachfamilie $\rightarrow$ Ähnlichkeiten zu germanischen Sprachen so wie dem Griechischen \\
\begin{center}
\begin{tabular}{|l|l|l|}
  \hline
  lateinisch & griechisch & deutsch \\
  \hline
  pater & patēr & Vater \\
  ager & agrós & Acker \\
  trēs & treĩs & drei \\
  decem & déka & zehn \\
  est & estí & ist \\
  \hline
\end{tabular}
\end{center}
Ursprünglich Sprache der Bewohner der mittelitalienischen Region Latium
\end{frame}
\begin{frame}{Sprachliche Besonderheiten}
\begin{itemize}
  \item Sehr freie Satzstellung $\rightarrow$ aber meist Verwendung von Subjekt-Objekt-Verb
  \item Flektierende Sprache mit synthetischer Syntax (Abl. abs. und AcI)
    \begin{itemize}
      \item Augusto regente pax erat in toto imperio romano $\rightarrow$ Als/weil Augustus regierte, herrschte im ganzen römischen Reich Frieden.
      \item Imperatorem venire audit $\rightarrow$ Er hört, dass der Imperator kommt.
    \end{itemize}
  \item Ablativ und Vokativ als eigene Fälle
\end{itemize}
\end{frame}
\section{Umsetzung}
\begin{frame}{Umsetzung}
\begin{itemize}
  \item Lexikon der RGL
  \item Morphologie der RGL
  \item Syntax der RGL
  \item Optional: rein sprachspezifische Konstrukte
\end{itemize}
\end{frame}
\subsection{Lexikon}
\begin{frame}{Lexikon}
\begin{itemize}
  \item Ca. 400 vorgegebene englische Wörter aus allen möglichen Wortarten und verschiedenen Bereichen
  \item Keine Klärung von Ambiguitäten (z.B. bank) $\rightarrow$ willkürliche Wahl der Übersetzung
  \item Viele moderne Begriffe (Auto, Eisenbahn, Computer, Fernseher) $\rightarrow$ Übersetzung mit Hilfe der Wikipedia
  \item Begriffe ohne auffindbare Übersetzung (ein-/ausschalten) $\rightarrow$ Wahl eines naheliegenden Wortes (accendere/exstinguere)
  \item Manche Differenzierungen nicht in jeder Sprache möglich
\end{itemize}
\end{frame}
\begin{frame}{}
\begin{block}{Beispiel}
{\scriptsize\ttfamily
  blue\_A = mkA ( variants { ''caeruleus'' ; ''caerulus'' } ) ; -- 3 L... \\
  boat\_N = mkN ''navicula'' ; -- -ae f. L... \\
  book\_N = mkN ''liber'' ''libri'' masculine ; -- Ranta; -bri m. L... \\
  boot\_N = mkN ''calceus'' ; -- -i m. L... \\
  boss\_N = mkN ''dux'' ''ducis'' ( variants { feminine ; masculine } ) ; -- ducis m./f. L... \\
  boy\_N = mkN ''puer'' ''pueri'' masculine ; -- -eri m. L... \\
  bread\_N = ( variants { (mkN ''panis'' ''panis'' masculine ) ; mkN ''pane'' } ) ; -- -is m./n. L...  \\
  break\_V2 = mkV2 (mkV ''rumpo'' ''rupi'' ''ruptum'' ''rumpere'') ; -- Ranta; 3 L...
}
\end{block}
\end{frame}
\subsection{Morphologie}
\begin{frame}{Morphologie}
\begin{itemize}
\item Einteilung in vier Gruppen: Nomina (Substantive, Adjektive, Pronomina, Numeralia), Verben, Partikel (Adverbien, Präpositionen, Konjunktionen), Interjektionen. Es gibt keine Artikel
\item Nomina und Verben werden flektiert, Partikel und Interjektionen werden nicht flektiert
\item Fünf Nomendeklinationen, drei Adjektivdeklinationen, Adjektivkomparation, Adverbbildung, vier Verbkonjugationen, Deponentia
\end{itemize}
\end{frame}
\begin{frame}{Smart Paradigms}
Pattern Matching um für ein Wort aus möglichst wenig Formen das Paradigma zu bestimmen
\begin{block}{Beispiel}
{\scriptsize\ttfamily
  noun : Str -> Noun = \textbackslash verbum ->  \\
    case verbum of \{ \\
      \_ + ''a''  => noun1 verbum ; \\
      \_ + ''us'' => noun2us verbum ; \\
      \_ + ''um'' => noun2um verbum ; \\
      \_ + ''er'' => noun2er verbum ;  \\
      \_ + ''u''  => noun4u verbum ;  \\
      \_ + ''es'' => noun5 verbum ;  \\
      \_  => noun3 verbum \\
      \} ; 
}
\end{block}
\end{frame}
\subsection{Syntax}
\begin{frame}{Syntax}
\begin{itemize}
\item Problem: Freie Wortstellung im Satz $\rightarrow$ Variationen erhöhen die Komplexität exponentiell
\item Zunächst: Beschränkung auf SOV-Wortstellung
\item Optional: Suche nach performanten Implementierungsmöglichkeiten für beschränkte oder volle Variation
\end{itemize}
\end{frame}
\subsection{Ausblick}
\begin{frame}{Ausblick}
\begin{itemize}
\item Anbinden eines größeres Lexikons
\item Integration in eine weboberfläche oder eine Android-App
\item Benutzung für die Lehre (Translation Quiz)
\end{itemize}
\end{frame}
\section{Literatur}
\begin{frame}{Literatur}
\begin{itemize}
  \item Bayer, Karl u. Lindauer, Josef: Lateinische Grammatik, C.C. Buchners Verlag, Bamberg 1994
  \item Leuman, M./Hofmann, J.B./ Szantyr, A.: Lateinische Grammatik. Auf der Grundlage des Werkes von Friedrich Stolz und Joseph Hermann Schmalz; Band 2: Lateinische Syntax und Stilistik, C.H. Beck'sche Verlagsbuchhandlung, München 1972
  \item Ranta, Aarne: Grammatical Framework. Programming with Multilingual Grammars, CSLI Publications, Stanford 2011 
  \item Ranta, Aarne: Grammatical Framework Tutorial, http://www.grammaticalframework.org/doc/tutorial/gf-tutorial.html 2010
\end{itemize}
\end{frame}
\end{document}
