%\documentclass[handout]{beamer}
\documentclass{beamer}
%\usetheme{Singapore} % anderes Layout
%\usetheme{Antibes} % anderes Layout
%\usecolortheme{lily}
\usepackage{color}
\usepackage{german}
\usepackage{latexsym,amssymb}
\usepackage{amsmath} % für begin{cases} ... \end{cases}
%\usepackage[utf8]{inputenc}
%\usepackage[T1]{fontenc}
%\usepackage{algorithm}
%\usepackage{algorithmicx}
%\usepackage{algpseudocode}
\usepackage{multicol}
\usepackage{graphicx}
\usepackage{fontspec}

\setbeamertemplate{footline}[frame number]
\parindent0pt
\parskip1.2ex

\def\nat{{\mathbb N}}
\def\bool{{\mathbb B}}
\def\real{{\mathbb R}}

\title{Implementierung einer Lateingrammatik im Grammatical Framework \\ \quad \\ \large Kolloquium Computerlinguistisches Arbeiten SS 2013 }
\author{Herbert Lange}

\date{\today} 

\begin{document}
\frame{\titlepage}

\section*{Überblick}
\subsection*{Inhalt}
\begin{frame}{Inhalt}
%\begin{multicols}{2}
\tableofcontents
%\end{multicols}
\end{frame}
\section{Einführung}
\subsection{Das Grammatical Framework}
\begin{frame}[fragile]{Das Grammatical Framework}
\begin{itemize}
\item Mächtigkeit äquivalent zu PMCFG (Parallel Multiple Context-Free Grammars)
$\rightarrow$ zwischen mild und voll kontext-sensitiv
\item Trennung von abstrakter und konkreter Syntax
\item Verschiedene API-Ebenen und Einbindungsmöglichkeiten (u.a. Java, JavaScript)
\end{itemize}
\end{frame}
\subsection{Die Ressource Grammar Library}
\begin{frame}{Die Ressource Grammar Library}
\begin{itemize}
  \item Minimaler Satz gemeinsamer Bestandteile verschiedener Sprachen (Beispielvokabular, Wort-/Satzarten, Syntaxregeln)
  \item ca. 38 Sprachen voll oder teilweise umgesetzt
  \item ca. 43 geschlossene Kategorien (Determiner, ...) und Phrasentypen
  \item ca. 22 offene Kategorien (Nomen, Verben, Adjektive, ...)
\end{itemize}
\end{frame}
\begin{frame}

\begin{block}{Beispiel}
{\scriptsize\ttfamily
> i alltenses/LangGer.gfo \\
linking ... OK \\
Languages: LangGer \\
47884 msec \\
Lang> p ''der Mann sieht die Frau'' \\
PhrUtt NoPConj (UttS (UseCl (TTAnt TPres ASimul) PPos (PredVP (DetCN (DetQuant DefArt NumSg) (UseN man\_N)) (ComplSlash (SlashV2a see\_V2) (DetCN (DetQuant DefArt NumSg) (UseN woman\_N)))))) NoVoc
}
\end{block}
{\tiny
\begin{tabular}{|l|l|l|l|}
\hline
Funktion & Wert & Argumente & Übersetzung\\
\hline
PhrUtt & Phr & (PConj) Utt (Voc) & der Mann sieht die Frau \\
NoPConj & PConj &  & (Keine Konjunktion) \\
UttS & Utt & S & der Mann sieht die Frau \\
UseCl & S & (Temp) (Pol) Cl & der Mann sieht die Frau \\
TTAnt & Temp & Tense Ant & (Tempus und Aspekt) \\
TPres & Tense & & (Präsens) \\
ASimul & Ant & & (Gleichzeitigkeit) \\
PPos & Pol & & (Positive Aussage) \\
PredVP & Cl & NP VP & der Mann sehen die Frau \\
DetCN & NP & Det CN & der Mann \\
DetQuant & Det & Quant Num & der \\
DefArt & Quant & & der \\
NumSg & Num & & (Singular) \\
UseN & CN & N & Mann \\
man\_N & N & & Mann \\
ComplSlash & VP & VPSlash NP & sehen die Frau \\
%SlashV2a & VPSlash & V2 & sehen \\
%see\_V2 & V2 & & sehen \\
\multicolumn{4}{|c|}{...} \\
NoVoc & Voc & & (Keine Anrede) \\
\hline
\end{tabular}
}
\end{frame}
\subsection{Die Lateinische Sprache}
\begin{frame}[fragile]{Die Lateinische Sprache}
Teil der indogermanischen Sprachfamilie $\rightarrow$ Ähnlichkeiten zu germanischen Sprachen so wie dem Griechischen \\
\begin{center}
\begin{tabular}{|l|l|l|}
  \hline
  lateinisch & griechisch & deutsch \\
  \hline
  pater & patēr & Vater \\
  ager & agrós & Acker \\
  trēs & treĩs & drei \\
  decem & déka & zehn \\
  est & estí & ist \\
  \hline
\end{tabular}
\end{center}
Ursprünglich Sprache der Bewohner der mittelitalienischen Region Latium
\end{frame}
\begin{frame}{Sprachliche Besonderheiten}
\begin{itemize}
  \item Sehr freie Satzstellung $\rightarrow$ aber meist Verwendung von Subjekt-Objekt-Verb
  \item Flektierende Sprache mit synthetischer Syntax (Abl. abs. und AcI)
    \begin{itemize}
      \item Augusto regente pax erat in toto imperio romano $\rightarrow$ Als/weil Augustus regierte, herrschte im ganzen römischen Reich Frieden.
      \item Imperatorem venire audit $\rightarrow$ Er hört, dass der Imperator kommt.
    \end{itemize}
  \item Ablativ und Vokativ als eigene Fälle
\end{itemize}
\end{frame}
\section{Umsetzung}
\begin{frame}{Umsetzung}
\begin{itemize}
  \item Lexikon der RGL
  \item Morphologie der RGL
  \item Syntax der RGL
  \item Optional: rein sprachspezifische Konstrukte
\end{itemize}
\end{frame}
\subsection{Lexikon}
\begin{frame}{Lexikon}
\end{frame}
\subsection{Morphologie}
\begin{frame}{Morphologie}
\end{frame}
\subsection{Syntax}
\begin{frame}{Syntax}
\end{frame}
\section{Literatur}
\begin{frame}{Literatur}
\begin{itemize}
  \item Ranta, Aarne: Grammatical Framework; Programming with Multilingual Grammars, CSLI Publications, Stanford 2011 
  \item Bayer, Karl u. Lindauer, Josef: Lateinische Grammatik, C.C. Buchners Verlag, Bamberg 1994
\end{itemize}
\end{frame}
\end{document}
