%\documentclass[handout]{beamer}
\documentclass{beamer}
%\usetheme{Singapore} % anderes Layout
%\usetheme{Antibes} % anderes Layout
%\usecolortheme{lily}
\usepackage{color}
\usepackage{german}
\usepackage{latexsym,amssymb}
\usepackage{amsmath} % für begin{cases} ... \end{cases}
\usepackage[utf8]{inputenc}
\usepackage[T1]{fontenc}
\usepackage{algorithm}
\usepackage{algorithmicx}
\usepackage{algpseudocode}
\usepackage{multicol}
\usepackage{graphicx}


\setbeamertemplate{footline}[frame number]
\parindent0pt
\parskip1.2ex

\def\nat{{\mathbb N}}
\def\bool{{\mathbb B}}
\def\real{{\mathbb R}}

\title{Implementierung einer Lateingrammatik im Grammatical Framework \\ \quad \\ \large Kolloquium Computerlinguistisches Arbeiten SS 2013 }
\author{Herbert Lange}

\date{\today} 

\begin{document}
\frame{\titlepage}

\section*{Überblick}
\subsection*{Inhalt}
\begin{frame}{Inhalt}
%\begin{multicols}{2}
\tableofcontents
%\end{multicols}
\end{frame}
\section{Einführung}
\subsection{Das Grammatical Framework}
\begin{frame}[fragile]{Titel}
Inhalt
\end{frame}
\subsection{Die Ressource Grammar Library}
\subsection{Die Lateinische Sprache}
\begin{frame}[fragile]{Die Lateinische Sprache}
\begin{itemize}
\item Freie Wortstellung
\end{itemize}
\end{frame}
\section{Umsetzung}
\subsection{Lexikon}
\subsection{Morphologie}
\subsection{Syntax}
\section{Literatur}
\begin{frame}{Literatur}
\begin{itemize}
\item Ranta, Aarne: ... \\
\end{itemize}
\end{frame}
\end{document}
