\subsection{Sprachwissenschaftliche Einordnung}
\label{subsec:sprachwissenschaft}
Die lateinische Sprache, auch als oskisch-umbrische Sprache bezeichnet, gehört zur indogermanischen Sprachfamilie und dort zur Unterfamilie der italischen Sprachen. Durch diese Verwandtschaft kann man bei Wörtern und Wortformen oft Entsprechungen zwischen der lateinischen Sprache und verschiedensten anderen Sprachen Westeuropas bis hin zu Mittelasien finden (vgl. Tabelle \ref{Idg-Entsprechungen}).\footnote{vgl. \cite{BAYER-LINDAUER1994} S.1}
\begin{table}[h]
\begin{tabular}{|l|l|l|}
\hline
lateinisch & altgriechisch & deutsch \\
\hline
pater & \pi\alpha\tau\'{\eta}\rho\ (=patēr) & Vater \\
ager & \alpha\gamma\rho\'{o}\varsigma\ (=agr\'{o}s)& Acker \\
trēs & \tau\rho\varepsilon\~{\iota}\varsigma\ (=treĩs) & drei \\
decem & \delta\'{\varepsilon}\kappa\alpha\ (=d\'{e}ka) & zehn \\
\hline
\end{tabular}
\caption{Wortentsprechungen in verschiedenen indogermanischen Sprachen (vgl. \cite{BAYER-LINDAUER1994} S.1)}
\label{Idg-Entsprechungen}
\end{table}
Entstanden ist es als ein in der Stadt Rom üblicher Dialekt parallel zu anderen ländlicheren Dialekten im Latium, einer Region in Mittelitalien. Im Laufe der Zeit verdrängte es jedoch die weiteren italischen Sprachen im Zuge der Ausdehnung des römischen Reichs.\footnote{vgl. \cite{METZLER2004} Lateinisch: S. 5359} \par
Die Sprachgeschichte kann in mehrere Epochen unterteilt werden. Üblicherweise beginnt man diese Einordnung mit der Epoche des Altlateins, das von ca. 240 v. Chr. bis 80 v. Chr. angesiedelt wird. Es reicht von den frühesten nachgewiesenen lateinischen Sprachzeugnissen bis zum Beginn der Zeit des klassischen Lateins. Dessen Zeitraum wird von ca. 80 v. Chr. bis 117. n. Chr. gerechnet und beginnt in etwa mit den ersten öffentlichen Auftritten des M. Tullius Cicero. Die bekannten Gerichtsreden des berühmten römischen Anwalts und Schriftstellers von ca. 80 v. Chr. sind noch größtenteils erhalten. Die nach-klassische Phase kann wiederum in verschiedene Epochen unterteilt werden, in denen unter anderem die romanischen Volkssprachen entstanden sind, bis hin zum so genannten Neulatein, das vom 15. Jahrhundert bis hin zum Beginn des 20. Jahrhundert die Sprache der Wissenschaft darstellte und noch immer großen Einfluss auf Begriffe des Alltags ausübt.\footnote{vgl. \cite{MUELLER-LANCE2006} S. 27ff.}  \par
Auch heute noch am bedeutendsten ist wohl das klassische Latein, das weiterhin in Schulen unterrichtet wird und sich vor allem mit seinem großen überlieferten Textkorpus hervorhebt. Da sich die meisten Lateingrammatiken auf diese Sprachepoche stützen, wird in dieser Arbeit primär das klassische Latein betrachtet. \par
In der Sprachwissenschaft ist auch weiterhin umstritten, in welchem Verhältnis das klassische Latein zum so genannten Vulgärlatein steht. Heutzutage geht man davon aus, dass das klassische Latein eine kaum wirklich gesprochene Sprache war und das Vulgärlatein nicht nur eine nach-klassische Sprachvariante ist, sondern bereits parallel zum klassischen Schriftlatein als gesprochene Sprache verwendet wurde. Allerdings fand das klassische Latein noch bis in das 5. Jahrhundert n. Chr. Verwendung als eine Art Schreibnorm, während sich das Vulgärlatein langsam zu den romanischen Sprachen weiterentwickelte.\footnote{vgl. \cite{METZLER2004} Lateinisch: S. 5359 und Vulgärlatein: S. 10719} \par
Formal gehört Latein zu den stark flektierenden Sprachen. Das heißt, dass in der lateinischen Sprache, wie für synthetische Sprachen üblich, syntaktische Klassen und Verhältnisse über Wortsuffixe ausgedrückt werden.\footnote{vgl. \cite{METZLER2004} Synthetisch: S. 9690}. Allerdings drücken bei flektierenden Sprachen, im Gegensatz zu agglutinierenden Sprachen, die Affixe meist mehr als ein grammatisches Merkmal aus.\footnote{vgl. \cite{METZLER2004} Flektierende Sprache: S. 3009} So ist bei der Verbform \textit{audio} das \textit{audi} der Verbstamm, um genau zu sein der Präsensstamm, des Verbs \textit{audire} und das Suffix \textit{-o} kodiert folgende Merkmale: 1. Person, Singular, Präsens, Indikativ, Aktiv.\footnote{vgl. \cite{BAYER-LINDAUER1994} S. 75} \par
Es gibt für Nomen fünf zum Teil genusbasierte Flexionsklassen, also verschiedene Typen der Flexion innerhalb einer Wortart,\footnote{vgl. \cite{METZLER2004} Flexion: S. 3011}, sechs verschiedene Kasus (Nominativ, Genitiv, Dativ, Akkusativ, Ablativ und Vokativ) und drei Genera (Maskulin, Feminin, Neutrum). Des weiteren gibt es ein voll flektierendes Pronomensystem und vier relativ stark synthetische Flexionsklassen für Verben.\footnote{\cite{METZLER2004} Lateinisch: S. 5359} Zu den Kasus sei anzumerken, dass der Ablativ im Lateinischen ein eigenständiger Kasus ist, jedoch der Vokativ oft mit dem Nominativ zusammenfällt.\footnote{vgl. \cite{BAYER-LINDAUER1994} S. 20f.} \par
Die Wortstellung des Lateinischen wird oft als sehr frei beschrieben, allerdings gibt es eine klare Präferenz der SOV-Wortstellung im Satz, also dass das Objekt des Satzes direkt auf das Subjekt folgt, und das Verb den Satz abschließt. Die Möglichkeiten zur Positionierung des Adjektivs im Bezug auf das Nomen sind allerdings durch nichts beschränkt.\footnote{\cite{METZLER2004} Lateinisch: S. 5359}
\subsection{Bedeutung in der heutigen Zeit}
\label{subsec:bedeutung}
Man kann sich natürlich über die Notwendigkeit streiten, sich in der heutigen Zeit noch mit der lateinischen Sprache zu beschäftigen. Es gibt aber auch ziemlich gute Gründe dafür, Latein nicht einfach als tote Sprache abzustempeln und nicht weiter zu betrachten. \par
Der am häufigsten, vor allem im Schulalter bei der Wahl einer zu lernenden Fremdsprache, vorgebrachte Grund ist, dass die lateinische Sprache als "`Mutter aller romanischen Sprachen"' später einen einfacheren Einstieg in das Erlernen z.B von Französisch oder Spanisch bietet. Auch galt Latein seit Jahrhunderten, und gilt weiterhin, als produktive Quelle für Fachbegriffe aus Wissenschaft, Forschung und Technik. So haben viele moderne Begriffe wie Computer\footnote{von lat. \textit{computere} - berechnen} und Monitor\footnote{von lat. \textit{monere} - mahnen} lateinische Wurzeln. Auch im Universitätsalltag wird man oft mit lateinischen Lehnwörtern konfrontiert. Man trifft sich zum Essen in der Mensa\footnote{von lat. \textit{mensa} - Tisch, Tafel} und studiert an Fakultäten\footnote{von lat. \textit{facultas} - Vermögen, Fähigkeit}. \par
Vor allem in der Sprachwissenschaft hat die lateinische Sprache eine besondere Bedeutung, da sie bei einem Vergleich verschiedener indogermanischer Sprachen als eine Art "`default"'-Sprache angesehen werden kann, denn sie bietet fast alle grammatischen Kategorien, die gewöhnlich benötigt werden. So kann Latein als Vergleichsparameter (\textit{tertium comparationis}) verwendet werden. Diese Stellung der lateinischen Sprache spiegelt sich auch in der Fachterminologie moderner Schulgrammatiken wieder, die fast ausschließlich von lateinischen Fachausdrücken geprägt ist.\footnote{vgl. \cite{MUELLER-LANCE2006} S. 10} \par
Als etwas ungewöhnliche aber noch relativ moderne Verwendung der lateinischen Sprache kann in \textit{latino sine flexione} gesehen werden. Diese von Giuseppe Peano, Anfang des 20. Jahrhunderts als Welthilfssprache entwickelte, vereinfachte Form der lateinischen Sprache fand bis ca. 1950 in mehreren wissenschaftlichen Veröffentlichungen Verwendung. Sie basiert auf dem üblichen lateinischen Wortschatz, der auch durch modernes romanisches Vokabular erweitert werden kann, und einer stark vereinfachten Morphologie.\footnote{vgl. \cite{METZLER2004} Latino sine flexione S. 5374}