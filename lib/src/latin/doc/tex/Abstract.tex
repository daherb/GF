\begin{abstract}
In dieser Arbeit sollen an einem konkreten Beispiel die nötigen Schritte gezeigt werden, um eine computergestützte Grammatik für eine natürliche Sprache zu entwerfen. Am Beispiel der lateinischen Sprache wird gezeigt, wie eine Grammatik, bestehend aus einem Lexikon, einem Morphologiesystem und einer Syntax, implementiert werden kann, die sich in ein größeres, multilinguales Grammatiksystem einpassen lässt. Dadurch können zum einen in der implementierten Sprache Sätze verarbeitet werden, aber auch in jede andere im System vorhandene Sprache übersetzt werden. Gezeigt wird ein rein regelbasierter Ansatz der sich von den statistischen Methoden durch seine Striktheit und Beschränktheit, aber auch durch seine Zuverlässigkeit abhebt.
\end{abstract}