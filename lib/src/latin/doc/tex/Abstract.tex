\begin{abstract}
In dieser Arbeit sollen an einem konkreten Beispiel die nötigen Schritte gezeigt werden, um eine computergestützte Grammatik für eine natürliche Sprache zu entwerfen. Es wird anhand der lateinischen Sprache gezeigt, wie eine Grammatik, bestehend aus einem Lexikon, einem Morphologiesystem und einer Syntax, implementiert werden kann, die sich in einem größeren, multilingualen Grammatiksystem verwenden lässt. Dadurch können in der implementierten Sprache Sätze verarbeitet werden und auch in jede andere im System vorhandene Sprache übersetzt werden. Gezeigt wird ein rein regelbasierter Ansatz der sich von den statistischen Methoden durch seine Striktheit und Beschränktheit, aber auch durch seine Zuverlässigkeit abheben soll.
\end{abstract}