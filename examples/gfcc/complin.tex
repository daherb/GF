\documentclass[12pt]{article}

\usepackage{isolatin1}

\setlength{\oddsidemargin}{0mm}
%\setlength{\evensidemargin}{0mm}
\setlength{\evensidemargin}{-2mm}
\setlength{\topmargin}{-16mm}
\setlength{\textheight}{240mm}
\setlength{\textwidth}{158mm}

%\setlength{\parskip}{2mm}
%\setlength{\parindent}{0mm}

\input{macros}

\newcommand{\begit}{\begin{itemize}}
\newcommand{\enit}{\end{itemize}}
\newcommand{\newone}{} %%{\newpage}
\newcommand{\heading}[1]{\subsection{#1}}
\newcommand{\explanation}[1]{{\small #1}}
\newcommand{\empha}[1]{{\em #1}}
\newcommand{\rarrow}{\; \rightarrow\;}

\newcommand{\nocolor}{} %% {\color[rgb]{0,0,0}}


\title{{\bf Single-Source Language Definitions and Code Generation as Linearization}}

\author{Aarne Ranta \\
  Department of Computing Science \\
  Chalmers University of Technology and the University of Gothenburg\\
  {\tt aarne@cs.chalmers.se}}

\begin{document}

\maketitle


\section{Introduction}

The experiment reported in this paper was prompted by a challenge
posted by Lennart Augustsson to the participants of the workshop
on Dependent Types in Programming held at Dagstuhl in September 2004.
The challenge was to use dependent types to write a compiler from
C to bytecode. This paper does not meet the challenge quite literally,
since our compiler is for a different subset of C than Augustsson's
specification, and since the bytecode that we generate is JVM instead
of his format. But it definitely makes use of dependent types.

Augustsson's  challenge did not specify \textit{how} dependent
types are to be used, and the first of the two points we make in this
paper (and its title) reflects our interpretation:
we use dependent types, in combination with higher-order abstract syntax (HOAS),
to define the grammar of the source language (here, the fragment of C).
The grammar constitutes the single, declarative source from which
the compiler front end is derived, comprising both parser and type
checker.

The second point, code generation by linearization, means that the
back end is likewise implemented by a grammar of the target
language (in this case, a fragment of JVM). This grammar is the
declarative source from which the compiler back end is derived.

The complete code of the compiler is 300 lines. It is found in 
the appendices of this paper.



\section{The Grammatical Framework}

The tool we have used for implementing the compiler is
GF, the \empha{Grammatical Framework} \cite{gf-jfp}. GF 
is similar to a Logical Framework (LF) extended with
a notation for defining concrete syntax. GF was originally
designed to help building multilingual
translation systems for natural languages and also
between formal and natural languages. The translation model
implemented by GF is very simple:
\begin{verbatim}


             parsing                       linearization
             ------------>                 ------------>
  Language_1               Abstract Syntax               Language_2
             <------------                 <------------
             linearization                 parsing
\end{verbatim}
An abstract syntax is similar to a \empha{theory}, or a
\empha{signature} in a logical framework. A 
concrete syntax defines, in a declarative way,
a translation of abstract syntax trees (well-formed terms) 
into concrete language structures, and from this definition, one can
derive both both linearization and parsing.
The number of languages related to one abstract syntax in
this way is of course not limited to two. Sometimes just just one
language is involved; 
GF then works much the same way as any grammar formalism or parser
generator. 
The largest application known to us has 88 languages and translates
numeral expressions from 1 to 999,999 between them \cite{gf-homepage}.
 
From the GF point of view, the goal of the compiler experiment
is to investigate if GF is capable of implementing
compilers using the ideas of single-source language definition
and code generation as linearization. The working hypothesis
was that it \textit{is} capable but inconvenient, and that,
working out a complete example, we would find out what 
should be done to extend GF into a compiler construction tool.


\subsection{Advantages and disadvantages}

Due to the way in which it is built, our compiler has
a number of unusual, yet attractive features:
\bequ
The front end is defined by a grammar of C as its single source.

The grammar defines both abstract and concrete syntax, and also
semantic well-formedness (types, variable scopes).

The back end is implemented by means of a grammar of JVM providing
another concrete syntax to the abstract syntax of C.

As a result of the way JVM is defined, only semantically well formed
JVM programs are generated.

The JVM grammar can also be used as a decompiler, which translates
JVM code back into C code.

The language has an interactive editor that also supports incremental
compilation.
\enqu
The problems that we encountered and their causes will be explained in 
the relevant sections of this report. To summarize,
\bequ
The scoping conditions resulting from HOAS are slightly different
from the standard ones of C.

Our JVM syntax is forced to be slightly different from original.

Using HOAS to encode all bindings is sometimes cumbersome.

The C parser derived from the GF grammar does not recognize all
valid programs.
\enqu
The first two shortcomings seem to be inevitable with the technique
we use. The best we can do with the JVM syntax is to use simple
postprocessing, on string level, to obtain valid JVM. The latter 
two shortcomings have to do with the current implementation of GF
rather than its core. They
suggest that GF should be fine-tuned to give better support
to compiler construction, which, after all, is not an intended
use of GF as it is now.



\section{The abstract syntax}

An \empha{abstract syntax} in GF consists of \texttt{cat} judgements
\[
\mbox{\texttt{cat}} \; C \; \Gamma
\]
declaring basic types (depending on a context $\Gamma$), 
and \texttt{fun} judgements 
\[
\mbox{\texttt{fun}} \; f \; \mbox{\texttt{:}} \; A
\]
declaring functions $f$ of any type $A$, which can be a basic type or
a function type.
\empha{Syntax trees} are well-formed terms of basic
types. As for notation, each judgement form is recognized by
its keyword (\texttt{cat}, \texttt{fun}, etc), 
and the same keyword governs all judgements
until the next keyword is encountered.



\subsection{Statements}

Statements in C may involve variables, expressions, and
other statements.
The following \texttt{cat} judgements of GF define the syntactic categories
that are needed to construct statements
\begin{verbatim}
  cat
    Stm ;
    Typ ;
    Exp Typ ;
    Var Typ ;
\end{verbatim}
The type \texttt{Typ} is the type of C's datatypes.
Expressions (\texttt{Exp}) is a dependent type, since it has a
nonempty context. The rules for \texttt{Exp}
will thus be rules to construct well-typed expressions of
a given type. \texttt{Var}\ is the type of variables,
of a given type, that get bound in C's variable
declarations.

Let us start with the simplest kind of statements:
declarations and assignments. The following \texttt{fun}
rules define their abstract syntax:
\begin{verbatim}
  fun
    Decl   : (A : Typ) -> (Var A -> Stm) -> Stm ;
    Assign : (A : Typ) -> Var A -> Exp A -> Stm -> Stm ;
\end{verbatim}
Dependent types are used in \texttt{Assign} to
control that a variable is always assigned a value of proper
type. The \texttt{Decl}\ function captures the rule that
a variable must be declared before it can be used or assigned to:
its second argument is a \empha{continuation}, which is
the sequence of statements that depend on (= may refer to)
the declared variable. 

We will treat all statements, except
\texttt{return}s, in terms of continuations. A sequence of
statements (which always has the type \texttt{Stm}) thus
always ends in a \texttt{return}, or, abruptly, in
an empty statement, \texttt{End}. Here are rules for some other
statement forms:
\begin{verbatim}
    Return : (A : Typ) -> Exp A -> Stm ;
    While  : Exp TInt -> Stm -> Stm -> Stm ;
    IfElse : Exp TInt -> Stm -> Stm -> Stm -> Stm ;
    Block  : Stm -> Stm -> Stm ;
    End    : Stm ;
\end{verbatim}
Here is an example of a piece of code and its abstract syntax.
\begin{verbatim}
  int x ;      Decl TInt (\x -> 
  x = 5 ;        Assign TInt x (EInt 5) (
  return x ;       Return TInt (EVar TInt x)))
\end{verbatim}
(Expression syntax will be explained in the next section.)



\subsection{Expressions and types}

We consider two C types: integers and floats.
\begin{verbatim}
    TInt   : Typ ;
    TFloat : Typ ;
\end{verbatim}
Well-typed expressions can be built from constants,
from variables, and by means of binary operations.
\begin{verbatim}
    EVar   : (A : Typ) -> Var A -> Exp A ;
    EInt   : Int -> Exp TInt ;
    EFloat : Int -> Int -> Exp TFloat ;
    ELtI   : Exp TInt -> Exp TInt -> Exp TInt ;
    ELtF   : Exp TFloat -> Exp TFloat -> Exp TInt ;
    EAddI, EMulI, ESubI : Exp TInt -> Exp TInt -> Exp TInt ;
    EAddF, EMulF, ESubF : Exp TFloat -> Exp TFloat -> Exp TFloat ;
\end{verbatim}
Notice that GF has a built-in type \texttt{Int} of
integer literals, but floats are constructed by hand.

Yet another expression for are function calls. To this
end, we need a notions of (user-defined) functions and
argument lists. The type of functions depends on an
argument type list and a value type. Expression lists
are formed to match type lists.
\begin{verbatim}
  cat
    ListTyp ;
    Fun ListTyp Typ ;
    ListExp ListTyp ;

  fun
    EApp    : (AS : ListTyp) -> (V : Typ) -> 
                Fun AS V -> ListExp AS -> Exp V ;

    NilTyp  : ListTyp ;
    ConsTyp : Typ -> ListTyp -> ListTyp ;

    NilExp  : ListExp NilTyp ;
    ConsExp : (A : Typ) -> (AS : ListTyp) -> 
                Exp A -> ListExp AS -> ListExp (ConsExp A AS) ;
\end{verbatim}



\subsection{Functions}

On the top level, a program is a sequence of functions.
Each function may refer to functions defined earlier
in the program. The idea to express the binding of
function symbols with HOAS is analogous to the binding
of variables in statements, using a continuation.
\begin{verbatim}
  cat
    Program ;
    Body ListTyp ;

  fun
    Empty : Program ;
    Funct : (AS : ListTyp) -> (V : Typ) -> 
              Body AS -> (Fun AS V -> Program) -> Program ;
\end{verbatim}
However, here we must also account for the binding of
a function's parameters in its body. We could use
vectors of variables, in the same way as vectors of
expressions are used to give arguments to functions.
However, this would lead to the need of cumbersome
projection functions when using the parameters
in the function body. A more elegant solution is
to use HOAS to build function bodies:
\begin{verbatim}
    BodyNil  : Stm -> Body NilTyp ;
    BodyCons : (A : Typ) -> (AS : ListTyp) -> 
                  (Var A -> Body AS) -> Body (ConsTyp A AS) ;
\end{verbatim}
The end result is an abstract syntax whose relation
to concrete syntax is somewhat remote. Here is an example of
the code of a function and its abstract syntax:
\begin{verbatim}
                    Funct (
  int abs (int x){    ConsTyp TInt NilTyp) TInt 
                      (BodyCons TInt NilTyp (\x -> 
                         BodyNil (
  if (x < 0){              IfElse (ELtI (EVar TInt x) (EInt 0))
    return 0 - x ;           (Block (Return TInt 
                                 (ESubI (EInt 0) (EVar TInt x))) 
    }                          End) 
  else return x ;            (Return TInt (EVar TInt x)) End))) 
  }                             (\abs -> Empty)
\end{verbatim}
Notice, in particular, how far from the function header the
name of the function appears in the syntax trees.

A more serious shortcoming of our way of defining functions 
is that it does not allow recursion. The reason is simple:
the function symbol is only bound in the continuation
of the program, not in the function body. It seems we could
save recursive functions with the following variant:
\begin{verbatim}
  cat
    BodyProgram ListTyp ;
    Body ListTyp ;

  fun
    mkBodyProgram : (AS : ListTyp) -> 
      Body AS -> Program -> BodyProgram ListTyp ;
    RecFunct : (AS : ListTyp) -> (V : Typ) -> 
      (Fun AS V -> BodyProgram AS) -> Program ;
\end{verbatim}
but we have not yet investigated this.



\section{The concrete syntax of C}

A concrete syntax, for a given abstract syntax, 
consists of \texttt{lincat} judgements
\[
\mbox{\texttt{lincat}} \; C \; \mbox{\texttt{=}} \; T
\]
defining the \empha{linearization types} $T$ of each category $C$,
and \texttt{lin} judgements
\[
\mbox{\texttt{lin}} \; f \; \mbox{\texttt{=}} \; t
\]
defining the \empha{linearization functions} $t$ of each function $f$
in the abstract syntax. The linearization functions are
checked to be well-typed with respect the \texttt{lincat}
definitions, and the syntax of GF forces them to be \empha{compositional}
in the sense that the linearization of a complex tree is always
a function of the linearizations of the subtrees. Schematically, if
\[
  f \colon A_{1} \rarrow \cdots \rarrow A_{n} \rarrow A 
\]
then
\[
  \sugmap{f} \colon 
    \sugmap{A_{1}} \rarrow \cdots 
    \rarrow \sugmap{A_{n}} \rarrow \sugmap{A} 
\]
and the linearization algorithm is simply
\[
  \sugmap{(f \; a_{1} \; \ldots \; a_{n})} \; = \;
   \sugmap{f} \; \sugmap{a_{1}} \; \ldots \; \sugmap{a_{n}}
\]
using the \sugmap{} notation for both linearization types, 
linearization functions, and linearizations of trees.

Because of compositionality, no case analysis on expressions
is possible in linearization rules. The values of linearization
therefore have to carry information on how they are used in
different situations. Therefore linearization
types are generally record types instead of just the string type.
The simplest record type that is used in GF is
\begin{verbatim}
  {s : Str}
\end{verbatim}
If the linearization type of a category is not explicitly
given by a \texttt{lincat} judgement, this type is
used by default. 



\subsection{Resource modules}

Resource modules define auxiliary notions that can be
used in concrete syntax. These notions include
\empha{parameter types} defined by \texttt{param}
judgements
\[
\mbox{\texttt{param}} \; P \; \mbox{\texttt{=}} 
  \; C_{1} \; \Gamma_{1} \; \mid \; \cdots \; \mid \; 
  \; C_{n} \; \Gamma_{n}
\]
and \empha{operations} defined by
\texttt{oper} judgements
\[
\mbox{\texttt{oper}} \; f \; \mbox{\texttt{:}} \; T \; \mbox{\texttt{=}} \; t
\]
These judgements are
similar to datatype and function definitions
in functional programming, with the restriction
that parameter types must be finite and operations
may not be recursive. It is due to these restrictions that
we can always derive a parsing algorithm from a set of
linearization rules.

The following string operations are useful in almost
all grammars. They are actually included in a GF \texttt{Prelude},
but are here defined from scratch to make the code shown in
the Appendices complete.
\begin{verbatim}
  oper
    SS  : Type = {s : Str} ;
    ss  : Str -> SS = \s -> {s = s} ;
    cc2 : (_,_ : SS) -> SS = \x,y -> ss (x.s ++ y.s) ;
\end{verbatim}



\subsection{Expressions}

We want to be able to recognixe and generate one and the same expression with
or without parentheses, depending on whether its precedence level
is lower or higher than expected. For instance, a sum used as
an operand of multiplication should be in parentheses. We
capture this by defining a parameter type of
precedence levels. Four levels are enough for the present
fragment of C, so we define the auxiliary notions
\begin{verbatim}
  param Prec = P0 | P1 | P2 | P3 ;
  oper PrecExp : Type = {s : Prec => Str} ;
\end{verbatim}
in a resource module (see Appendix D), and
\begin{verbatim}
  lincat Exp = PrecExp ;
\end{verbatim}
in the concrete syntax of C itself. The following auxiliary notions
are also used in the concrete syntax of C:
\begin{verbatim}
  oper
    paren : Str -> Str = \str -> "(" ++ str ++ ")" ;
    ex : PrecExp -> Str = \exp -> exp.s ! P0 ;
    constant : Str -> PrecExp = \c -> {s = \\_ => c} ;
    infixN : Prec -> Str -> PrecExp -> PrecExp -> PrecExp = \p,f,x,y ->
      {s = mkPrec (x.s ! (nextPrec ! p) ++ f ++ y.s ! (nextPrec ! p)) ! p} ;
    infixL : Prec -> Str -> PrecExp -> PrecExp -> PrecExp = \p,f,x,y ->
      {s = mkPrec (x.s ! p ++ f ++ y.s ! (nextPrec ! p)) ! p} ;
\end{verbatim}
Here are the linearization rules of expressions:
\begin{verbatim}
  lin
    EVar  _ x    = constant x.s ;
    EInt    n    = constant n.s ;
    EFloat a b   = constant (a.s ++ "." ++ b.s) ;
    EMulI, EMulF = infixL P2 "*" ;
    EAddI, EAddF = infixL P1 "+" ;
    ESubI, ESubF = infixL P1 "-" ;
    ELtI,  ELtF  = infixN P0 "<" ;
    EApp args val f exps = constant (f.s ++ paren exps.s) ;
\end{verbatim}
A useful addition to GF when fine-tuned for compiler
construction might be a hard-coded treatment of
precedence---in particular, to permit the addition
of superfluous parentheses, which is not allowed by
the present grammar.



\subsection{Statements}

Statements in C have
the simplest linearization type, \verb6{s : Str}6.
We use a handful of auxiliary operations to regulate
the use of semicolons on a high level.
\begin{verbatim}
  oper
    continues : Str -> SS -> SS = \s,t -> ss (s ++ ";" ++ t.s) ; 
    continue  : Str -> SS -> SS = \s,t -> ss (s ++ t.s) ;
    statement : Str -> SS       = \s   -> ss (s ++ ";"); 
\end{verbatim}
As for declarations, which bind variables, we notice the
special projection \verb6.$06 to refer to the bound variable.
Technically, terms that are linearized must be in $\eta$-long
form, which guarantees that a variable symbol can always be
found, and that a field representing it can be added to the
linearization record.
\begin{verbatim}
  lin
    Decl  typ cont = continues (typ.s ++ cont.$0) cont ;
    Assign _ x exp = continues (x.s ++ "=" ++ ex exp) ;
    Return _ exp   = statement ("return" ++ ex exp) ;
    While exp loop = continue  ("while" ++ paren (ex exp) ++ loop.s) ;
    IfElse exp t f = continue  ("if" ++ paren (ex exp) ++ t.s ++ "else" ++ f.s) ;
    Block stm      = continue  ("{" ++ stm.s ++ "}") ;
    End            = ss [] ;
\end{verbatim}



\subsection{Functions}

The only new problem presented by functions is the proper
distribution of commas in parameter and argument lists.
Since compositionality prevents taking cases of list forms, e.g.
\begin{verbatim}
  -- NilExp = ss [] ;
  -- ConsExp _ _ e NilExp = e ;
  -- ConsExp _ _ e es = ss (e.s ++ "," ++ es.s) ;
\end{verbatim}
we have to encode the size of a list (zero, one, or more) in a parameter:
\begin{verbatim}
  param
    Size = Zero | One | More ;
  oper
    nextSize : Size -> Size = \n -> case n of {
      Zero => One ;
      _    => More
      } ;
    separator : Str -> Size -> Str = \t,n -> case n of {
      Zero => [] ;
     _ => t
     } ;
\end{verbatim}
For expression lists, we now have
\begin{verbatim}
  lincat 
    ListExp = {s : Str ; size : Size} ;
  lin
    NilExp = ss [] ;
    ConsExp _ _ e es = {
      s = ex e ++ separator "," es.size ++ es.s ;
      size = nextSize es.size
      } ;
\end{verbatim}
Parameter lists are collected as components of function bodies
(because of HOAS), and their size is encoded as yet another 
component.
\begin{verbatim}
  lincat
    Body = {s,s2 : Str ; size : Size} ;
  lin
    Empty = ss [] ;
    Funct args val body cont = ss (
      val.s ++ cont.$0 ++ paren body.s2 ++ "{" ++ 
      body.s ++ "}" ++ ";" ++ cont.s) ;

    BodyNil stm = stm ** {s2 = [] ; size = Zero} ;
    BodyCons typ _ body = {
      s  = body.s ; 
      s2 = typ.s ++ body.$0 ++ separator "," body.size ++ body.s2 ;
      size = nextSize body.size
      } ;
\end{verbatim}



\section{The concrete syntax of JVM}

JVM syntax is, linguistically, more straightforward than
the syntax of C, and could be described by a regular
expression. The translation from our abstract syntax to JVM,
however, is tricky because variables are replaced by
their addresses (relative to the frame pointer), and
linearization must therefore maintain a symbol table that permits
the lookup of a variable address. As shown in the code
in Appendix C, we have not quite succeeded to do this
in the code generated by linearization. 
Instead, we use variable symbols instead of addresses
in \texttt{load} and \texttt{store} instructions, and
generate \texttt{alloc} pseudoinstructions from declarations.
Then we use another pass, written in Haskell (Appendix E), 
to replace variable symbols by their addresses. The following example
shows how the three representations (C, pseudo-JVM, JVM) look like 
for a piece of code.
\begin{verbatim}
  int x ;   alloc i x      ; x has address 0
  int y ;   alloc i y      ; y has address 1
  x = 5 ;   i _push 5      ipush 5
            i _store x     istore 0
  y = x ;   i _load x      iload 0
            i _store y     istore 1
\end{verbatim}

A related problem is the generation of fresh labels for
jumps. We solve this by maintaining a growing label suffix
as a field of the linearization of statements into
instructions. The problem remains that the two branches
in an \texttt{if-else} statement can use the same
labels. Making them unique will have to be
added to the post-processing pass. This is
always possible, because labels are nested in a
disciplined way, and jumps can never go to remote labels.

As it turned out laborious to thread the label counter
to expressions, we decided to compile comparison \verb6x < y6
expressions into function calls, which should be provided
by a run-time library. This would no more work for the 
conjunction \verb6x && y6
and disjunction \verb6x || y6, if we want to keep their semantics
lazy, since function calls are strict in their arguments.

The JVM syntax used is from the Jasmin assembler
\cite{jasmin}, with small deviation which will
be removed shortly.



\subsection{A code example}

Here is a complete C source program, the JVM code obtained by linearization, and
the postprocessed JVM code.
\small
\begin{verbatim}
int abs (int x){        .method abs (i)i ;
  if (x < 0){           .limit locals i ;       .method abs(i)i;
    return 0 - x ;      .limit stack 1000 ;     .limit locals 1
    }                   alloc i x ;             .limit stack 1000 ;
  else return x ;       i _load x ;             iload 0
  } ;                   ipush 0 ;               ipush 0 ;
int main () {           call ilt ;              call ilt ;
  int i ;               ifzero FALSE_ ;         ifzero FALSE_;
  i = abs (16);         ipush 0 ;               ipush 0 ;
  } ;                   i _load x ;             iload 0
                        isub ;                  isub ;
                        i _return ;             ireturn
                        ;                       ;
                        goto TRUE_ ;            goto TRUE_;
                        FALSE_ ;                FALSE_ ;
                        i _load x ;             iload 0
                        i _return ;             ireturn
                        TRUE_ ;                 TRUE_ ;
                        .end method ;           .end method ;
                        .method main () i ;     .method main()i;
                        .limit locals i ;       .limit locals 1
                        .limit stack 1000 ;     .limit stack 1000 ;
                        alloc i i ;             ipush 16 ;
                        ipush 16 ;              invoke abs (i)i ;
                        invoke abs (i)i ;       istore 0
                        i _store i ;            .end method ;
                        .end method ;
\end{verbatim}
\normalsize



\subsection{How to restore code generation by linearization}

SInce postprocessing is needed, we have not quite achieved
the goal of code generation as linearization.
If linearization is understood in the
sense of GF. In GF, linearization rules must be
compositional, and can only depend on parameters from
finite parameter sets. Hence it is not possible to encode
linearization with updates to and lookups from a symbol table,
as is usual in code generation. 

Compositionality also prevents optimizations during linearization
by clever instruction selection, elimination of superfluous
labels and jumps, etc.

It would of course be possible to implement the compiler
back end in GF in the traditional way, as a noncompositional
function from the abstract syntax of C to a different abstract
syntax of JVM. The abstract syntax notation of GF permits
definitions of functions, and the GF interpreter can be used
for evaluating terms into normal form. Thus one could write
\begin{verbatim}
  fun 
    transStm : Env -> Stm -> EnvInstr ;
  def
    transStm env (Decl typ rest) = ... 
    transStm env (Assign typ var exp rest) = ... 
\end{verbatim}
This would be cumbersome in practice, because
GF does not have facilities like built-in lists and tuples, 
or monads. Of course, the compiler could no longer be
inverted into a decompiler, in the way true linearization
can be inverted into a parser.

Yet another possibility is to change the abstract syntax
so that variables do not only carry a string with them but
also a relative address. This would certainly be possible
with dependent types; but it would clutter the abstract
syntax in a way that is hard to motivate when we are in
the business of describing the syntax of C.

Perhaps the key idea would be to hard-code some support
for symbol tables into the extension of GF tuned for
compiler construction. For instance, the linearization
of a binding could store, in addition to the variable
symbol field \verb6.$06, an integer-valued fiels \verb6.#06.
These fields correspond to an automatic renaming of variables
to \verb6x1, x2, x36,\ldots starting from the outermost one.
Linearization to C could then use the \verb6.$06 field, as
in this paper, and linearization to JVM the \verb6.#06 field.




\section{Related work}

The theoretical ideas behind our compiler experiment
are familiar from various sources.
Building single-source language definitions with
dependent types and higher-order abstract syntax
has been studied in various logical frameworks 
\cite{harper-honsell,magnusson-nordstr,twelf}.
The idea of using a common abstract syntax for different 
languages was clearly exposed by Landin \cite{landin}. The view of
code generation as linearization is a central aspect of
the classic compiler textbook by Aho, Sethi, and Ullman
\cite{aho-ullman}. 
The use of the same grammar both for parsing and linearization
is a guiding principle of unification-based linguistic grammar 
formalisms \cite{pereira-shieber}. Interactive editors derived from
grammars have been used in various programming and proof
assistants \cite{teitelbaum,metal,magnusson-nordstr}.

Even though the different ideas are well-known, they are
applied less in practice than in theory. In particular,
we have not seen them used together to construct a complete
compiler. In our view, putting these ideas together is
an attractive approach to compiling, since a compiler written
in this way is completely declarative, hence concise, 
and therefore easy to modify and extend. What is more, if
a new language construct is added, the GF type checker
verifies that the addition is propagated to all components
of the compiler. As the implementation is declarative, 
it is also self-documenting, since a human-readable 
grammar defines the syntax and static
semantics that is actually used in the implementation.


\section{Conclusion}

We managed to compile a large subset of C, and growing it
does not necessarily pose any new kinds of problems. 
Using HOAS and dependent types to describe the abstract
syntax of C works fine, and defining the concrete syntax
of C on top of this using GF linearization machinery is 
already possible, even though more support could be
desired for things like literals and precedences.

The parser generated by GF is not able to parse all
source programs, because some cyclic parse
rules (of the form $C ::= C$) are generated from our grammar. 
Recovery from cyclic rules is ongoing work in GF independently of this
experiment. 
For the time being, the interactive editor is the best way to
construct C programs using our grammar.

The most serious difficulty with using GF as a compiler tool
is how to generate machine code by linearization if this depends on
an evolving symbol table mapping variables to addresses.
Since the compositional linearization model of GF does not
support this, we needed postprocessing to get real JVM code
from the linearization result. The question is this problem can
be solved by some simple and natural new feature of GF.



\bibliographystyle{plain}

\bibliography{gf-bib}


\newpage
\section*{Appendix A: The abstract syntax}

\small
\begin{verbatim}
abstract Imper = {

  cat
    Program ;
    Typ ;
    ListTyp ;
    Fun ListTyp Typ ;
    Body ListTyp ;
    Stm ;
    Exp Typ ;
    Var Typ ;
    ListExp ListTyp ;

  fun
    Empty : Program ;
    Funct : (AS : ListTyp) -> (V : Typ) -> 
              (Body AS) -> (Fun AS V -> Program) -> Program ;
    BodyNil  : Stm -> Body NilTyp ;
    BodyCons : (A : Typ) -> (AS : ListTyp) -> 
                  (Var A -> Body AS) -> Body (ConsTyp A AS) ;

    Decl   : (A : Typ) -> (Var A -> Stm) -> Stm ;
    Assign : (A : Typ) -> Var A -> Exp A -> Stm -> Stm ;
    Return : (A : Typ) -> Exp A -> Stm ;
    While  : Exp TInt -> Stm -> Stm -> Stm ;
    IfElse : Exp TInt -> Stm -> Stm -> Stm -> Stm ;
    Block  : Stm -> Stm -> Stm ;
    End    : Stm ;

    EVar   : (A : Typ) -> Var A -> Exp A ;
    EInt   : Int -> Exp TInt ;
    EFloat : Int -> Int -> Exp TFloat ;
    ELtI   : Exp TInt -> Exp TInt -> Exp TInt ;
    ELtF   : Exp TFloat -> Exp TFloat -> Exp TInt ;
    EApp   : (AS : ListTyp) -> (V : Typ) -> Fun AS V -> ListExp AS -> Exp V ;
    EAddI, EMulI, ESubI : Exp TInt -> Exp TInt -> Exp TInt ;
    EAddF, EMulF, ESubF : Exp TFloat -> Exp TFloat -> Exp TFloat ;

    TInt, TFloat : Typ ;

    NilTyp  : ListTyp ;
    ConsTyp : Typ -> ListTyp -> ListTyp ;

    NilExp  : ListExp NilTyp ;
    ConsExp : (A : Typ) -> (AS : ListTyp) -> 
                 Exp A -> ListExp AS -> ListExp (ConsExp A AS) ;
}
\end{verbatim}
\normalsize
\newpage


\section*{Appendix B: The concrete syntax of C}

\small
\begin{verbatim}
concrete ImperC of Imper = open ResImper in {
  flags lexer=codevars ; unlexer=code ; startcat=Stm ;

  lincat
    Exp = PrecExp ;
    Body = {s,s2 : Str ; size : Size} ;
    ListExp = {s : Str ; size : Size} ;

  lin
    Empty = ss [] ;
    Funct args val body cont = ss (
      val.s ++ cont.$0 ++ paren body.s2 ++ "{" ++ 
      body.s ++ "}" ++ ";" ++ cont.s) ;
    BodyNil stm = stm ** {s2 = [] ; size = Zero} ;
    BodyCons typ _ body = {
      s  = body.s ; 
      s2 = typ.s ++ body.$0 ++ separator "," body.size ++ body.s2 ;
      size = nextSize body.size
      } ;

    Decl  typ cont = continues (typ.s ++ cont.$0) cont ;
    Assign _ x exp = continues (x.s ++ "=" ++ ex exp) ;
    Return _ exp   = statement ("return" ++ ex exp) ;
    While exp loop = continue  ("while" ++ paren (ex exp) ++ loop.s) ;
    IfElse exp t f = continue  ("if" ++ paren (ex exp) ++ t.s ++ "else" ++ f.s) ;
    Block stm      = continue  ("{" ++ stm.s ++ "}") ;
    End            = ss [] ;
 
    EVar  _ x    = constant x.s ;
    EInt    n    = constant n.s ;
    EFloat a b   = constant (a.s ++ "." ++ b.s) ;
    EMulI, EMulF = infixL P2 "*" ;
    EAddI, EAddF = infixL P1 "+" ;
    ESubI, ESubF = infixL P1 "-" ;
    ELtI,  ELtF  = infixN P0 "<" ;
    EApp args val f exps = constant (f.s ++ paren exps.s) ;

    TInt    = ss "int" ;
    TFloat  = ss "float" ;
    NilTyp  = ss [] ;
    ConsTyp = cc2 ;

    NilExp = ss [] ** {size = Zero} ;
    ConsExp _ _ e es = {
      s = ex e ++ separator "," es.size ++ es.s ;
      size = nextSize es.size
      } ;
}
\end{verbatim}
\normalsize
\newpage


\section*{Appendix C: The concrete syntax of JVM}

\small
\begin{verbatim}
concrete ImperJVM of Imper = open ResImper in {

  flags lexer=codevars ; unlexer=code ; startcat=Stm ;

  lincat
    Body  = {s,s2 : Str} ; -- code, storage for locals
    Stm = Instr ;

  lin
    Empty = ss [] ;
    Funct args val body rest = ss (
      ".method" ++ rest.$0 ++ paren args.s ++ val.s ++ ";" ++
      ".limit" ++ "locals" ++ body.s2 ++ ";" ++
      ".limit" ++ "stack" ++ "1000" ++ ";" ++
      body.s ++
      ".end" ++ "method" ++ ";" ++
      rest.s 
      ) ;
    BodyNil stm = stm ;
    BodyCons a as body = instrb a.s [] (body ** {s3 = []});

    Decl  typ cont = instrb typ.s (
      "alloc" ++ typ.s ++ cont.$0
      ) cont ;
    Assign t x exp = instrc (
      exp.s ++ 
      t.s ++ "_store" ++ x.s
      ) ;
    Return t exp   = instr (
      exp.s ++ 
      t.s ++ "_return") ;
    While exp loop = 
      let 
        test = "TEST_" ++ loop.s2 ; 
        end = "END_" ++ loop.s2
      in instrl (
        test ++ ";" ++
        exp.s ++ 
        "ifzero" ++ end ++ ";" ++ 
        loop.s ++
        "goto" ++ test ++ ";" ++ 
        end
        ) ;
    IfElse exp t f = 
      let 
        false = "FALSE_" ++ t.s2 ++ f.s2 ; 
        true  = "TRUE_" ++ t.s2 ++ f.s2
      in instrl (
        exp.s ++ 
        "ifzero" ++ false ++ ";" ++ 
        t.s ++
        "goto" ++ true ++ ";" ++
        false ++ ";" ++
        f.s ++ 
        true
        ) ;
    Block stm      = instrc stm.s ;
    End            = ss [] ** {s2,s3 = []} ;

    EVar  t x  = instr (t.s ++ "_load" ++ x.s) ;
    EInt    n  = instr ("ipush" ++ n.s) ;
    EFloat a b = instr ("fpush" ++ a.s ++ "." ++ b.s) ;
    EAddI      = binop "iadd" ;
    EAddF      = binop "fadd" ;
    ESubI      = binop "isub" ;
    ESubF      = binop "fsub" ;
    EMulI      = binop "imul" ;
    EMulF      = binop "fmul" ;
    ELtI       = binop ("call" ++ "ilt") ;
    ELtF       = binop ("call" ++ "flt") ;
    EApp args val f exps = instr (
      exps.s ++
      "invoke" ++ f.s ++ paren args.s ++ val.s
      ) ;

    TInt   = ss "i" ;
    TFloat = ss "f" ;

    NilTyp = ss [] ;
    ConsTyp = cc2 ;

    NilExp = ss [] ;
    ConsExp _ _ = cc2 ;
}

\end{verbatim}
\normalsize
\newpage

\section*{Appendix D: Auxiliary operations for concrete syntax}

\small
\begin{verbatim}
resource ResImper = {

  -- precedence

  param
    Prec = P0 | P1 | P2 | P3 ;
  oper
    PrecExp   : Type = {s : Prec => Str} ;
    ex        : PrecExp -> Str = \exp -> exp.s ! P0 ;
    constant  : Str -> PrecExp = \c -> {s = \\_ => c} ;
    infixN : Prec -> Str -> PrecExp -> PrecExp -> PrecExp = \p,f,x,y ->
      {s = \\k => mkPrec (x.s ! (nextPrec ! p) ++ f ++ y.s ! (nextPrec ! p)) ! p ! k} ;
    infixL : Prec -> Str -> PrecExp -> PrecExp -> PrecExp = \p,f,x,y ->
      {s = mkPrec (x.s ! p ++ f ++ y.s ! (nextPrec ! p)) ! p} ;

    nextPrec : Prec => Prec = table {P0 => P1 ; P1 => P2 ; _ => P3} ;
    mkPrec : Str -> Prec => Prec => Str = \str -> table {
      P3 => table {                -- use the term of precedence P3...
        _   => str} ;              -- ...always without parentheses
      P2 => table {                -- use the term of precedence P2...
        P3  => paren str ;     -- ...in parentheses if P3 is expected...
        _   => str} ;              -- ...otherwise without parentheses
      P1 => table {
        P3 | P2 => paren str ;
        _  => str} ;
      P0 => table {
        P0  => str ;
        _   => paren str}
      } ;

  -- string operations

    SS  : Type = {s : Str} ;
    ss  : Str -> SS = \s -> {s = s} ;
    cc2 : (_,_ : SS) -> SS = \x,y -> ss (x.s ++ y.s) ;

    paren : Str -> Str = \str -> "(" ++ str ++ ")" ;

    continues : Str -> SS -> SS = \s,t -> ss (s ++ ";" ++ t.s) ; 
    continue  : Str -> SS -> SS = \s,t -> ss (s ++ t.s) ;
    statement : Str -> SS       = \s   -> ss (s ++ ";"); 

  -- taking cases of list size

  param
    Size = Zero | One | More ;
  oper
    nextSize : Size -> Size = \n -> case n of {
      Zero => One ;
      _    => More
      } ;
    separator : Str -> Size -> Str = \t,n -> case n of {
      Zero => [] ;
     _ => t
     } ;

  -- operations for JVM

    Instr  : Type = {s,s2,s3 : Str} ; -- code, variables, labels
    instr  : Str -> Instr = \s -> 
      statement s ** {s2,s3 = []} ;
    instrc : Str -> Instr -> Instr = \s,i -> 
      ss (s ++ ";" ++ i.s) ** {s2 = i.s2 ; s3 = i.s3} ;
    instrl : Str -> Instr -> Instr = \s,i -> 
      ss (s ++ ";" ++ i.s) ** {s2 = i.s2 ; s3 = "L" ++ i.s3} ;
    instrb : Str -> Str -> Instr -> Instr = \v,s,i -> 
      ss (s ++ ";" ++ i.s) ** {s2 = v ++ i.s2 ; s3 = i.s3} ;
    binop  : Str -> SS -> SS -> SS = \op, x, y ->
      ss (x.s ++ y.s ++ op ++ ";") ;
}
\end{verbatim}
\normalsize
\newpage


\section*{Appendix E: Translation to real JVM}

This program is written in Haskell. Most of the changes concern
spacing and could be done line by line; the really substantial
change is due to the need to build a symbol table of variables
stored relative to the frame pointer and look up variable
addresses at each load and store.
\small
\begin{verbatim}
module JVM where

mkJVM :: String -> String
mkJVM = unlines . reverse . fst . foldl trans ([],([],0)) . lines where
  trans (code,(env,v)) s = case words s of
    ".method":f:ns -> ((".method " ++ f ++ concat ns):code,([],0))
    "alloc":t:x:_  -> (code, ((x,v):env, v + size t))
    ".limit":"locals":ns -> chCode (".limit locals " ++ show (length ns - 1))
    t:"_load" :x:_ -> chCode (t ++ "load "  ++ look x) 
    t:"_store":x:_ -> chCode (t ++ "store " ++ look x)
    t:"_return":_  -> chCode (t ++ "return")
    "goto":ns      -> chCode ("goto " ++ concat ns)
    "ifzero":ns    -> chCode ("ifzero " ++ concat ns) 
    _ ->   chCode s
   where
     chCode c = (c:code,(env,v))
     look x = maybe (x ++ show env) show $ lookup x env
     size t = case t of
       "d" -> 2
       _ -> 1
\end{verbatim}
\normalsize
\newpage



\end{document}

\begin{verbatim}
\end{verbatim}

