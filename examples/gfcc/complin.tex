\documentclass[12pt]{article}

\usepackage[latin1]{inputenc}
%\input{psfig.sty}

\setlength{\oddsidemargin}{0mm}
%\setlength{\evensidemargin}{0mm}
\setlength{\evensidemargin}{-2mm}
\setlength{\topmargin}{-16mm}
\setlength{\textheight}{240mm}
\setlength{\textwidth}{158mm}

%\setlength{\parskip}{2mm}
%\setlength{\parindent}{0mm}

\input{macros}

\newcommand{\begit}{\begin{itemize}}
\newcommand{\enit}{\end{itemize}}
\newcommand{\HOAS}{higher-order abstract syntax}
\newcommand{\newone}{} %%{\newpage}
\newcommand{\heading}[1]{\subsection{#1}}
\newcommand{\explanation}[1]{{\small #1}}
\newcommand{\empha}[1]{{\em #1}}
\newcommand{\commentOut}[1]{}
\newcommand{\rarrow}{\; \rightarrow\;}

\newcommand{\nocolor}{} %% {\color[rgb]{0,0,0}}


\title{{\bf Declarative Language Definitions and Code Generation as Linearization}}

\author{Aarne Ranta \\
  Department of Computing Science \\
  Chalmers University of Technology and the University of Gothenburg\\
  {\tt aarne@cs.chalmers.se}}

\date{30 November 2004}

\begin{document}

\maketitle


\subsection*{Abstract}

{\em
This paper presents a compiler for a fragment of the C programming
language, with JVM (Java Virtual Machine) as target language.
The compiler is implemented in a purely declarative way:
its definition consists of an abstract syntax of program
structures and two concrete syntaxes matching the abstract
syntax: one for C and one for JVM. From these grammar components,
the compiler is derived by using the GF (Grammatical Framework)
grammat tool: the front end consists of parsing and semantic
checking in accordance to the C grammar, and the back end consists
of linearization in accordance to the JVM grammar. The tool provides
other functionalities as well, such as decompilation and interactive
editing. 
}

\section{Introduction}

The experiment reported in this paper was prompted by a challenge
posted by Lennart Augustsson to the participants of the workshop
on Dependent Types in Programming held at Dagstuhl in September 2004.
The challenge was to use dependent types to write a compiler from
C to bytecode. This paper does not meet the challenge quite literally,
since our compiler is for a different subset of C than Augustsson's
specification, and since the bytecode that we generate is JVM instead
of his format. But it definitely makes use of dependent types.

Augustsson's  challenge did not specify \textit{how} dependent
types are to be used, and the first of the two points we make in this
paper (and its title) reflects our interpretation:
we use dependent types, in combination with higher-order abstract syntax (HOAS),
to define the grammar of the source language (here, the fragment of C).
The grammar constitutes the single, declarative source from which
the compiler front end is derived, comprising both parser and type
checker.

The second point, code generation by linearization, means that the
back end is likewise implemented by a grammar of the target
language (in this case, a fragment of JVM). This grammar is the
declarative source from which the compiler back end is derived.
In addition, some postprocessing is needed to
make the code conform to Jasmin assembler requirements.

The complete code of the compiler is 300 lines: 250 lines for the grammars,
50 lines for the postprocessor. The code 
is presented in the appendices of this paper.



\section{The Grammatical Framework}

The tool we have used for implementing the compiler is
GF, the \empha{Grammatical Framework} \cite{gf-jfp}. GF 
is similar to a Logical Framework (LF) 
\cite{harper-honsell}
extended with
a notation for defining concrete syntax. GF was originally
designed to help building multilingual
translation systems for natural languages and also
between formal and natural languages. The translation model
implemented by GF is very simple:
\begin{verbatim}
             parsing                       linearization
             ------------>                 ------------>
  Language_1               Abstract Syntax               Language_2
             <------------                 <------------
             linearization                 parsing
\end{verbatim}
An abstract syntax is similar to a \empha{theory}, or a
\empha{signature} in a logical framework. A 
concrete syntax defines, in a declarative way,
a translation of abstract syntax trees (well-formed terms) 
into concrete language structures, and from this definition, one can
derive both linearization and parsing. 

To give an example,
a (somewhat simplified) translator for addition expressions
consists of the abstract syntax rule
\begin{verbatim}
  fun EAdd : (A : Typ) -> Exp A -> Exp A -> Exp A ;
\end{verbatim}
the C concrete syntax rule
\begin{verbatim}
  lin EAdd _ x y = {s = x.s ++ "+" ++ y.s ; prec = 2} ;
\end{verbatim}
and the JVM concrete syntax rule
\begin{verbatim}
  lin EAdd t x y = {s = x.s ++ y.s ++ t.s ++ "_add"} ;
\end{verbatim}
The abstract syntax rule uses a type argument to capture
the fact that addition is polymorphic (which is a simplification,
because we will restrict the rule to numeric types only)
and that both operands have the same type as the value.
The C rule shows that the type information is suppressed,
and that the expression has precedence level 2 (which is a simplification,
since we will also treat associativity).
The JVM rule shows how addition is translated to stack machine
instructions, where the type of the postfixed addition instruction has to
be made explicit. Our compiler, like any GF translation system, will
consist of rules like these.

The number of languages related to one abstract syntax in
a translation system is of course not limited to two. 
Sometimes just one language is involved; 
GF then works much the same way as any grammar 
formalism or parser generator. 
The largest number of languages in an application known to us is 88;
its domain are numeral expressions from 1 to 999,999 \cite{gf-homepage}.

In addition to linearization and parsing, GF supports grammar-based
\empha{multilingual authoring} \cite{khegai}: interactive editing
of abstract syntax trees with immediate feedback as linearized texts,
and the possibility to textual through the parsers. 
 
From the GF point of view, the goal of the compiler experiment
is to investigate if GF is capable of implementing
compilers using the ideas of single-source language definition
and code generation as linearization. The working hypothesis
was that it \textit{is} capable but inconvenient, and that,
working out a complete example, we would find out what 
should be done to extend GF into a compiler construction tool.


\subsection{Advantages and disadvantages}

Due to the way in which it is built, our compiler has
a number of unusual, yet attractive features:
\bequ
The front end is defined by a grammar of C as its single source.

The grammar defines both abstract and concrete syntax, and also
semantic well-formedness (types, variable scopes).

The back end is implemented by means of a grammar of JVM providing
another concrete syntax to the abstract syntax of C.

As a result of the way JVM is defined, only semantically well formed
JVM programs are generated.

The JVM grammar can also be used as a decompiler, which translates
JVM code back into C code.

The language has an interactive editor that also supports incremental
compilation.
\enqu
The problems that we encountered and their causes will be explained in 
the relevant sections of this report. To summarize,
\bequ
The scoping conditions resulting from \HOAS\ are slightly different
from the standard ones of C.

Our JVM syntax is slightly different from the specification, and
hence needs some postprocessing.

Using \HOAS\ to encode all bindings is sometimes cumbersome.
\enqu
The first shortcoming seems to be inevitable with the technique
we use: just like lambda calculus, our C semantics allows
overshadowing of earlier bindings by later ones. 
The second problem is systematically solved by using
an intermediate JVM format, where symbolic variable addresses
are used instead of numeric stack addresses.
The last shortcoming is partly inherent in the problem of binding:
to spell out, in any formal notation,
what happens in complex binding structures \textit{is}
complicated. But it also suggests ways in which GF could be 
tuned to give better support
to compiler construction, which, after all, is not an intended
use of GF as it is now.








\section{The abstract syntax}

An \empha{abstract syntax} in GF consists of \texttt{cat} judgements
\[
\mbox{\texttt{cat}} \; C \; \Gamma
\]
declaring basic types (depending on a context $\Gamma$), 
and \texttt{fun} judgements 
\[
\mbox{\texttt{fun}} \; f \; \mbox{\texttt{:}} \; A
\]
declaring functions $f$ of any type $A$, which can be a basic type or
a function type.
\empha{Syntax trees} are well-formed terms of basic
types, in $\eta$-long normal form.

As for notation, each judgement form is recognized by
its keyword (\texttt{cat}, \texttt{fun}, etc), 
and the same keyword governs all judgements
until the next keyword is encountered.

The abstract syntax that we will present is no doubt closer
to C than to JVM. One reason is that what we are building is a
\textit{C compiler}, and match with the target language is a
secondary consideration. Another, more general reason is that
C is a higher-level language and JVM which means, among
other things, that C makes more semantic distinctions.
In general, the abstract syntax of a translation system
must reflect all semantic distinctions that can be made
in the languages involved, and then it is a good idea to
start with looking at what the most distinctive language
needs.



\subsection{Statements}

Statements in C may involve variables, expressions, and
other statements.
The following \texttt{cat} judgements of GF define the syntactic categories
that are needed to construct statements
\begin{verbatim}
  cat
    Stm ;
    Typ ;
    Exp Typ ;
    Var Typ ;
\end{verbatim}
The type \texttt{Typ} is the type of C's datatypes.
The type of expressions is a dependent type, 
since it has a nonempty context, indicating that \texttt{Exp} takes
a \texttt{Typ} as argument. The rules for \texttt{Exp}
will thus be rules to construct well-typed expressions of
a given type. \texttt{Var}\ is the type of variables,
of a given type, that get bound in C's variable
declarations.

Let us start with the simplest kind of statements:
declarations and assignments. The following \texttt{fun}
rules define their abstract syntax:
\begin{verbatim}
  fun
    Decl   : (A : Typ) -> (Var A -> Stm) -> Stm ;
    Assign : (A : Typ) -> Var A -> Exp A -> Stm -> Stm ;
\end{verbatim}
The \texttt{Decl}\ function captures the rule that
a variable must be declared before it can be used or assigned to:
its second argument is a \empha{continuation}, which is
the sequence of statements that depend on (= may refer to)
the declared variable. 
The \texttt{Assign} function uses dependent types to 
control that a variable is always assigned a value of proper
type. 

We will treat all statements, except
\texttt{return}s, in terms of continuations. A sequence of
statements (which always has the type \texttt{Stm}) thus
always ends in a \texttt{return}, or, abruptly, in
an empty statement, \texttt{End}. Here are rules for some other
statement forms:
\begin{verbatim}
    While  : Exp TInt -> Stm -> Stm -> Stm ;
    IfElse : Exp TInt -> Stm -> Stm -> Stm -> Stm ;
    Block  : Stm -> Stm -> Stm ;
    Return : (A : Typ) -> Exp A -> Stm ;
    End    : Stm ;
\end{verbatim}
Here is an example of a piece of code and its abstract syntax.
\begin{verbatim}
  int x ;      Decl (TNum TInt) (\x -> 
  x = 5 ;        Assign (TNum TInt) x (EInt 5) (
  return x ;       Return (TNum TInt) (EVar (TNum TInt) x)))
\end{verbatim}
The details of expression and type 
syntax will be explained in the next section.

Our binding syntax is more liberal than C's in two ways.
First,
lambda calculus permits overshadowing previous bindings
by new ones, e.g. to write \verb6\x -> (\x -> f x)6.
The corresponding overshadowing of declarations is not
legal in C, within one and the same block.
Secondly, we allow declarations anywhere in a block,
not just in the beginning.
The second deviation would be easy to mend, whereas
the first one is inherent to the method of \HOAS.



\subsection{Types and expressions}

Our fragment of C has two types: integers and floats.
Many operators of C are overloaded so that they can
be used for both of these types, as well as for
some other numeric types---but not for e.g.\ arrays
and structures. We capture this distinction by a notion
reminiscent of \empha{type classes}: we introduce a special
category of numeric types, and a coercion of numeric types
into types in general.
\begin{verbatim}
  cat 
    NumTyp ;
  fun
    TInt, TFloat : NumTyp ;
    TNum : NumTyp -> Typ ;  
\end{verbatim}
Well-typed expressions are built from constants,
from variables, and by means of binary operations.
\begin{verbatim}
    EVar   : (A : Typ) -> Var A -> Exp A ;
    EInt   : Int -> Exp (TNum TInt) ;
    EFloat : Int -> Int -> Exp (TNum TFloat) ;
    ELt    : (n : NumTyp) -> let Ex = Exp (TNum n) in 
               Ex -> Ex -> Exp (TNum TInt) ;
    EAdd, EMul, ESub : (n : NumTyp) -> let Ex = Exp (TNum n) in 
               Ex -> Ex -> Ex ;
\end{verbatim}
Notice that the category \texttt{Var} has no constructors,
but its expressions are only created by
variable bindings in \HOAS. 
Notice also that GF has a built-in type \texttt{Int} of
integer literals, but floats are constructed by hand.

Yet another expression form are function calls. To this
end, we need a notion of (user-defined) functions and
argument lists. The type of functions depends on an
argument type list and a value type. Expression lists
are formed to match type lists.
\begin{verbatim}
  cat
    ListTyp ;
    Fun ListTyp Typ ;
    ListExp ListTyp ;

  fun
    EAppNil : (V : Typ) -> Fun NilTyp V -> Exp V ;
    EApp    : (AS : ListTyp) -> (V : Typ) -> 
                Fun AS V -> ListExp AS -> Exp V ;

    NilTyp  : ListTyp ;
    ConsTyp : Typ -> ListTyp -> ListTyp ;

    OneExp  : (A : Typ) -> Exp A -> ListExp (ConsTyp A NilTyp) ;
    ConsExp : (A : Typ) -> (AS : ListTyp) -> 
                Exp A -> ListExp AS -> ListExp (ConsExp A AS) ;
\end{verbatim}
The separation between zero-element applications and other
applications is a concession to the concrete syntax of C:
it in this way that we can control the use of commas so that
they appear between arguments (\texttt{(x,y,z)}) but not
after the last argument (\texttt{(x,y,z,)}).
The compositionality of linearization (Section~\ref{compositionality} below)
forbids case analysis on the length of the lists.


\subsection{Functions}

On the top level, a program is a sequence of functions.
Each function may refer to functions defined earlier
in the program. The idea to express the binding of
function symbols with \HOAS\ is analogous to the binding
of variables in statements, using a continuation.
As with variables, the principal way to build function symbols is as
bound variables (in addition, there can be some
built-in functions, unlike in the case of variables).
The continuation of can be recursive, which we express by
making the function body into a part of the continuation;
the category \texttt{Rec} is the combination of a function
body and the subsequent function definitions.
\begin{verbatim}
  cat
    Program ;
    Rec ListTyp ;
  fun
    Empty : Program ;
    Funct : (AS : ListTyp) -> (V : Typ) -> 
              (Fun AS V -> Rec AS) -> Program ;
    FunctNil : (V : Typ) -> 
                 Stm -> (Fun NilTyp V -> Program) -> Program ;
\end{verbatim}
For syntactic reasons similar to function application
expressions in the previous section, we have distinguished between
empty and non-empty argument lists.

The tricky problem with function definitions
is that they involve two nested binding constructions:
the outer binding of the function symbol and the inner
binding of the function parameter lists.
For the latter, we could use
vectors of variables, in the same way as vectors of
expressions are used to give arguments to functions.
However, this would lead to the need of cumbersome
projection functions when using the parameters
in the function body. A more elegant solution is
to use \HOAS\ to build function bodies:
\begin{verbatim}
    RecOne  : (A : Typ) -> 
                   (Var A -> Stm) -> Program -> Rec (ConsTyp A NilTyp) ;
    RecCons : (A : Typ) -> (AS : ListTyp) -> 
                  (Var A -> Rec AS) -> Program -> Rec (ConsTyp A AS) ;
\end{verbatim}
The end result is an abstract syntax whose relation
to concrete syntax is somewhat remote. Here is an example of
the code of a function and its abstract syntax:
\begin{verbatim}
                      let int = TNum TInt in
  int fact            Funct (ConsTyp int NilTyp) int (\fact -> 
        (int n) {                                       RecOne int (\n -> 
    int f ;             Decl int (\f -> 
    f = 1 ;             Assign int f (EInt 1) ( 
    while (1 < n) {     While (ELt int (EInt 1) (EVar int n)) (Block (
      f = n * f ;         Assign int f (EMul int (EVar int n) (EVar int f)) (
      n = n - 1 ;         Assign int n (ESub int (EVar int n) (EInt 1)) 
    }                     End)) 
    return f ;          (Return int (EVar int f))) End))) 
  } ;                   Empty)
\end{verbatim}


\subsection{The \texttt{printf} function}

To give a valid type to the C function \texttt{printf}
is one of the things that can be done with
dependent types \cite{cayenne}. We have not defined \texttt{printf}
in its full strength, partly because of the difficulties to compile
it into JVM. But we use special cases of \texttt{printf} as
statements, to be able to print values of different types.
\begin{verbatim}
    Printf : (A : Typ) -> Exp A -> Stm -> Stm ;
\end{verbatim}





\section{The concrete syntax of C}

A concrete syntax, for a given abstract syntax, 
consists of \texttt{lincat} judgements
\[
\mbox{\texttt{lincat}} \; C \; \mbox{\texttt{=}} \; T
\]
defining the \empha{linearization types} $T$ of each category $C$,
and \texttt{lin} judgements
\[
\mbox{\texttt{lin}} \; f \; \mbox{\texttt{=}} \; t
\]
defining the \empha{linearization functions} $t$ of each function $f$
in the abstract syntax. The linearization functions are
checked to be well-typed with respect the \texttt{lincat}
definitions, and the syntax of GF forces them to be \empha{compositional}
in the sense that the linearization of a complex tree is always
a function of the linearizations of the subtrees. Schematically, if
\[
  f \colon A_{1} \rarrow \cdots \rarrow A_{n} \rarrow A 
\]
then
\[
  \sugmap{f} \colon 
    \sugmap{A_{1}} \rarrow \cdots 
    \rarrow \sugmap{A_{n}} \rarrow \sugmap{A} 
\]
and the linearization algorithm is simply
\[
  \sugmap{(f \; a_{1} \; \ldots \; a_{n})} \; = \;
   \sugmap{f} \; \sugmap{a_{1}} \; \ldots \; \sugmap{a_{n}}
\]
using the \sugmap{} notation for both linearization types, 
linearization functions, and linearizations of trees.

\label{compositionality}
Because of compositionality, no case analysis on expressions
is possible in linearization rules. The values of linearization
therefore have to carry information on how they are used in
different situations. Therefore linearization
types are generally record types instead of just the string type.
The simplest record type that is used in GF is
\bece
\verb6{s : Str}6
\ence
If the linearization type of a category is not explicitly
given by a \texttt{lincat} judgement, this type is
used by default. 

With \HOAS, a syntax tree can have variable-bindings in its
constituents. The linearization of such a constituent
is compositionally defined to be the record linearizing the body
extended with fields for each of the variable symbols:
\[
\sugmap{(\lambda x_{0} \rarrow \cdots \rarrow \lambda x_{n} \rarrow b)}
\;=\;
\sugmap{b} *\!* \{\$_{0} = \sugmap{x_{0}} ; \ldots ; \$_{n} = \sugmap{x_{n}}\}
\]
Notice that the variable symbols can
always be found because linearizable trees are in $\eta$-long normal form.
Also notice that we are here using the 
\sugmap{} notation in yet another way, to denote the magic operation
that converts variable symbols into strings.


\subsection{Resource modules}

Resource modules define auxiliary notions that can be
used in concrete syntax. These notions include
\empha{parameter types} defined by \texttt{param}
judgements
\[
\mbox{\texttt{param}} \; P \; \mbox{\texttt{=}} 
  \; C_{1} \; \Gamma_{1} \; \mid \; \cdots \; \mid \; 
  \; C_{n} \; \Gamma_{n}
\]
and \empha{operations} defined by
\texttt{oper} judgements
\[
\mbox{\texttt{oper}} \; f \; \mbox{\texttt{:}} \; T \; \mbox{\texttt{=}} \; t
\]
These judgements are
similar to datatype and function definitions
in functional programming, with the restriction
that parameter types must be finite and operations
may not be recursive. It is due to these restrictions that
we can always derive a parsing algorithm from a set of
linearization rules.

In addition to types defined in \texttt{param} judgements,
initial segments of natural numbers, \texttt{Ints n},
can be used as parameter types. This is the most important parameter
type we use in the syntax of C, to represent precedence.

The following string operations are useful in almost
all grammars. They are actually included in a GF \texttt{Prelude},
but are here defined from scratch to make the code shown in
the Appendices complete.
\begin{verbatim}
  oper
    SS    : Type = {s : Str} ;
    ss    : Str -> SS = \s -> {s = s} ;
    cc2   : (_,_ : SS) -> SS = \x,y -> ss (x.s ++ y.s) ;
    paren : Str -> Str = \str -> "(" ++ str ++ ")" ;
\end{verbatim}



\subsection{Precedence}

We want to be able to recognize and generate one and the same expression with
or without parentheses, depending on whether its precedence level
is lower or higher than expected. For instance, a sum used as
an operand of multiplication must be in parentheses. We
capture this by defining a parameter type of
precedence levels. Five levels are enough for the present
fragment of C, so we use the enumeration type of 
integers from 0 to 4 to define the \empha{inherent precedence level}
of an expression
\begin{verbatim} 
  oper 
    Prec    : PType = Predef.Ints 4 ;
    PrecExp : Type = {s : Str ; p : Prec} ;
\end{verbatim}
in a resource module (see Appendix D), and
\begin{verbatim}
  lincat Exp = PrecExp ;
\end{verbatim}
in the concrete syntax of C itself. 

To build an expression that has a certain inherent precedence level,
we use the operation
\begin{verbatim}
    mkPrec : Prec -> Str -> PrecExp = \p,s -> {s = s ; p = p} ;
\end{verbatim}
To use an expression of a given inherent level at some expected level,
we define a function that says that, if the inherent level is lower
than the expected level, parentheses are required. 
\begin{verbatim}
    usePrec : PrecExp -> Prec -> Str = \x,p ->
      ifThenStr
        (less x.p p) 
        (paren x.s)
        x.s ;
\end{verbatim}
(The code shown in Appendix D is at the moment more cumbersome,
due to holes in the support for integer arithmetic in GF.)

With the help of \texttt{mkPrec} and \texttt{usePrec},
we can now define the main high-level operations that are
used in the concrete syntax itself---constants (highest level),
non-associative infixes, and left associative infixes:
\begin{verbatim}
    constant : Str -> PrecExp = mkPrec 4 ;

    infixN : Prec -> Str -> (_,_ : PrecExp) -> PrecExp = \p,f,x,y ->
      mkPrec p (usePrec x (nextPrec p) ++ f ++ usePrec y (nextPrec p)) ;
    infixL : Prec -> Str -> (_,_ : PrecExp) -> PrecExp = \p,f,x,y ->
      mkPrec p (usePrec x p ++ f ++ usePrec y (nextPrec p)) ;
\end{verbatim}
(The code in Appendix D adds to this an associativity parameter, 
which is redundant in GF, but which we use to instruct the Happy 
parser generator.)


\subsection{Expressions}

With the machinery introduced, the linearization rules of expressions
are simple and concise:
\begin{verbatim}
    EVar  _ x  = constant x.s ;
    EInt    n  = constant n.s ;
    EFloat a b = constant (a.s ++ "." ++ b.s) ;
    EMul _     = infixL 3 "*" ;
    EAdd _     = infixL 2 "+" ;
    ESub _     = infixL 2 "-" ;
    ELt _      = infixN 1 "<" ;

    EAppNil val f = constant (f.s ++ paren []) ;
    EApp args val f exps = constant (f.s ++ paren exps.s) ;
\end{verbatim}


\subsection{Types}

Types are expressed in two different ways: 
in declarations, we have \texttt{int} and  \texttt{float}, but
as formatting arguments to \texttt{printf}, we have
\verb6"%d"6 and \verb6"%f"6, with the quotes belonging to the
names. The simplest solution in GF is to linearize types
to records with two string fields.
\begin{verbatim}
  lincat
    Typ, NumTyp = {s,s2 : Str} ;
  lin
    TInt    = {s = "int" ; s2 = "\"%d\""} ; 
    TFloat  = {s = "float" ; s2 = "\"%f\""} ;
\end{verbatim}


\subsection{Statements}

Statements in C have
the simplest linearization type, \verb6{s : Str}6.
We use a handful of auxiliary operations to regulate
the use of semicolons on a high level.
\begin{verbatim}
  oper
    continues : Str -> SS -> SS = \s,t -> ss (s ++ ";" ++ t.s) ; 
    continue  : Str -> SS -> SS = \s,t -> ss (s ++ t.s) ;
    statement : Str -> SS       = \s   -> ss (s ++ ";"); 
\end{verbatim}
As for declarations, which bind variables, we notice the
projection \verb6.$06 to refer to the bound variable.
Also notice the use of the \texttt{s2} field of the type
in \texttt{printf}. 
\begin{verbatim}
  lin
    Decl  typ cont = continues (typ.s ++ cont.$0) cont ;
    Assign _ x exp = continues (x.s ++ "=" ++ exp.s) ;
    While exp loop = continue  ("while" ++ paren exp.s ++ loop.s) ;
    IfElse exp t f = continue  ("if" ++ paren exp.s ++ t.s ++ "else" ++ f.s) ;
    Block stm      = continue  ("{" ++ stm.s ++ "}") ;
    Printf t e     = continues ("printf" ++ paren (t.s2 ++ "," ++ e.s)) ;
    Return _ exp   = statement ("return" ++ exp.s) ;
    Returnv        = statement "return" ;
    End            = ss [] ;
\end{verbatim}


\subsection{Functions and programs}

The category \texttt{Rec} of recursive function bodies with continuations
has three components: the function body itself, the parameter list, and
the program that follows. We express this by a linearization type that
contains three strings:
\begin{verbatim}
  lincat Rec = {s,s2,s3 : Str} ;
\end{verbatim}
The body construction rules accumulate the parameter list
independently of the two other components:
\begin{verbatim}
  lin
    RecOne typ stm prg = stm ** {
      s2 = typ.s ++ stm.$0 ;
      s3 = prg.s
      } ;
    RecCons typ _ body prg = {
      s  = body.s ; 
      s2 = typ.s ++ body.$0 ++ "," ++ body.s2 ;
      s3 = prg.s
      } ;
\end{verbatim}
The top-level program construction rules rearrange the three
components into a linear structure:
\begin{verbatim}
    FunctNil val stm cont = ss (
      val.s ++ cont.$0 ++ paren [] ++ "{" ++ 
      stm.s ++ "}" ++ ";" ++ cont.s) ;
    Funct args val rec = ss (
      val.s ++ rec.$0 ++ paren rec.s2 ++ "{" ++ 
      rec.s ++ "}" ++ ";" ++ rec.s3) ;
\end{verbatim}


%%\subsection{Lexing and unlexing}



\section{The concrete syntax of JVM}

JVM syntax is, linguistically, more straightforward than
the syntax of C, and could even be defined by a regular
expression. However, the JVM syntax that our compiler
generates does not comprise full JVM, but only the fragment
that corresponds to well-formed C programs.

The JVM syntax we use is a symbolic variant of the Jasmin assembler
\cite{jasmin}.
The main deviation from Jasmin are
variable addresses, as described in Section~\ref{postproc}.
The other deviations have to do with spacing: the normal
unlexer of GF puts spaces between constituents, whereas
in JVM, type names are integral parts of instruction names.
We indicate gluing uniformly by generating an underscore
on the side from which the adjacent element is glued. Thus
e.g.\ \verb6i _load6 becomes \verb6iload6.


\subsection{Symbolic JVM}
\label{postproc}

What makes the translation from our abstract syntax to JVM 
tricky is that variables must be replaced by
numeric addresses (relative to the frame pointer).
Code generation must therefore maintain a symbol table that permits
the lookup of variable addresses. As shown in the code
in Appendix C, we do not treat symbol tables
in linearization, but instead generated code in
\empha{Symbolic JVM}---that is, JVM with symbolic addresses.
Therefore we need a postprocessor that resolves the symbolic addresses,
shown in Appendix E.

To make the postprocessor straightforward,
Symbolic JVM has special \texttt{alloc} instructions,
which are not present in real JVM.
Our compiler generates \texttt{alloc} instructions from
variable declarations.
The postprocessor comments out the \texttt{alloc} instructions, but we
found it a good idea not to erase them completely, since they make the
code more readable.

The following example shows how the three representations (C, Symbolic JVM, JVM) 
look like for a piece of code.
\begin{verbatim}
  int x ;   alloc i x      ; x gets address 0
  int y ;   alloc i y      ; y gets address 1
  x = 5 ;   ldc 5          ldc 5
            i _store x     istore 0
  y = x ;   i _load x      iload 0
            i _store y     istore 1
\end{verbatim}


\subsection{Labels and jumps}

A problem related to variable addresses 
is the generation of fresh labels for
jumps. We solve this in linearization
by maintaining a growing label suffix
as a field of the linearization of statements into
instructions. The problem remains that statements on the
same nesting level, e.g.\ the two branches
of an \texttt{if-else} statement can use the same
labels. Making them unique must be
added to the post-processing pass. This is
always possible, because labels are nested in a
disciplined way, and jumps can never go to remote labels.

As it turned out laborious to thread the label counter
to expressions, we decided to compile comparison 
expressions (\verb6x < y6) into function calls, and provide the functions in
a run-time library. This will no more work for the 
conjunction (\verb6x && y6)
and disjunction (\verb6x || y6), if we want to keep their semantics
lazy, since function calls are strict in their arguments.





\subsection{How to restore code generation by linearization}

Since postprocessing is needed, we have not quite achieved
the goal of code generation as linearization---if
linearization is understood in the
sense of GF. In GF, linearization can only depend 
on parameters from finite parameter sets. Since the size of
a symbol table can grow indefinitely, it is not 
possible to encode linearization with updates to and 
lookups from a symbol table, as is usual in code generation. 

One attempt we made to achieve JVM linearization with
numeric addresses was to alpha-convert abstract syntax syntax trees
so that variables get indexed with integers that indicate their 
depths in the tree. This hack works in the present fragment of C
because all variables need the same amount of memory (one word), 
but would break down if we added double-precision floats. Therefore
we have used the less pure (from the point of view of
code generation as linearization) method of
symbolic addresses.

It would certainly be possible to generate variable addresses
directly in the syntax trees by using dependent types; but this 
would clutter the abstract
syntax in a way that is hard to motivate when we are in
the business of describing the syntax of C. The abstract syntax would
have to, so to say, anticipate all demands of the compiler's
target languages. 


\subsection{Problems with the JVM bytecode verifier}

An inherent restriction for linearization in GF is compositionality.
This prevents optimizations during linearization
by clever instruction selection, elimination of superfluous
labels and jumps, etc. One such optimization, the removal
of unreachable code (i.e.\ code after a \texttt{return} instruction)
is actually required by the JVM byte code verifier.
The solution is, again, to perform this optimization in postprocessing.
What we currently do, however, is to be careful and write
C programs so that they always end with a return statement in the
outermost block.

Another problem related to \texttt{return} instructions is that
both C and JVM programs have a designated \texttt{main} function.
This function must have a certain type, which is different in C and
JVM. In C, \texttt{main} returns an integer encoding what
errors may have happend during execution. The JVM 
\texttt{main}, on the other hand, returns a \texttt{void}, i.e.\
no value at all. A \texttt{main} program returning an
integer therefore provokes a JVM bytecode verifier error.
The postprocessor could take care of this; but currently
we just write programs with void \texttt{return}s in the 
\texttt{main} functions.

The parameter list of \texttt{main} is also different in C (empty list)
and JVM (a string array \texttt{args}). We handle this problem
with an \empha{ad hoc} postprocessor rule.

Every function prelude in JVM must indicate the maximum space for
local variables, and the maximum evaluation stack space (within
the function's own stack frame). The locals limit is computed in
linearization by maintaining a counter field. The stack limit
is blindly set to 1000; it would be possible to set an
accurate limit in the postprocessing phase.


\section{Translation as linearization vs.\ transfer}

Many of the problems we have encountered in code generation by
linearization are familiar from
translation systems for natural languages. For instance, to translate
the English pronoun \eex{you} to German, you have to choose
between \eex{du, ihr, Sie}; for Italian, there are four
variants, and so on. To deal with this by linearization,
all semantic distinctions made in any of the involved languages 
have to be present in the common abstract syntax. The usual solution to 
this problem is not a universal abstract syntax, but
\empha{transfer}: translation does not just linearize
the same syntax trees to another language, but uses
a noncompositional function that translates
trees of one language into trees of another.

Using transfer in the
back end is precisely what traditional compilers do.
The transfer function in our case would be a noncompositional
function from the abstract syntax of C to a different abstract
syntax of JVM. The abstract syntax notation of GF permits
definitions of functions, and the GF interpreter can be used
for evaluating terms into normal form. Thus one could write
the code generator just like in any functional language:
by sending in an environment and a syntax tree, and
returning a new environment with an instruction list:
\begin{verbatim}
  fun 
    transStm : Env -> Stm -> EnvInstr ;
  def
    transStm env (Decl typ cont) = ... 
    transStm env (While (ELt a b) stm cont) = ... 
    transStm env (While exp stm cont) = ... 
\end{verbatim}
This would be cumbersome in practice, because
GF does not have programming-language facilities 
like built-in lists and tuples, or monads. Moreover,
the compiler could no longer be inverted into a decompiler, 
in the way true linearization can be inverted into a parser.



\section{Parser generation}
\label{bnfc}

The whole GF part of the compiler (parser, type checker, Symbolic JVM
generator) can be run in the GF interpreter. 
The weakest point of the resulting compiler, by current
standards, is the parser. GF is a powerful grammar formalism, which
needs a very general parser, taking care of ambiguities and other
problems that are typical of natural languages but should be
overcome in programming languages by design. The parser is moreover run
in an interpreter that takes the grammar (in a suitably compiled form)
as an argument. 

Fortunately, it is easy to replace the generic, interpreting GF parser
by a compiled LR(1) parser. GF supports the translation of a concrete
syntax into the \empha{Labelled BNF} (LBNF) format, %% \cite{lbnf}, 
which in turn can be translated to parser generator code
(Happy, Bison, or JavaCUP), by the BNF Converter \cite{bnfc}.
The parser we are therefore using in the compiler is a Haskell
program generated by Happy \cite{happy}. 

We regard parser generation
as a first step towards developing GF into a 
production-quality compiler compiler. The efficiency of the parser
is not the only relevant thing. Another advantage of an LR(1)
parser generator is that it performs an analysis on the grammar
finding conflicts, and provides a debugger. It may be
difficult for a human to predict how a context-free grammar
performs at parsing; it is much more difficult to do this for
a grammar written in the abstract way that GF permits (cf.\ the
example in Appendix B).

The current version of the C grammar is ambiguous. GF's own
parser returns all alternatives, whereas the parser generated by
Happy rules out some of them by its normal conflict handling
policy. This means, in practice, that extra brackets are
sometimes needed to group staments together.


\subsection{Another notation for \HOAS}

Describing variable bindings with \HOAS\ is sometimes considered
unintuitive. Let us consider the declaration rule of C (without
type dependencies for simplicity):
\begin{verbatim}
  fun Decl : Typ -> (Var -> Stm) -> Stm ;
  lin Decl typ stm = {s = typ.s ++ stm.$0 ++ ";" ++stm.s} ;
\end{verbatim}
Compare this with a corresponding LBNF rule (also using a continuation):
\begin{verbatim}
  Decl. Stm ::= Typ Ident ";" Stm ;
\end{verbatim}
To explain bindings attached to this rule, one can say, in natural language, 
that the identifier gets bound in the statement that follows.
This means that syntax trees formed by this rule do not have 
the form \verb6(Decl typ x stm)6, but the form \verb6(Decl typ (\x -> stm))6.

One way to formalize the informal binding rules stated beside
BNF rules is to use \empha{profiles}: data structures describing
the way in which the logical arguments of the syntax tree are
represented by the linearized form. The declaration rule can be
written using a profile notation as follows:
\begin{verbatim}
  Decl [1,(2)3]. Stm ::= Typ Ident ";" Stm ;
\end{verbatim}
When compiling GF grammars into LBNF, we were forced to enrich
LBNF by a (more general) profile notation
(cf.\ \cite{gf-jfp}, Section 3.3). This suggested at the same
time that profiles could provide a user-fiendly notation for
\HOAS\ avoiding the explicit use of lambda calculus.



\section{Using the compiler}

Our compiler is invoked, of course, by the command \texttt{gfcc}.
It produces a JVM \texttt{.class} file, by running the
Jasmin bytecode assembler \cite{jasmin} on a Jasmin (\texttt{.j})
file:
\begin{verbatim}
  % gfcc factorial.c
  > > wrote file factorial.j
  Generated: factorial.class
\end{verbatim}
The Jasmin code is produced by a postprocessor, written in Haskell
(Appendix E), from the Symbolic JVM format that is produced by
linearization. The reasons why actual Jasmin is not generated
by linearization are explained in Section~\ref{postproc} above.

In addition to the batch compiler, GF provides an interactive 
syntax editor, in which C programs can be constructed by
stepwise refinements, local changes, etc.\ \cite{khegai}. The user of the
editor can work simultaneously on all languages involved.
In our case, this means that changes can be done both to
the C code and to the JVM code, and they are automatically
carried over from one language to the other.
\commentOut{
A screen dump of the editor is shown in Fig~\ref{demo}.

\begin{figure}
\centerline{\psfig{figure=demo2.epsi}} \caption{
GF editor session where an integer
expression is expected to be given. The left window shows the
abstract syntax tree, and the right window the evolving C and
JVM code. The editor focus is shadowed, and the refinement alternatives
are shown in a pop-up window.
}
\label{demo}
\end{figure}
}


\section{Related work}

The theoretical ideas behind our compiler experiment
are familiar from various sources.
Single-source language and compiler definitions
can be built using attribute grammars \cite{knuth-attr}.
The use of
dependent types in combination with higher-order abstract syntax
has been studied in various logical frameworks 
\cite{harper-honsell,magnusson-nordstr,twelf}.
The addition of linearization rules to type-theoretical 
abstract syntax is studied in \cite{semBNF}, which also 
compares the method with attribute grammars.

The idea of using a common abstract syntax for different 
languages was clearly exposed by Landin \cite{landin}. The view of
code generation as linearization is a central aspect of
the classic compiler textbook by Aho, Sethi, and Ullman
\cite{aho-ullman}. 
The use of one and the same grammar both for parsing and linearization
is a guiding principle of unification-based linguistic grammar 
formalisms \cite{pereira-shieber}. Interactive editors derived from
grammars have been developed in various programming and proof
assistants \cite{teitelbaum,metal,magnusson-nordstr}.

Even though the different ideas are well-known, 
we have not seen them used together to construct a complete
compiler. In our view, putting these ideas together is
an attractive approach to compiling, since a compiler written
in this way is completely declarative, hence concise, 
and therefore easy to modify and extend. What is more, if
a new language construct is added, the GF type checker
verifies that the addition is propagated to all components
of the compiler. As the implementation is declarative, 
it is also self-documenting, since a human-readable 
grammar defines the syntax and static
semantics that is actually used in the implementation.


\section{Conclusion}

The \texttt{gfcc} compiler translates a representative 
fragment of C to JVM, and growing the fragment
does not necessarily pose any new kinds of problems. 
Using \HOAS\ and dependent types to describe the abstract
syntax of C works fine, and defining the concrete syntax
of C on top of this using GF linearization machinery is 
possible. To build a parser that is more efficient than
GF's generic one, GF offers code generation for standard 
parser tools.

One result of the experiment is the beginning of a
library for dealing with typical programming language structures
such as precedence. This library is exploited in the parser
generator, which maps certain parameters used into GF grammars
into precedence directives in labelled BNF grammars.

The most serious difficulty with JVM code generation by linearization
is to maintain a symbol table mapping variables to addresses.
The solution we have chosen is to generate Symbolic JVM, that is,
JVM with symbolic addresses, and translate the symbolic addresses to
(relative) memory locations by a postprocessor. 

Since the postprocessor works uniformly for the whole Symbolic JVM,
building a new compiler to generate JVM should now be 
possible by just writing GF grammars. The most immediate
idea for developing GF as a compiler tool is to define
a similar symbolic format for an intermediate language,
which uses three-operand code and virtual registers.



\bibliographystyle{plain}

\bibliography{gf-bib}


\newpage
\subsection*{Appendix A: The abstract syntax}

\small
\begin{verbatim}
abstract Imper = PredefAbs ** {
  cat
    Program ;
    Rec ListTyp ;
    Typ ;
    NumTyp ;
    ListTyp ;
    Fun ListTyp Typ ;
    Body ListTyp ;
    Stm ;
    Exp Typ ;
    Var Typ ;
    ListExp ListTyp ;

  fun
    Empty : Program ;
    Funct : (AS : ListTyp) -> (V : Typ) -> (Fun AS V -> Rec AS) -> Program ;
    FunctNil : (V : Typ) -> Stm -> (Fun NilTyp V -> Program) -> Program ;
    RecOne  : (A : Typ) -> (Var A -> Stm) -> Program -> Rec (ConsTyp A NilTyp) ;
    RecCons : (A : Typ) -> (AS : ListTyp) -> 
                  (Var A -> Rec AS) -> Program -> Rec (ConsTyp A AS) ;

    Decl    : (A : Typ) -> (Var A -> Stm) -> Stm ;
    Assign  : (A : Typ) -> Var A -> Exp A -> Stm -> Stm ;
    While   : Exp (TNum TInt) -> Stm -> Stm -> Stm ;
    IfElse  : Exp (TNum TInt) -> Stm -> Stm -> Stm -> Stm ;
    Block   : Stm -> Stm -> Stm ;
    Printf  : (A : Typ) -> Exp A -> Stm -> Stm ;
    Return  : (A : Typ) -> Exp A -> Stm ;
    Returnv : Stm ;
    End     : Stm ;

    EVar   : (A : Typ) -> Var A -> Exp A ;
    EInt   : Int -> Exp (TNum TInt) ;
    EFloat : Int -> Int -> Exp (TNum TFloat) ;
    ELt    : (n : NumTyp) -> let Ex = Exp (TNum n) in Ex -> Ex -> Exp (TNum TInt) ;
    EAdd, EMul, ESub : (n : NumTyp) -> let Ex = Exp (TNum n) in Ex -> Ex -> Ex ;
    EAppNil : (V : Typ) -> Fun NilTyp V -> Exp V ;
    EApp    : (AS : ListTyp) -> (V : Typ) -> Fun AS V -> ListExp AS -> Exp V ;

    TNum   : NumTyp -> Typ ;  
    TInt, TFloat : NumTyp ;
    NilTyp  : ListTyp ;
    ConsTyp : Typ -> ListTyp -> ListTyp ;
    OneExp  : (A : Typ) -> Exp A -> ListExp (ConsTyp A NilTyp) ;
    ConsExp : (A : Typ) -> (AS : ListTyp) -> 
                 Exp A -> ListExp AS -> ListExp (ConsTyp A AS) ;
}
\end{verbatim}
\normalsize
\newpage


\subsection*{Appendix B: The concrete syntax of C}

\small
\begin{verbatim}
concrete ImperC of Imper = open ResImper in {
  flags lexer=codevars ; unlexer=code ; startcat=Program ;

  lincat
    Exp = PrecExp ;
    Typ, NumTyp = {s,s2 : Str} ;
    Rec = {s,s2,s3 : Str} ;
  lin
    Empty = ss [] ;
    FunctNil val stm cont = ss (
      val.s ++ cont.$0 ++ paren [] ++ "{" ++ stm.s ++ "}" ++ ";" ++ cont.s) ;
    Funct args val rec = ss (
      val.s ++ rec.$0 ++ paren rec.s2 ++ "{" ++ rec.s ++ "}" ++ ";" ++ rec.s3) ;
    RecOne typ stm prg = stm ** {
      s2 = typ.s ++ stm.$0 ;
      s3 = prg.s
      } ;
    RecCons typ _ body prg = {
      s  = body.s ; 
      s2 = typ.s ++ body.$0 ++ "," ++ body.s2 ;
      s3 = prg.s
      } ;

    Decl  typ cont = continues (typ.s ++ cont.$0) cont ;
    Assign _ x exp = continues (x.s ++ "=" ++ exp.s) ;
    While exp loop = continue  ("while" ++ paren exp.s ++ loop.s) ;
    IfElse exp t f = continue  ("if" ++ paren exp.s ++ t.s ++ "else" ++ f.s) ;
    Block stm      = continue  ("{" ++ stm.s ++ "}") ;
    Printf t e     = continues ("printf" ++ paren (t.s2 ++ "," ++ e.s)) ;
    Return _ exp   = statement ("return" ++ exp.s) ;
    Returnv        = statement "return" ;
    End            = ss [] ;
 
    EVar  _ x  = constant x.s ;
    EInt    n  = constant n.s ;
    EFloat a b = constant (a.s ++ "." ++ b.s) ;
    EMul _     = infixL 3 "*" ;
    EAdd _     = infixL 2 "+" ;
    ESub _     = infixL 2 "-" ;
    ELt _      = infixN 1 "<" ;
    EAppNil val f = constant (f.s ++ paren []) ;
    EApp args val f exps = constant (f.s ++ paren exps.s) ;

    TNum t  = t ;     
    TInt = {s = "int" ; s2 = "\"%d\""} ; TFloat  = {s = "float" ; s2 = "\"%f\""} ;
    NilTyp  = ss [] ; ConsTyp = cc2 ;
    OneExp _ e = e ; ConsExp _ _ e es = ss (e.s ++ "," ++ es.s) ;
}
\end{verbatim}
\normalsize
\newpage


\subsection*{Appendix C: The concrete syntax of JVM}

\small
\begin{verbatim}
concrete ImperJVM of Imper = open ResImper in {
  flags lexer=codevars ; unlexer=code ; startcat=Program ;

  lincat
    Rec = {s,s2,s3 : Str} ; -- code, storage for locals, continuation
    Stm = Instr ;

  lin
    Empty = ss [] ;
    FunctNil val stm cont = ss (
      ".method" ++ "public" ++ "static" ++ cont.$0 ++ paren [] ++ val.s ++ ";" ++
      ".limit" ++ "locals" ++ stm.s2 ++ ";" ++
      ".limit" ++ "stack" ++ "1000" ++ ";" ++
      stm.s ++
      ".end" ++ "method" ++ ";" ++ ";" ++
      cont.s 
      ) ;
    Funct args val rec = ss (
      ".method" ++ "public" ++ "static" ++ rec.$0 ++ paren args.s ++ val.s ++ ";" ++
      ".limit"  ++ "locals" ++ rec.s2 ++ ";" ++
      ".limit"  ++ "stack"  ++ "1000" ++ ";" ++
      rec.s ++
      ".end" ++ "method" ++ ";" ++ ";" ++
      rec.s3 
      ) ;

    RecOne typ stm prg = instrb typ.s (
      ["alloc"] ++ typ.s ++ stm.$0 ++ stm.s2) {s = stm.s ; s2 = stm.s2 ; s3 = prg.s};

    RecCons typ _ body prg = instrb typ.s (
      ["alloc"] ++ typ.s ++ body.$0 ++ body.s2) 
         {s = body.s ; s2 = body.s2 ; s3 = prg.s};

    Decl  typ cont = instrb typ.s (
      ["alloc"] ++ typ.s ++ cont.$0
      ) cont ;
    Assign t x exp = instrc (exp.s ++ t.s ++ "_store" ++ x.s) ;
    While exp loop = 
      let 
        test = "TEST_" ++ loop.s2 ; 
        end = "END_" ++ loop.s2
      in instrl (
        "label" ++ test ++ ";" ++
        exp.s ++ 
        "ifeq" ++ end ++ ";" ++ 
        loop.s ++
        "goto" ++ test ++ ";" ++ 
        "label" ++ end
        ) ;
    IfElse exp t f = 
      let 
        false = "FALSE_" ++ t.s2 ++ f.s2 ; 
        true  = "TRUE_" ++ t.s2 ++ f.s2
      in instrl (
        exp.s ++ 
        "ifeq" ++ false ++ ";" ++ 
        t.s ++
        "goto" ++ true ++ ";" ++
        "label" ++ false ++ ";" ++
        f.s ++ 
        "label" ++ true
        ) ;
    Block stm      = instrc stm.s ;
    Printf t e     = instrc (e.s ++ "invokestatic" ++ t.s ++ "runtime/printf" ++ paren (t.s) ++ "v") ;
    Return t exp   = instr (exp.s ++ t.s ++ "_return") ;
    Returnv        = instr "return" ;
    End            = ss [] ** {s2,s3 = []} ;

    EVar  t x  = instr (t.s ++ "_load" ++ x.s) ;
    EInt    n  = instr ("ldc" ++ n.s) ;
    EFloat a b = instr ("ldc" ++ a.s ++ "." ++ b.s) ;
    EAdd       = binopt "_add" ;
    ESub       = binopt "_sub" ;
    EMul       = binopt "_mul" ;
    ELt t  = binop ("invokestatic" ++ t.s ++ "runtime/lt" ++ paren (t.s ++ t.s) ++ "i") ;
    EAppNil val f = instr (
      "invokestatic" ++ f.s ++ paren [] ++ val.s
      ) ;
    EApp args val f exps = instr (
      exps.s ++
      "invokestatic" ++ f.s ++ paren args.s ++ val.s
      ) ;

    TNum t = t ;
    TInt   = ss "i" ;
    TFloat = ss "f" ;
    NilTyp = ss [] ;
    ConsTyp = cc2 ;
    OneExp _ e = e ;
    ConsExp _ _ = cc2 ;
}
\end{verbatim}
\normalsize
\newpage

\subsection*{Appendix D: Auxiliary operations for concrete syntax}

\small
\begin{verbatim}
resource ResImper = open Predef in {

  -- precedence

  param PAssoc = PN | PL | PR ;

  oper 
    Prec    : PType = Predef.Ints 4 ;
    PrecExp : Type = {s : Str ; p : Prec ; a : PAssoc} ;

    mkPrec : Prec -> PAssoc -> Str -> PrecExp = \p,a,f -> 
      {s = f ; p = p ; a = a} ;

    usePrec : PrecExp -> Prec -> Str = \x,p ->
      case <<x.p,p> : Prec * Prec> of {
        <3,4> | <2,3> | <2,4> => paren x.s ;
        <1,1> | <1,0> | <0,0> => x.s ;
        <1,_> | <0,_>         => paren x.s ;
        _ => x.s
        } ;

    constant : Str -> PrecExp = mkPrec 4 PN ;

    infixN : Prec -> Str -> (_,_ : PrecExp) -> PrecExp = \p,f,x,y ->
      mkPrec p PN (usePrec x (nextPrec p) ++ f ++ usePrec y (nextPrec p)) ;
    infixL : Prec -> Str -> (_,_ : PrecExp) -> PrecExp = \p,f,x,y ->
      mkPrec p PL (usePrec x p ++ f ++ usePrec y (nextPrec p)) ;
    infixR : Prec -> Str -> (_,_ : PrecExp) -> PrecExp = \p,f,x,y ->
      mkPrec p PR (usePrec x (nextPrec p) ++ f ++ usePrec y p) ;

    nextPrec : Prec -> Prec = \p -> case <p : Prec> of {
      4 => 4 ; 
      n => Predef.plus n 1
      } ;

  -- string operations

    SS    : Type = {s : Str} ;
    ss    : Str -> SS = \s -> {s = s} ;
    cc2   : (_,_ : SS) -> SS = \x,y -> ss (x.s ++ y.s) ;
    paren : Str -> Str = \str -> "(" ++ str ++ ")" ;

    continues : Str -> SS -> SS = \s,t -> ss (s ++ ";" ++ t.s) ; 
    continue  : Str -> SS -> SS = \s,t -> ss (s ++ t.s) ;
    statement : Str -> SS       = \s   -> ss (s ++ ";"); 

  -- operations for JVM

    Instr  : Type = {s,s2,s3 : Str} ; -- code, variables, labels
    instr  : Str -> Instr = \s -> 
      statement s ** {s2,s3 = []} ;
    instrc : Str -> Instr -> Instr = \s,i -> 
      ss (s ++ ";" ++ i.s) ** {s2 = i.s2 ; s3 = i.s3} ;
    instrl : Str -> Instr -> Instr = \s,i -> 
      ss (s ++ ";" ++ i.s) ** {s2 = i.s2 ; s3 = "L" ++ i.s3} ;
    instrb : Str -> Str -> Instr -> Instr = \v,s,i -> 
      ss (s ++ ";" ++ i.s) ** {s2 = v ++ i.s2 ; s3 = i.s3} ;
    binop  : Str -> SS -> SS -> SS = \op, x, y ->
      ss (x.s ++ y.s ++ op ++ ";") ;
    binopt : Str -> SS -> SS -> SS -> SS = \op, t ->
      binop (t.s ++ op) ;
}
\end{verbatim}
\normalsize
\newpage


\subsection*{Appendix E: Translation of Symbolic JVM to Jasmin}

\small
\begin{verbatim}
module Main where
import Char
import System

main :: IO ()
main = do
  jvm:src:_ <- getArgs
  s <- readFile jvm
  let cls = takeWhile (/='.') src
  let obj = cls ++ ".j"
  writeFile  obj $ boilerplate cls
  appendFile obj $ mkJVM cls s
  putStrLn $ "wrote file " ++ obj

mkJVM :: String -> String -> String
mkJVM cls = unlines . reverse . fst . foldl trans ([],([],0)) . lines where
  trans (code,(env,v)) s = case words s of
    ".method":p:s:f:ns
        | f == "main" -> (".method public static main([Ljava/lang/String;)V":code,([],1))
        | otherwise  -> (unwords [".method",p,s, f ++ typesig ns] : code,([],0))
    "alloc":t:x:_  -> (("; " ++ s):code, ((x,v):env, v + size t))
    ".limit":"locals":ns -> chCode (".limit locals " ++ show (length ns))
    "invokestatic":t:f:ns | take 8 f == "runtime/" -> 
          chCode $ "invokestatic " ++ "runtime/" ++ t ++ drop 8 f ++ typesig ns 
    "invokestatic":f:ns  -> chCode $ "invokestatic " ++ cls ++ "/" ++ f ++ typesig ns 
    "alloc":ns           -> chCode $ "; " ++ s
    t:('_':instr):[";"]  -> chCode $ t ++ instr
    t:('_':instr):x:_    -> chCode $ t ++ instr ++ " " ++ look x
    "goto":ns            -> chCode $ "goto " ++ label ns
    "ifeq":ns            -> chCode $ "ifeq " ++ label ns
    "label":ns           -> chCode $ label ns ++ ":"
    ";":[] -> chCode ""
    _ -> chCode s
   where
     chCode c = (c:code,(env,v))
     look x   = maybe (error $ x ++ show env) show $ lookup x env
     typesig  = init . map toUpper . concat
     label    = init . concat
     size t   = case t of
       "d" -> 2
       _ -> 1

boilerplate :: String -> String
boilerplate cls = unlines [
  ".class public " ++ cls, ".super java/lang/Object",
  ".method public <init>()V","aload_0",
  "invokenonvirtual java/lang/Object/<init>()V","return",
  ".end method"]
\end{verbatim}
\normalsize
\newpage






\end{document}

\begin{verbatim}
\end{verbatim}

